\chapter{Conclusion}
\label{conclusion}

Through the lens of the Pescadero estuary in Northern California, this work has investigated hydrodynamics and salt transport in small, shallow bar-built estuaries, particularly those that are intermittently connected to the ocean. 

Observations and analysis of the Pescadero estuary in particular showed that while closed, this estuary remains salt-stratified and limited mixing is wind induced. With this background stratification, the transition from the closed to the open state either occurs rapidly with full mixing of the water column, or more slowly with incomplete mixing between fresher surface and saltier lower waters. 

While open to tidal influence, California's bar built estuaries are highly influenced by the nearshore wave environment. Like traditional estuaries, the water level within the estuary follows tidal ocean forcing. Unlike estuaries with deep inlets, the shallow mouth cuts off part of the low tide as the ocean retreats below the mouth. This perched mouth limits tidal velocities and does not allow ocean gravity waves to enter the estuary while permitting infragravity motions to pass through the inlet. High velocities induced by these infragravity motions are seen to be energetically important. On the large ebb the ocean retreats from the mouth, and nearshore wave forcing is cut off. Tidally, the salt field is  advected upstream but disconnects from the ocean and is removed primarily through shear entrainment with fresher water during spring tides. During neap tides, salt water may remain trapped in deep pools of the estuary for several tidal cycles. 

Calculating salt dispersion is a quantifiable way of estimating mixing and transport within estuaries. For the Pescadero estuary, salinity dispersion calculations reveal that in the highly stratified flow, a very different dispersion regime exists at low salinities than at high. Data suggest that processes in the Butano marsh contribute to the salinity-dependent dispersion as low salinity water farther upstream or higher in the water column can access marsh features that increase dispersion among these waters through trapping and vertical mixing. Higher salinity water deeper in the water column (or downstream) is bathymetrically restricted access to these features, and mixes very little. While impossible to conclusively point a finger at the process driving increased low-salinity dispersion, anthropogenic influence on the bar-built estuaries of North America has probably altered these dispersion regimes. Massive wetland loss in the state of California, where many bar-built estuaries sit amidst highly developed areas would eliminate a vegetative cause of dispersion. Logging and mining that has increased sedimentation of rivers, creeks, and estuaries, may alter dispersive processes as estuaries become shallower or channels migrate or infill on scales faster than naturally likely.  

In the face of dynamic estuarine changes by tides, freshwater, wind, sand transport, and human intervention, observations revealed that small amplitude tsunami forcing also has a dramatic effect on an ever-changing estuary. The ocean wide tsunami generated by the 2011 Tohok$\overline{\mathrm{u}}$ earthquake induced high velocity flow in the Pescadero estuary that mixed the water column much more than is typically seen. Access of the tsunami to the estuary was modulated by tides, and high velocities were on an infragravity timescale and modulated by the tsunami. This interesting cascading effect of variable frequency forcing suggests that the tsunami event can be framed as an energetic insertion of a forcing between usual infragravity and tidal periods.

Moving forward in time, the future of bar-built estuaries on the west coast of North America is imperiled by a host of anthropogenic threats. Sea level rise, limited freshwater, pollution, and land-use changes will likely jeopardize to the existence and health of small estuaries. These threats raise substantially more questions, and point to the need for better understanding and improved conservation practices to protect our coastal resources. 
