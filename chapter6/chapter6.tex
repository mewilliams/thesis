\chapter{Conclusion}
\label{conclusion}

Through the lens of the Pescadero estuary in Northern California, this work has investigated hydrodynamics and salt transport in small, shallow bar-built estuaries, particularly those that are intermittently connected to the ocean. 

\emph{Chapter 2 conclusions:}
The Pescadero estuary itself is ...... While closed, the estuary remains salt-stratified despite limited mixing by large wind events. The transition from the closed to the open state either occurs rapidly with full mixing of the water column, or more slowly where mixing between fresher surface and saltier lower waters may not occur. 

\emph{Chapter 3 conclusions:}
While in the open state, California's bar built estuaries remain highly influenced by the nearshore wave environment. Like traditional estuaries, the water level within the estuary follows tidal ocean forcing. Unlike estuaries with deep inlets, the shallow mouth cuts off part of the low tide as the ocean retreats below the mouth. This perched mouth limits tidal velocities and does not allow ocean gravity waves to enter the estuary while permitting infragravity motions to pass through the inlet. High velocities induced by these infragravity motions are seen to be energetically important. On the large ebb the ocean retreats from the mouth, nearshore forcing is cut off. The salt field is tidally advected upstream but disconnects from the ocean and is removed primarily through shear entrainment with fresher water. 

\emph{transient nature of open state}


\emph{Chapter 4 conclusions:} \emph{point to method to utilize traditional estuarine framework outside the bounds of which it is normally applied.}
Calculating salt dispersion is a quantifyable way of estimating mixing and transport within estuaries. Calculation for the Pescadero estuary, and reveal that in the highly stratified flow, very different dispersion \emph{happens} at low salinities than at high. Data suggest that processes in the Butano marsh contribute to \emph{this} as low salinity water farther upstream or higher in the water column had access to \emph{marsh things} which higher salinity water deeper in the water column (or downstream) is bathymetrically removed/not allowed access to.  \emph{what does this mean for wetlands? - either for existence/non-existence, or for ......}.  While impossible to \emph{(word that means really sure about something)} point a conclusive finger at the process driving increased low-salinity dispersion, anthropogenic influence in the bar-built estuaries of North America most certainly have altered these dispersion regimes... Either massive wetland loss in the state, where many bar-built estuaries sit amidst highly developed areas, or through \emph{processes like} logging and resulting sedimentation and \emph{channel migration on maybe faster timescales than geologically likely}... 

\emph{Chapter 5 conclusions:}
And finally in the face of dynamic estuarine changes by tides, freshwater, wind, sand transport, and human intervention, \emph{chapter 5} revealed that small-scale tsunami \emph{forcing} also \emph{makes things change a lot...}

Going forward, the bar-built estuaries on the west coast of North America face \emph{threats} from a host of anthropogenic \emph{things}... \emph{pollution, lack of water if/when too much is sucked up for agriculture, sea level rise, land-use changes}
