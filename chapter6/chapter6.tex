\chapter{Conclusion}
\label{conclusion}

Through the lens of the Pescadero estuary in Northern California, this work has investigated hydrodynamics and salt transport in small, shallow bar-built estuaries, particularly those that are intermittently connected to the ocean. 

\emph{Chapter 2 conclusions:}

\emph{Chapter 3 conclusions:}
While in the open state, California's bar built estuaries remain highly influenced by the nearshore wave environment. Like traditional estuaries, the water level within the estuary follows tidal ocean forcing. Unlike estuaries with deep inlets, the shallow mouth cuts off part of the low tide as the ocean retreats below the mouth. 


1322 2. The perched mouth limits tidal velocities and does not allow ocean gravity waves to
1323 enter the estuary while permitting infragravity motions to pass through the inlet.
1324 3. High velocities induced by these infragravity motions are energetically important.
1325 4. On the large ebb the ocean retreats from the mouth, nearshore forcing is cut o↵, and
1326 frictional control describes the velocity.
1327 5. The salt field is tidally advected upstream but disconnects from the ocean and is
1328 removed by shear.

\emph{transient nature of open state}


\emph{Chapter 4 conclusions:}
Calculating salt dispersion is a quantifyable way of estimating mixing and transport within estuaries. Calculation for the Pescadero estuary, and reveal that in the highly stratified flow, very different dispersion \emph{happens} at low salinities than at high. Data suggest that processes in the Butano marsh contribute to \emph{this} as low salinity water farther upstream or higher in the water column had access to \emph{marsh things} which higher salinity water deeper in the water column (or downstream) is bathymetrically removed/not allowed access to.  \emph{what does this mean for wetlands?}. In a state where ma

\emph{Chapter 5 conclusions:}
And finally, \emph{chapter 5} revealed that 
