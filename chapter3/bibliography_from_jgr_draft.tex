
%%%%%%%%
% BIBLIOGRAPHY?

\begin{thebibliography}
\bibitem[\textit{Baldock et al.}(2008)]{baldock_morphodynamic_2008}Baldock,
T. E., F. Weir, and M. G. Hughes (2008), Morphodynamic evolution of
a coastal lagoon entrance during swash overwash, \emph{Geomorphology},
95(3-4), 398\textendash 411, doi:10.1016/j.geomorph.2007.07.001.

\bibitem[\textit{Battjes}(1988)]{battjes_surf-zone_1988}Battjes, J. A. (1988),
Surf-zone dynamics, \emph{Annu. Rev. Fluid Mech.}, 20(1), 257\textendash 291,
doi:10.1146/annurev.fl.20.010188.001353. 

\bibitem[\textit{Beach and Sternberg}(1988)]{beach_suspended_1988}Beach,
R. A., and R. W. Sternberg (1988), Suspended sediment transport in
the surf zone: Response to cross-shore infragravity motion, \emph{Mar.
Geol.}, 80(1\textendash 2), 61\textendash 79, doi:10.1016/0025-3227(88)90072-2. 

\bibitem[\textit{Becker et al.}(2014)]{becker_water_2014}Becker, J.M.,
M.A. Merrifield, and M. Ford (2014), Water level effects on breaking
wave setup for Pacific Island fringing reefs, \emph{J. Geophys. Res.
Oceans}, 119, 914-932, doi:10.1002/2013JC009373

\bibitem[\textit{Behrens et al.}(2013)]{behrens_episodic_2013}Behrens,
D. K., F. A. Bombardelli, J. L. Largier, and E. Twohy (2013), Episodic
closure of the tidal inlet at the mouth of the Russian River \textemdash{}
A small bar-built estuary in California, \emph{Geomorphology}, 189,
66\textendash 80, doi:10.1016/j.geomorph.2013.01.017. 

\bibitem[\textit{Contardo and Symonds}(2013)]{contardo_infragravity_2013}Contardo,
S., and G. Symonds (2013), Infragravity response to variable wave
forcing in the nearshore, \emph{J. Geophys. Res. Oceans}, 118(12),
7095\textendash 7106, doi:10.1002/2013JC009430. 

\bibitem[\textit{Dodet et al.}(2013)]{dodet_wave-current_2013}Dodet, G.,
X. Bertin, N. Bruneau, A. B. Fortunato, A. Nahon, and A. Roland (2013),
Wave-current interactions in a wave-dominated tidal inlet, \emph{J.
Geophys. Res. Oceans}, 118, 1-19, doi:10.1002/jgrc.20146.

\bibitem[\textit{Elwany et al.}(1998)]{elwany_opening_1998}Elwany, M.
H. S., R. E. Flick, and S. Aijaz (1998), Opening and closure of a
marginal Southern California lagoon inlet, \emph{Estuaries}, 21(2),
246\textendash 254. 

\bibitem[\textit{Fortunato et al.}(2014)]{fortunato_morphological_2014}Fortunato,
A. B. et al. (2014), Morphological evolution of an ephemeral tidal
inlet from opening to closure: The Albufeira inlet, Portugal, \emph{Cont.
Shelf Res}., 73, 49\textendash 63, doi:10.1016/j.csr.2013.11.005.

\bibitem[\textit{Gale et al.}(2007)]{gale_processes_2007}Gale, E., C.
Pattiaratchi, and R. Ranasinghe (2007), Processes driving circulation,
exchange and flushing within intermittently closing and opening lakes
and lagoons, \emph{Mar. Freshwater Res.}, 58(8), 709, doi:10.1071/MF06121. 

\bibitem[\textit{Green and MacDonald}(2001)]{green_processes_2001}Green,
M. O., and I. T. MacDonald (2001), Processes driving estuary infilling
by marine sands on an embayed coast, \emph{Mar. Geol.}, 178(1\textendash 4),
11\textendash 37, doi:10.1016/S0025-3227(01)00188-8. 

\bibitem[\textit{Hanes et al.}(2011)]{hanes_waves_2011}Hanes, D. M.,
K. Ward, and L. H. Erikson (2011), Waves and tides responsible for
the intermittent closure of the entrance of a small, sheltered tidal
wetland at San Francisco, CA, \emph{Cont. Shelf Res}., 31(16), 1682\textendash 1687,
doi:10.1016/j.csr.2011.07.004. 

\bibitem[\textit{Hill}(1994)]{hill_fortnightly_1994}Hill, A. E. (1994), Fortnightly
tides in a lagoon with variable choking, \emph{Estuar. Coast. Shelf
Sci}., 38(4), 423\textendash 434, doi:10.1006/ecss.1994.1029. 

\bibitem[\textit{Largier and Taljaard}(1991)]{largier_dynamics_1991}Largier,
J. L., and S. Taljaard (1991), The dynamics of tidal intrusion, retention,
and removal of seawater in a bar-built estuary, \emph{Estuar. Coast.
Shelf Sci}., 33(4), 325\textendash 338.

\bibitem[\textit{MacCready and Geyer}(2010)]{maccready_advances_2010}MacCready,
P., and W. R. Geyer (2010), Advances in estuarine physics, \emph{Annu.
Rev. Mar. Sci.}, 2(1), 35\textendash 58, doi:10.1146/annurev-marine-120308-081015.

\bibitem[\textit{Malhadas et al.}(2009)]{malhadas_effect_2009}Malhadas,
M. S., P. C. Leit�o, A. Silva, and R. Neves (2009), Effect of coastal
waves on sea level in �bidos Lagoon, Portugal, \emph{Cont. Shelf Res}.,
29(9), 1240\textendash 1250, doi:10.1016/j.csr.2009.02.007.

\bibitem[\textit{Mei and Liu}(1993)]{mei_surface_1993}Mei, C. C., and
P. L. Liu (1993), Surface waves and coastal dynamics, \emph{Annu.
Rev. Fluid Mech.}, 25(1), 215\textendash 240, doi:10.1146/annurev.fl.25.010193.001243. 

\bibitem[\textit{Monismith}(2007)]{monismith_hydrodynamics_2007}Monismith, S. G.
(2007), Hydrodynamics of coral reefs, \emph{Annu. Rev. Fluid Mech.},
39(1), 37\textendash 55, doi:10.1146/annurev.fluid.38.050304.092125. 

\bibitem[\textit{Munk}(1950)]{munk_origin_1950}Munk, W. H. (1950), Origin
and generation of waves, \emph{Int. Conf. Coastal. Eng.}, 1(1), 1,
doi:10.9753/icce.v1.1. 

\bibitem[\textit{Ralston et al.}(2013)]{ralston_effects_2013}Ralston,
D. K., W. R. Geyer, P. A. Traykovski, and N. J. Nidzieko (2013), Effects
of estuarine and fluvial processes on sediment transport over deltaic
tidal flats, \emph{Cont. Shelf Res}., 60, Supplement, S40\textendash S57,
doi:10.1016/j.csr.2012.02.004. 

\bibitem[\textit{Ranasinghe and Pattiaratchi}(1999)]{ranasinghe_circulation_1999}Ranasinghe,
R., and C. Pattiaratchi (1999), Circulation and mixing characteristics
of a seasonally open tidal inlet: a field study, \emph{Mar. Freshwater
Res.}, 50(4), 281\textendash 290.

\bibitem[\textit{Ranasinghe and Patttiaratchi}(2003)]{ranasinghe_seasonal_2003}Ranasinghe,
R., and C. Pattiaratchi (2003), The seasonal closure of tidal inlets:
causes and effects, \emph{Coast. Eng. J.}, 45(4), 601\textendash 627,
doi:10.1142/S0578563403000919. 

\bibitem[\textit{Rydberg and Wickbom}(1996)]{rydberg_tidal_1996}Rydberg,
L., and L. Wickbom (1996), Tidal choking and bed friction in Negombo
Lagoon, Sri Lanka, \emph{Estuaries}, 19(3), 540\textendash 547, doi:10.2307/1352516. 

\bibitem[\textit{Schubert and Bokuniewicz}(1991)]{schubert_infragravity_1991}Schubert,
C. E., and H. J. Bokuniewicz (1991), Infragravity wave motion in a
tidal inlet, in \emph{Coastal Sediments \textquoteright 91: proceedings
of a Specialty Conference on Quantitative Approaches to Coastal Processes:
Seattle, Washington, June 25-27, 1991}, vol. 2, edited by N.C. Kraus,
K.J. Gingerich, and D.L. Kriebel, pp. 1434\textendash 1446, ASCE,
Seattle, WA. 

\bibitem[\textit{Sharples et al.}(2003)]{sharples_quantifying_2003}Sharples,
J., M. J. Coates, and J. E. Sherwood (2003), Quantifying turbulent
mixing and oxygen fluxes in a Mediterranean-type, microtidal estuary,
\emph{Ocean Dynam.}, 53(3), 126\textendash 136, doi:10.1007/s10236-003-0037-8. 

\bibitem[\textit{Simpson et al.}(2004)]{simpson_reynolds_2004}Simpson,
J. ., N. . Fisher, and P. Wiles (2004), Reynolds stress and TKE production
in an estuary with a tidal bore, \emph{Estuar. Coast. Shelf Sci}.,
60(4), 619\textendash 627, doi:10.1016/j.ecss.2004.03.006. 

\bibitem[\textit{Soulsby}(1983)]{soulsby_bottom_1983}Soulsby, R. L. (1983),
The bottom boundary layer of shelf seas, in \emph{Physical Oceanography
of Coastal and Shelf Seas, }vol. 35, edited by B. Johns, pp. 189\textendash 266,
Elsevier. 

\bibitem[\textit{Stigebrandt}(1980)]{stigebrandt_aspects_1980}Stigebrandt,
A. (1980), Some aspects of tidal interaction with fjord constrictions,
\emph{Estuar. Coast. Mar. Sci.}, 11(2), 151\textendash 166, doi:10.1016/S0302-3524(80)80038-7. 

\bibitem[\textit{Uncles et al.}(2014)]{uncles_infragravity_2014}Uncles, R.
J., J. A. Stephens, and C. Harris (2014), Infragravity currents in
a small r�a: Estuary-amplified coastal edge waves?, \emph{Estuar.
Coast. Shelf Sci}., doi:10.1016/j.ecss.2014.04.019. 

\bibitem[\textit{Wolanski et al.}(2004)]{wolanski_undular_2004}Wolanski,
E., D. Williams, S. Spagnol, and H. Chanson (2004), Undular tidal
bore dynamics in the Daly Estuary, Northern Australia, \emph{Estuar.
Coast. Shelf Sci}., 60(4), 629\textendash 636, doi:doi:10.1016/j.ecss.2004.03.001. 

\end{thebibliography}