\chapter{Tidally discontinuous estuarine hydrodynamics}
\label{ch3}



%***********************************************************************************************************************************************************************************************
% INTRODUCTION

\section{Introduction$^1$}

\footnotetext[1]{This chapter has been submitted to the \emph{Journal of Geophysical Research: Oceans} as ``Tidally discontinuous ocean forcing in bar-built estuaries: the interaction of tides, infragravity motions, and frictional control'' by M.E. Williams and M.T. Stacey. At time of submission of this dissertation, the manuscript was in review.}

California's coast is dotted with small estuaries draining the pristine,
agricultural, or highly urbanized watersheds found along the state's
coastline. Similar to bar-built estuaries found in Mediterranean climates
on wave-dominated coasts worldwide, the inlets of these estuaries
may become choked closed by sand for extended periods of time (e.g.
\cite{elwany_opening_1998,hanes_waves_2011,fortunato_morphological_2014,ranasinghe_seasonal_2003}).
Nearshore sediment transport processes compete with ebb tidal and
river flow to set the open or closed state of the inlet \parencite{behrens_episodic_2013}.
While open, the inlets of intermittently closed estuaries remain highly
constricted, altering tidally-driven hydrodynamics and salt transport
in these systems.

Some studies in temporarily open/closed estuaries (TOCE) in South
Africa, intermittently closed and open lakes and lagoons (ICOLLs)
in Australia, and intermittently closed estuaries in California sheds
some insight on the hydrodynamics in these systems. Tidal forcing
is typically highly attenuated in these estuaries (e.g. \cite{ranasinghe_circulation_1999,gale_processes_2007}).
Salt movement in the South African Palmiet estuary occurs as ocean
water moves up estuary as a density current, becomes trapped in deep
pools of the estuary and then is removed by shear-induced mixing from
above \parencite{largier_dynamics_1991}. Similar salt-wedge or two-layer
structure has been observed in the open state of other intermittently
closed estuaries (e.g. \cite{sharples_quantifying_2003}). Overall,
hydrodynamic observations are limited.

Many of the previous observations have looked at overall circulation
patterns of an estuary on seasonal scales, leaving the effects of
short term processes like a changing wave climate or spring-neap
cycle under described. In contrast to large estuaries with deep
inlets, the shallow mouth of open-state intermittently open estuaries
alters tides and causes ocean waves to break outside the inlet. The
effects of these features on estuarine hydrodynamics have not been
characterized. And, though estuaries with closing inlets exist in
various parts of the world, those on the west coast of North America
have received little attention. 

Because sparse hydrodynamic characterizations of open-state hydrodynamics
of intermittently closed estuaries exist, we must look to processes
important to these systems in other contexts. The mouth or inlets
of these systems cross beaches with active wave environments (see
\cite{mei_surface_1993,battjes_surf-zone_1988}). Wave setup observed in lagoonal
estuaries (e.g. \cite{malhadas_effect_2009}) may be analogous
to similar setup on coral reef lagoons (e.g. \cite{becker_water_2014})
and wave-driven flow on these reefs \parencite{monismith_hydrodynamics_2007} may also
be important in lagoonal estuaries. 

Here we present results of extensive field observations in the Pescadero
estuary of Northern California aimed at characterizing hydrodynamics
of small bar-built estuaries with intermittent connections to the
ocean. 


%***********************************************************************************************************************************************************************************************
% EXPERIMENTAL SITE AND OVERVIEW OF CONDITIONS

\section{Experimental site and overview of conditions}

As part of a longer-term study to analyze dominant drivers of hydrodynamics
and salt transport in California's small estuarine systems, extensive
field measurements were made in the Pescadero estuary.


\subsection{Site description }

The mouth of the Pescadero estuary sits on the California coast midway
between the Golden Gate and the northern extent of Monterey Bay (Figure
\ref{f1_maps}a). The mouth is constrained to the south by a rocky
cliff. To the north, the Pescadero State Beach is the southern extent
of approximately 10 km of sandy beach with three other small inlets.
The inlet is approximately 100 m long, and opens into a shallow lagoonal
estuary (Figure \ref{f1_maps}b). 


\begin{figure}[tp]
\begin{subfigure}{.5\textwidth}
	\centering
	\includegraphics[width=.8\linewidth]{chapter3/figures/fig1a.pdf}
	%\caption{1a}
	\label{fig:sfig1a_coastline}
\end{subfigure}
\begin{subfigure}{.5\textwidth}
	\centering
	\includegraphics[width=\linewidth]{chapter3/figures/figure1b_ed2.png}
	%\caption{1b}
	\label{fig:sfig1b_pdomap}
\end{subfigure}
\caption{(a) The Pescadero estuary sits on the California coast between Monterey
and San Francisco Bays. The location of sensors referenced for tidal
and wave conditions are shown. (b) The coastline and depth at high
tide in the Pescadero estuary. Pescadero Creek flows into the estuary
from the northeast, and Butano Creek from the south. The narrow channel
connects the estuary to a northern pond through culverts.}
\label{f1_maps}
\end{figure}


The lower estuary has a maximum high tide depth of approximately 2.7
m, in the deep channel 350 m upstream from the mouth (Figure \ref{f1_maps}b) and maximum spring tidal range of 1.4 m. The mouth of this estuary
typically closes with sand between August and October due to nearshore
sediment transport and reopens with the start of the California rainy
season in November. During the dry water year 2012, the mouth closed and opened
several times, not remaining open for an extended period until early
March 2012. Results presented here are exclusively from the open inlet
state of the estuary. 

The Pescadero creek arrives to the ocean through the largest wetland
complex between the San Francisco Bay and Elkhorn Slough on Monterey
Bay. Two creeks, the Pescadero and Butano, connect 700 m upstream
of the mouth. The Butano Creek meanders through extensive salt marsh
while the Pescadero creek remains more channeled and riverine up to
the confluence. On the Pescadero Creek upstream of the confluence,
a channel connects a shallow pond to the estuary through dilapidated
culverts\emph{, }installed as part of a restoration project. The presence
of the North Pond likely influences the inlet dynamics by increasing
the tidal prism, but due to very limited connectivity through small
culverts, we ignore its effect on hydrodynamics.

The Pescadero watershed covers 210 km\textsuperscript{2} in the coastal
Santa Cruz mountains. A USGS gauge (11162500), 8.5 km upstream of
the mouth captures captures 57 \% of the watershed. Unaccounted for
freshwater diversions exist due to agriculture in the watershed, but
freshwater flow into the estuary, \emph{$Q_{R}$, }is approximated
by scaling the gauged freshwater discharge, $Q_{R,G}$, by the entire
watershed area. The rainfall dominated coastal watershed is flashy during the wet winter, and dry during the summer season, reflecting the Mediterranean-climate seasonality of the region.


\subsection{Field measurements and data processing}


\subsubsection{Measurements}

Measurements were made in the lower Pescadero estuary from 17 April
- 5 June, 2012 with high frequency velocity measurements limited to
a shorter subset of this time. A suite of instruments measuring temperature,
salinity, pressure, and velocity were distributed longitudinally and
vertically in the water column (Figure \ref{f1_maps}b).

An acoustic Doppler velocimeter (ADV; Nortek Vector) was installed
on 18 April recording in 512 s bursts every 15 minutes at a sampling
rate of 8 Hz at the near mouth (NM) station. The instrument's sampling
volume was positioned 20 cm above the bed. A collocated CTD recorded
salinity and pressure at a frequency of 1 Hz. The deep channel (DC)
station had a bed-mounted acoustic Doppler current profiler (ADCP;
RDI Workhorse Monitor 1200 kHz). The height of the ADCP plus blanking
distance meant that the first depth cell was located 71 cm above
the bed at DC. The bed at DC is approximately 19 cm below MLLW. Co-located
with these velocity instruments were bed, surface, and mid-column
CTDs. A CTD was moored upstream in the Pescadero Creek (PC) at a height of 20 cm above the bed, sampling every 30 s. Previous attempts to collect data in the mouth, even for a short duration (under 24 hours), left instruments
buried in the sand, so proximity of moorings to the mouth was limited. 

Tidal, freshwater, and ocean wave condition measurements were obtained
from NOAA and USGS gauges (cf. \ref{sub:ext_cond}).




\subsubsection{Data processing\label{sub:Data-processing}}

Estuarine depth measurements were made using several bed-mounted CTDs
with the atmospheric pressure offset calculated from measurements
made at the coastal Half Moon Bay airport (KHAF). Estuarine significant
wave height, $H_{s,e}$ was calculated as $H_{s,e}=\sqrt{4m_{o}}$
where $m_{o}$ is the variance of the a 20 min moving average signal
subtracted from the depth signal to remove the tidal component. Surface
oscillations due to waves and bores were recorded, but oscillations
considered are of low frequency (period $>$ 30 s) so depth attenuation
of the wave signal should be minimal. Some slight attenuation of the
signal may occur, especially at high tide, and this would result in
a slight underestimation of the estuarine significant wave height.

Velocity measurements (ADV and ADCP) were rotated in to principle
coordinates (u,v) based on a linear fit of each instrument's dataset.
Tidal velocities,$\overline{<u>}$, are approximated by the depth
averaged 20 min moving average of ADCP velocity measurements. The
moving average is intended to remove the higher frequency velocity
signal. Due to the blanking distance above the bed-mounted ADCP, 'depth-averaged'
velocities are missing flow at the bottom of the water column. While
this missing data is likely relevant, the study was limited to available
instruments commonly deployed in much deeper water.

The definition of the infragravity band varies by author, with a lower end period between 20 and 30 s and an upper end period between 4 and 8 min. Here we use the gravity wave band within a period of 1 s to 30 s and infragravity
period from 30 s to 5 min, consistent with Munk \parencite*{munk_origin_1950}.

To observe general salt dynamics, CTD sensor data were interpolated
onto an x-z cross section of the estuary. Vertical distribution of
CTDs at DC was concentrated in the upper water column because of an
instrument failure. We know that sharp gradients exist in the vertical
so to observe salinity in the water column with time (Figure \ref{f11_SUtz}a-c),
the vertical profile from interpolated x-z data (Figure \ref{f10_Sxz}a-f)
was used so that upstream and downstream sensors contribute to the
estimated profile. Any interpolation method makes stability analysis
difficult because the pycnocline may be 10 cm thick and our measurements
lack the fine resolution needed to calculate an accurate Brunt\textendash V\"ais\"al\"a
frequency, $N$.


\subsection{Overview of conditions\label{sub:ext_cond}}

General conditions are based on a streamflow gauge maintained by USGS,
and a tide gauge and wave buoy maintained by NOAA (Figure \ref{f2_QTHs}). 

In California's Mediterranean climate, precipitation is seasonal and
does not usually occur during the summer months. The last notable
rainstorm of water year 2012 occurred between April 11 and 15, so
freshwater was present in the Pescadero estuary during the study,
but no rain was recorded (Figure \ref{f2_QTHs}a). 


\begin{figure}[tp]
\includegraphics[width=\linewidth]{chapter3/figures/figure2_externalsensors.pdf}

\protect\caption{Conditions during field deployment. The freshwater streamflow into
the estuary (a) is scaled up by watershed area from the USGS Pescadero
gauge (11162500). San Francisco tidal water level data (b) is given
by NOAA measurements at Crissy Field (9414290). The NOAA NDBC buoy
(46012) gives ocean significant wave height (c) 40 km WNW of the Pescadero
estuary. Gray area indicates period that sensors were collecting data.
Higher frequency measurements were obtained during the first 5 days
to three weeks of the experiment, but CTD and some velocity data were
collected for the entire shaded period. \label{f2_QTHs}}
\end{figure}

The California coast experiences a mixed semi-diurnal tide. The tidal
range is between 1.1 m and 2.7 m at the closest NOAA gauge in San
Francisco (NOAA 9414290, Figure \ref{f1_maps}a, Figure \ref{f2_QTHs}b).
The study encompassed spring-neap variation.

A NOAA National Data Buoy Center (NDBC 46012) wave buoy in
208.8 m depth approximately 40 km WNW of the Pescadero estuary gives
a proxy for the wave climate experienced at the coast. Significant
wave heights offshore varied between 0.6 m and 4.8 m during the study
(Figure \ref{f1_maps}a, \ref{f2_QTHs}c). 




%***********************************************************************************************************************************************************************************************
% ANALYSIS OF HYDRODYNAMICS

\section{Analysis of hydrodynamics}

Estuaries are driven by ocean tides, but lagoonal estuaries often
have an attenuated tidal signal compared to the ocean coast primarily
due to friction losses (e.g. \cite{rydberg_tidal_1996}).
By comparing the tidal ocean water level with the tidal estuarine
water level in the Pescadero estuary (Figures \ref{f3_HUHse}a, \ref{f4_wl_schematic}),
a picture of tidally-varying estuarine-ocean connection emerges. Tidal
attenuation in the Pescadero estuary occurs only on the large ebb
of the semidiurnal mixed tide. Throughout the rest of the tidal cycle,
the ocean and estuarine water levels are nearly the same.


\begin{figure}
\centering
\includegraphics{chapter3/figures/figure0_schematic}

\protect\caption{For most of the mixed semidiurnal tidal cycle, the estuary and ocean
water levels are approximately the same, allowing for connection between
the nearshore and estuarine environments. As the tide falls to it's
daily low low state, the ocean retreats below the mouth of the perched
estuary.\label{f4_wl_schematic}}
\end{figure}

\begin{figure}[tp]
\includegraphics[width=\linewidth]{chapter3/figures/figure3_3day.pdf}

\protect\caption{Time series of water level (a), depth-averaged velocity (b), and estuarine
significant wave height (c). Ocean water level variations (a - gray
line) due to tides are much greater than the estuarine water level
changes (a - black and red line). Depth-averaged velocity (b) at DC
shows a sinusoidal velocity and a frictionally controlled velocity
during different phases of the tide. The presence of water surface
oscillations is determined by an estuarine significant wave height
(c). A threshold of 1 cm determines the presence of ``waves'' and
the water level (a) and depth-averaged velocity (b) time series have
been color-coded to presence of of this infragravity forcing (black
- nearshore forced, red - no infragravity oscillations). \label{f3_HUHse}}
\end{figure}

This two-phase connection where for 16-17 hours of the 24 h tidal
cycle the nearshore and estuary are connected but for 7-8 hours the
bodies of water separate sets the hydrodynamic conditions and determines
different forcing during these distinct phases of the tide (Figure
\ref{f4_wl_schematic}). Here we examine processes during each phase
and describe how this tidally varying connection affects estuarine
dynamics. 


\subsection{Nearshore-estuarine connection}

Observations show that while the estuary water level and and ocean
water level are the same, the estuary and nearshore are connected
and tidal forcing is accompanied by velocity oscillations attributed
to infragravity motions. 


\subsubsection{Tidal timescales}

Looking at tidal timescales approximated by the depth average of the
20 min moving average ADCP dataset, $\overline{<u>}$, for the small
spring tide conditions on 23 to 26 April, 2012, tidal velocities oscillate
between -30 cm s\textsuperscript{-1} and +20 cm s\textsuperscript{-1}.
Measured ebb-dominance may be influenced by the two-layer nature of
the flow and missing data in the lower 71 cm of the water column where
positive velocities would be observed on the flood. Maximum estuarine
water level leads peak flood tidal velocities by 1-2 hours (Figure
\ref{f3_HUHse}a,b). 


\subsubsection{Infragravity timescales}

On top of the tidally induced velocities and changes in water level,
oscillating motions are observed during the nearshore-estuarine connection
phase of the tides in both velocity and as a surface expression (Figure
\ref{f5_adv}). These oscillations in $\eta$ and $u$ are attributed
to the presence of infragravity motions in the nearshore, due to their
observed period within the infragravity band of 30 s to 5 min (Figure
\ref{f6_spectra}). No instrumentation was deployed in the nearshore
during this field campaign, but infragravity motions are consistently
present on sandy beaches \parencite{contardo_infragravity_2013} and we expect
them to be present here. By characterizing the wave-like surface oscillations
as an estuarine significant wave height, $H_{s,e}$ (Figure \ref{f3_HUHse}c),
we can use this as a metric for the presence of infragravity motions
in the estuary and thus as a proxy for nearshore-estuarine connection.
When $H_{s,e}$ is above 1 cm (Figure \ref{f3_HUHse}, black lines/points),
the relationship between ocean tides and estuarine tides are nearly
1:1, and tidal velocities obey an expected sinusoidal shape in relation
to tidal water level. When $H_{s,e}$ falls below this threshold,
the relation between the estuarine and ocean water level diverges,
and a disconnected state persists (Figure \ref{f3_HUHse}a,c).



%%%%% NEEDS SUBFIGURES!!%%%%%%%%%%
\begin{figure}[tp]
\includegraphics[width=\linewidth]{chapter3/figures/fig4_onetide_ADV.pdf}
\includegraphics[width=\linewidth]{chapter3/figures/fig4_zoomed_ADV.pdf}
\protect\caption{Measurements from an ADV measuring 20 cm above the bed at the NM station.
(a) is the pressure record and (b) shows u and v components of velocity
where u is principle flow determined by the 7.5 day dataset, and v
is lateral flow orthogonal to u. The remaining plots (c-f) show two
bursts when the tide was flooding (c,e) and ebbing (d,f) at approximately
the same water depth. During flood large velocity oscillations occur
with the surface motions. On ebb, these wave-like oscillations are
cutoff in both pressure and velocity. The first 25 s of each pressure
burst has been removed due to apparent aliasing. \label{f5_adv}}
\end{figure}


\begin{figure}[tp]
\centering
\includegraphics[width=\linewidth]{chapter3/figures/fig5_vps_U_P_edited.pdf}

\protect\caption{Variance preserving spectra of (a) depth (1 Hz CTD measurement), (b)
u, and (c) v (8 Hz ADV measurements). Where data exists, spectra are
computed on each burst. The white dashed lines indicate the infragravity
frequency limits of $\frac{1}{300}$ Hz to $\frac{1}{30}$ Hz. The
CTD was always submerged but the ADV came out of the water at some
low tides. \label{f6_spectra}}
\end{figure}

Infragravity motions, aside from suggesting that the ocean and nearshore
water levels are high enough to maintain a connection with the estuary
through the shallow mouth, induce high velocity oscillations. On an
incoming tide, infragravity motions in the nearshore cause bores to
propagate into and up the estuary, and in phase with these bores are
velocity oscillations larger than the tidal velocities (Figure 4c,e).
These oscillations are consistent during the nearshore-estuarine connected
phase of the tide, only shutting off when the nearshore water level
retreats from the mouth (Figure \ref{f3_HUHse},\ref{f6_spectra}). 

Gravity waves of a period 1 s to 30 s break in the nearshore and this
energy is filtered out by the perched mouth (Figure \ref{f6_spectra}).
In the small Pescadero estuary, infragravity surface motions were
observed at all moorings, and the velocity signature was present at
all locations where velocity was measured.


\subsection{Estuary-nearshore disconnection and frictional control}

As the ocean water level falls due to the outgoing tide, the estuarine
water level also retreats, but at a much slower rate (Figure \ref{f3_HUHse}d).
The shallow mouth bathymetrically constrains the flow of water out
of the estuary, and infragravity motions do not cause sufficient setup
in the nearshore to overcome the water level difference and infragravity
motions in the estuary are shut off. 

Infragravity oscillations persistent on the flood tide are not present
at the same estuarine water level on the large ebb (Figure \ref{f5_adv}d,f).
The shutdown of these motions occurs both in water level and velocity
(Figure \ref{f6_spectra}), further indication that the motions are
externally forced.

In this state where the estuary is disconnected from nearshore forcing,
the depth-averaged tidal velocity is no longer sinusoidal, but has
a maximum ebb velocity as the infragravity motions are shutting off
(cf \ref{sub:Transitions}) and then slows. The shape of the velocity
curve, the change in the rate of estuarine water level fall, and visual
observations suggest that the estuary reaches a frictionally controlled
state. Similar conditions are observed on tidal flats at low tide
(e.g. \cite{ralston_effects_2013}). The analytical description
of frictionally controlled velocity can be arrived at starting with
the incompressible Reynolds-averaged Navier-Stokes (RANS) equations:

\begin{eqnarray}
\frac{\partial u_{i}}{\partial t}+u_{j}\frac{\partial u_{i}}{\partial x_{j}} & = & -\frac{1}{\rho_{0}}\frac{\partial p}{\partial x_{i}}+\frac{\partial}{\partial x_{j}}\left(\nu_{T}\frac{\partial u_{i}}{\partial x_{j}}\right)-g\delta_{i3}\nonumber \\
\frac{\partial u_{i}}{\partial x_{i}}=0
\end{eqnarray}
The x-momentum RANS equation along a bed of slope angle $\theta$
is 

\begin{eqnarray}
 & \frac{\partial u}{\partial t}+u\frac{\partial u}{\partial x}+v\frac{\partial u}{\partial y}+w\frac{\partial u}{\partial z} & =\nonumber \\
 & -\frac{1}{\rho_{0}}\frac{\partial p}{\partial x}+\frac{\partial}{\partial x}\left(\nu_{_{T}}\frac{\partial u}{\partial x}\right)+\frac{\partial}{\partial y}\left(\nu_{_{T}}\frac{\partial u}{\partial y}\right)+\frac{\partial}{\partial z}\left(\nu_{_{T}}\frac{\partial u}{\partial z}\right)-g\sin\theta,\label{eq:rans_x}\\
\nonumber 
\end{eqnarray}
where $\nu_{_{T}}$ is turbulent viscosity. In the frictionally controlled
state, the flow is assumed to be fully developed, steady state, two-dimensional,
and the water surface slope is assumed to be the same as the bed slope.
Using these assumptions, equation \ref{eq:rans_x} reduces to:

\begin{eqnarray}
0 & = & \frac{\partial}{\partial z}\left(\nu_{_{T}}\frac{\partial u}{\partial z}\right)-g\sin\theta\label{eq:simplified}
\end{eqnarray}
Depth-averaging ($\overline{u}=\frac{1}{h}\int_{0}^{h}u\mathrm{d}z$)
equation \ref{eq:simplified} with bed stress ($\nu_{_{T}}\frac{\partial u}{\partial z}\Bigr|_{z=0}=\frac{\tau_{b}}{\rho}$)
and free surface ($\nu_{_{T}}\frac{\partial u}{\partial z}\Bigr|_{z=h}=0$)
boundary conditions where $\tau_{b}$ is bed stress and $z$ is positive
upward from the bed gives

\begin{eqnarray}
0 & = & \frac{\tau_{b}}{h\rho}-g\sin\theta\label{eq:fc_w_taubed}
\end{eqnarray}
Using a friction velocity ($u_{*}$) parameterization, $\tau_{b}=\rho u_{*}^{2}$,
$u_{*}^{2}=c_{_D}\overline{u}^{2}$, and the small-angle approximation
of $\sin\theta\sim S$, where $c_{_D}$ is a bed drag coefficient that
accounts for roughness due to bed material and bed-forms, $S$ is
the bed slope, $\overline{u}$ is a depth-averaged velocity gives

\begin{eqnarray}
c_{_D}\overline{u}^{2} & = & gSh\label{eq:frictionalcontrol}
\end{eqnarray}
 Given relatively constant values of $S$ and $c_{_D}$, the depth-averaged
velocity $\overline{u}\sim h^{\frac{1}{2}}$. This describes the shape
of the velocity curve during this frictionally controlled state (Figure
\ref{f3_HUHse}b, \ref{f8_ltHU}b,d).


\subsection{Longer term variability}

Within the open-mouth estuarine conditions observed during the field
experiment, varying wave conditions were experienced and the spring-neap
cycle was represented. The effects of this longer term variability
is shown to alter the processes occurring on tidal and infragravity
timescales. 


\subsubsection{Tidal timescale processes}

General semidiurnal forcing within the Pescadero estuary is due to
ocean tides, but even neglecting the long friction ebb, the variation
in tidal amplitude within the estuary is highly non-linear and cannot
be explained by ocean tidal forcing alone. 


%%%%% NEEDS SUBFIGURES!!%%%%%%%%%%
\begin{figure}
\begin{subfigure}{.666\textwidth}
	\centering
\includegraphics[width=\textwidth]{chapter3/figures/figure7_alternate.pdf}
\end{subfigure}
\begin{subfigure}{.333\textwidth}
\includegraphics[width=\textwidth]{chapter3/figures/figure7d_wavesetup_swh_vs_setup.pdf}
\end{subfigure}
\protect\caption{The relation between ocean water level and estuarine water level depends on ocean wave conditions for much of the tide. By subtracting the
semidiurnal high tide in San Francisco (a, gray line) from the high
tide water level in Pescadero (a, black/red line), the resulting setup
(b) is in agreement with the size of ocean waves as given by the deep
water wave buoy (c). This relationship is illustrated in (e) where
figures (b) and (c) are plotted against each other on a time interpolant.
An increase of 1 m in ocean waves causes a 10 cm increase in estuarine
water level setup within the estuary. The amplitude of estuarine wave
height also increases with larger waves (d) with an approximate 3
cm increase in the maximum estuarine wave height for every 1 m increase
in ocean significant wave height (f). \label{f7_setup}}
\end{figure}


The tidal water level variation in the Pescadero estuary is shown
to be an attenuated but wave-amplified version of the ocean tides.
As observed in other coastal lagoons (e.g. \cite{malhadas_effect_2009}),
larger ocean waves cause setup in the nearshore, resulting in higher
tides within the estuary. Subtracting the maximum estuarine water
level from the maximum tidal elevation in San Francisco to remove
effects of phasing we see linear agreement between significant wave
heights in the coastal ocean and increased estuarine sea level (Figure
\ref{f7_setup}). During our field measurements, a one meter increase
in significant wave height corresponded to a 0.1 m increase in tidal
elevation. This relationship seems to hold while the estuary is ocean-forced. 

Wave setup should also affect estuarine hydrodynamics at low low tide
as the ocean water level where the nearshore and estuary disconnect
would be lower. These effects are not separable from effects of erosion
and accretion of the bed of the inlet without high resolution surveying
data. 

Larger waves in the ocean also correspond with larger waves within
the estuary. As the ocean significant wave height increases, so does
the tidally-modulated estuarine significant wave height (Figure \ref{f7_setup}d).
Comparing peak $H_{s,e}$ values to the ocean significant wave height
$H_{s,o}$, we see a 1 m increase in ocean waves increases the maximum
estuarine significant wave height by 3 cm. 

%%%%%%% NEEDS SUBFIGURES!!%%%%%%%%%%
\begin{figure}
\centering
\includegraphics{chapter3/figures/figure8_u_h.pdf}
\includegraphics[width=0.7\linewidth]{chapter3/figures/f_extra_FC_cd.pdf}

\protect\caption{The general trend of oscillation between ocean/wave forced and non-wavy
persists through the spring-neap cycle, shown by three weeks of estuarine
water level (a) and depth-averaged velocity (b) data. Depth and velocity
data are plotted against each other in (c) and (d) for the spring
tides (28 April to 4 May is excluded). The frictionally controlled
regime is clear during the spring by the shape of $\overline{u}$,
plotted with $h^{\frac{1}{2}}$ in (d), but with higher low tide during
the neap, the ebb may not be long or ocean water level low enough
to reach a frictional state. \label{f8_ltHU}}
\end{figure}


With this alteration of tidal timescale water levels by wave setup,
either the duration of high tide or the tidal velocities would have
to increase as the amount of water entering and leaving the water
increases. Both cases seem to exist in this dataset. The tidal
water level on 27 April is distorted in alignment with ocean significant
wave heights of 4 m (Figure \ref{f8_ltHU}a). Here the tidal signal
was transitioning into a neap, so velocities and water level should
have been decreasing, but water level increased and the tidal velocity
was roughly as high as the previous day. A bigger jump in wave setup
on 2 May corresponded with an increase in velocities. In general,
higher flood velocities occur with higher water levels whether the
higher water level is attributed to the spring tides or wave setup. 

Over the longer term, the spring tides induce higher tidal velocities
than the neap tides (Figure \ref{f8_ltHU}). Wave shut off occurs
consistently on the large ebb of the semidiurnal spring or neap tide
(Figure \ref{f8_ltHU}, red lines). The shape of the depth-averaged
velocity with time is consistent with our understanding of frictional
control for spring tide conditions (Figure \ref{f8_ltHU}b,d). During
the neap tide, the frictionally dominated velocity-water level relationship
of $\overline{u}\sim h^{\frac{1}{2}}$ is not clear. 


\subsubsection{Infragravity time scale processes}

On time scales representative of higher frequency infragravity motions,
an increased wave climate in the ocean corresponds to an increased
wave climate within the estuary (Figure \ref{f7_setup}d)\emph{, }and corresponding velocity oscillations are greater.




%***********************************************************************************************************************************************************************************************
% DISCUSSION

\section{Discussion}


\subsection{Frictionally controlled and ocean influenced States\label{sub:disc_FC_tides}}

Tidal velocities and water levels in the estuary are driven by the
tidal amplitude in the ocean for most of the semidiurnal tide, and
then exhibit frictional dependence as the ocean water level retreats
below the perched estuarine mouth. 

This frictional control period of the tidal cycle is highlighted by
the $\overline{u}\sim h^{\frac{1}{2}}$ relationship (Figure \ref{f8_ltHU}d)
and accompanied by the shutoff of infragravity motions. Frictional
control may exist in other parts of the shallow estuary, but is visually
most pronounced at the mouth. To accurately calculate a coefficient
of drag, $c_{_D}$, for the frictionally controlled state, accurate
survey data and measurements in the mouth are needed. We can, however,
constrain the drag coefficient with some estimations. Assuming the
water surface slope matches the bed slope at the mouth and is flat
across the the lower estuary and no hydraulic jumps occur between
DC and the mouth (Figure \ref{f4_wl_schematic}), by conservation
of mass at the mouth (subscript m) and DC station (subscript DC),
\begin{eqnarray}
\overline{u}_{m} & = & \frac{\overline{u}_{DC}A_{DC}}{h_{m}A_{m}},\label{eq:consmass_udc_um}
\end{eqnarray}
where $A\mbox{}$ is cross-sectional area of the channel. Taking the
DC station cross section to be triangular, $A_{DC}=\frac{1}{2}b_{DC}h_{DC}$
where \textbf{$b\mbox{}$} is the width of the estuarine or inlet
cross section, and taking the mouth to be rectangular, $A_{m}=b_{m}h_{m}$.\emph{
}The mouth accretes and erodes regularly, but is based on the depth
at NM, we estimate the bed at the mouth to be $h_{m}=h_{DC}-0.8$
m. Estimating the bed slope to be $S$ as 0.001 to 0.01 to Characteristic
early frictional control values of $\overline{{u}}{}_{DC}=-0.3$ m
s\textsuperscript{-1} and $h_{DC}=1.45$ m give a range of $c_{_D}$
from 0.014 to 0.14. Characteristic late frictional control values
of $\overline{{u}}{}_{DC}=-0.2$ m s\textsuperscript{-1} and $h_{DC}=1$
m give a range of $c_{_D}$ from 0.002 to 0.02.

A typical value for $c_{_D}$ of a sandy bed is 0.004 \parencite{soulsby_bottom_1983}
and this value has been observationally recorded in inlets to tidal
lagoons \parencite{rydberg_tidal_1996}. Bedforms at the mouth of
the Pescadero estuary probably increase the drag above $c_{_D}$ =
0.004, but do not probably account for the very high drag suggested
early in the ebb. The changing drag coefficient may be a function
of several factors: (1) sediment transport at the mouth creates ripples
and waves which are constantly changing, (2) early frictional control
may be set by the inlet slope while later the shallow embayment of
the estuary may further control the flow, (3) what appears to be frictional
control early in the ebb based on the apparent $\overline{u}\sim h^{\frac{1}{2}}$
relationship may actually be a control set by the ocean water level
and the water slope may not yet match the bed slope, and (4) these
measurements are rough approximations. Surveying and further measurements
at the mouth would provide insight into the applicability of the frictional
control framework. 

This tidal control contrasts with limited tidal amplitudes in other
estuaries caused by tidal choking. Bed friction coupled with Bernoulli
acceleration explains water level differences between the ocean and
lagoon in tidally choked systems (e.g. \cite{rydberg_tidal_1996,hill_fortnightly_1994})
 based on the formulation \emph{$\left|h_{o}-h_{i}\right|=\left(1+\frac{c_{_D}BL}{A}\right)\frac{u^{2}}{2g}$}
\parencite{stigebrandt_aspects_1980}. This formulation expects water
surface slope above the inlet to be independent of the bed slope,
a condition not observed in the Pescadero estuary on the large ebb.
Furthermore, in the Pescadero estuary, a very short channel between
the ocean and the estuary limits the frictional effects on the rising
tide and slow velocities make the Bernoulli acceleration term very
small. 

The tidal oscillation between estuarine connection to the nearshore
and frictional control exists on a background of wave setup. Under
larger wave climates, the transition to frictional control may occur
at a slightly higher estuarine water level than the ocean wave climate
is smaller, and this we attribute to the influence of wave setup on
the lagoon. This setup is likely tidally dependent as seen in coral
reef flats \parencite{becker_water_2014}.


\subsection{Infragravity waves}

The presence of infragravity waves in sandy inlets has been previously
observed\emph{ \parencite{schubert_infragravity_1991}}, but recently has
received renewed attention. Forcing has been measured as currents
up to 40 cm s\textsuperscript{-1} in the R�a de Santiuste in Northern
Spain \parencite{uncles_infragravity_2014}. Work focused on wave-current interaction
in the inlet to the Portuguese Albufeira lagoon noted the presence
of velocities on an infragravity timescale equal to 50\% of the tidal
velocities \parencite{dodet_wave-current_2013}. Elsewhere in California, surface
oscillations on an infragravity timescale were observed in the Russian
River estuary at a measurement location 2.4 km from the mouth for
the two days between the closed inlet opening to the reclosure in
early October 2010. Observations 7.4 km from the mouth did not exhibit
these oscillations (data not shown). Similar to the tidally discontinuous
forcing in the Pescadero estuary, surface oscillations shut off for
part of the tidal cycle. The dominant infragravity frequencies were
lower in the Russian River estuary than in the Pescadero estuary,
probably because the higher frequencies dissipate closer to the mouth
as seen in the Pescadero estuary and inferred by the lack of an infragravity
signal 7.4 km from the mouth of the Russian River. Given the expected
presence of infragravity motions on beaches worldwide \parencite{contardo_infragravity_2013}
and given the observed presence of an infragravity signal in sandy
inlets and bar-built estuaries, we conclude that this forcing is likely
ubiquitous in this type of estuary.

Considering the kinetic energy per unit volume in the water column,
\begin{eqnarray}
KE & = & \frac{1}{2}\rho(u^{2}+v^{2}),\label{eq:KE}
\end{eqnarray}
 infragravity motions are shown to cause bursts of kinetic energy
in the Pescadero estuary (calculated from the point ADV measurement,
Figure \ref{f9_KE}). In a highly stratified estuary, such as Pescadero,
this energy no doubt contributes to mixing the water column thus reducing
density stratification. Work to quantify this mixing is ongoing. In
small estuaries where residence time during the open state is small
or wind-mixing is limited, energetic velocity bursts due to infragravity
in the nearshore should be a dominant driver of buoyancy mixing. 


\begin{figure}
\includegraphics{chapter3/figures/figure10_KE_all.pdf}\protect\caption{Velocity and kinetic energy measurements by the ADV at NM. Principle
flow $u$ is depicted in black and lateral flow $v$ is gray in (a),
and kinetic energy of both horizontal components of velocity is given
in (b). \label{f9_KE}}
\end{figure}


In larger, density stratified estuaries, such as California's Russian River
estuary, these high energy events likely contribute to buoyancy mixing
near the mouth, thus setting the salinity of ocean-sourced water being
advected upstream where other processes may dominate mixing and transport.
In largely marine systems such as the Albufeira lagoon in Portugal
where freshwater inflow is low, these high velocities near the mouth
will still contribute to mixing, but with no discernible density stratification
influence will be limited to biological or geological relevance. 

Infragravity motions are seen to be relevant for sediment transport
in the surf zone, tidal inlets, and in the closed state of intermittently
closed, bar-built estuaries, swash overwash is the mechanism by which
the sand berm grows \parencite{beach_suspended_1988,green_processes_2001,baldock_morphodynamic_2008}.
In the Pescadero estuary, where in the open state velocities attributed
to infragravity motions may be three times larger than the tidal velocity,
we expect this flow to be responsible for resuspending sediment and
dominant in geomorphological change. While the mouth of the Pescadero
estuary is closed, swash overwash is observed to occur on infragravity
timescales, and this overwash has been seen to raise the water level
of the closed lagoon (data not shown), which must be accompanied by
geomorphic change as seen in the accretion of sediments at the mouth
of other estuaries \parencite{baldock_morphodynamic_2008}.


\subsection{Transitions\label{sub:Transitions}}

The two states of nearshore-estuarine connection and frictional control
where infragravity oscillations disappear dominate the majority of
the tidal cycle in the tidally-discontinuous Pescadero estuary. The
transitions between the two states occur rapidly. 

We observe the cutting off of wave forcing near peak ebb velocity
(Figures \ref{f3_HUHse}, \ref{f8_ltHU}). Using the same mass conservation
approach as in section \ref{sub:disc_FC_tides}, using characteristic
early frictional control values of $\overline{{u}}{}_{DC}=-0.3$ m
s\textsuperscript{-1} and $h_{DC}=1.45$ m, the estimated Froude
number at the estuary mouth,
\begin{eqnarray}
Fr_{m} & = & \frac{\overline{u_{m}}}{\sqrt{gh_{m}}},\label{eq:Fr}
\end{eqnarray}
is $Fr_{m}=0.26$, which is subcritical. This suggests that the shut
down of wave-like motions in the Pescadero estuary is not due to wave
blocking. Late frictional control values of $\overline{{u}}{}_{DC}=-0.2$
m s\textsuperscript{-1} and $h_{DC}=1$ m give a $Fr_{m}=0.76$,
which remains subcritical but is much higher than the early ebb value.
Visual observations of the far ocean end of the inlet suggest that
a further narrowing of the inlet at low tide may accelerate
flows and create super critical flow. Sand waves also form making
it difficult to distinguish hydraulic jumps from flow over a wavy
bed. Rapid sediment transport at the inlet prevented us from being
able to make accurate measurements there, but future work should attempt
to improve our understanding of the downstream control of these estuaries. 

As the nearshore environment retreats from the mouth of the estuary,
flow out is constrained to a sandy channel. As the ocean water level
rises and nearshore-estuarine connection is again established, initial
flow remains limited to the narrow channel. Infragravity motions then
cause bores to propagate up the channel. As the tide continues to
rise, the connection changes as swash spills over the beach (Figure
\ref{f10_mouth_yz}). Initially bore height and velocity are in phase,
but at higher tides more nonlinearities are introduced.

The infragravity motion-induced bores resemble tidal bores seen in
some shallow estuaries on coasts with large tidal ranges. Field observations
of tidal bores show that undular tidal bores do little to dissipate
tidal energy \parencite{wolanski_undular_2004} while breaking tidal bores
contribute an energetic burst of turbulent kinetic energy (TKE) \parencite{simpson_reynolds_2004}. The bore structure in the Pescadero estuary as the tide rises is usually undular but has been observed to be breaking when the ocean wave climate is very large. 



\begin{figure}
\centering
\includegraphics{chapter3/figures/mouth_yz_schematic.pdf} \caption{At low water, the inlet tends to be channelized and flow is laterally
restricted. As the water level increases, flow across the flat beach
is possible, introducing further non-linearities to wave motions.} \label{f10_mouth_yz}

\end{figure}



\subsection{Salt transport}


\subsubsection{Description of salinity response in the Pescadero estuary}

The salt response to the tidally disconnected forcing should depend
on freshwater inflow, size, and shape of the estuary. The response
of the salt field in the Pescadero estuary is described here in general
terms applicable to other salt stratified bar-built estuaries.

%%%%%%% NEEDS SUBFIGURES!!%%%%%%%%%%
\begin{figure}
\includegraphics[width=\linewidth]{chapter3/figures/figure_saltprofile_xz_edited.pdf}

\includegraphics[width=.9\linewidth]{chapter3/figures/figure_saltprofile_h_u_t.pdf}

\protect\caption{Interpolated salt cross-sections of the Pescadero estuary for six
different times (a-f) on the tidal cycle, shown in (g,h). Location
of CTDs used for interpolation are shown with (+), and x is positive
upstream. (g,h) gives the estuarine water level and depth-averaged
velocity at DC corresponding to the salt-profiles above. \label{f10_Sxz}}
\end{figure}


Looking first at salt movement at several snapshots in time in an
estuarine cross section, Figure \ref{f10_Sxz} (a-f) depicts the salt
field in x and z as interpolated with nine CTDs. The location of each
instrument is marked with a (+), and the corresponding estuarine water
level and depth-averaged velocity are shown (Figure \ref{f10_Sxz}g,h).
Starting with a completely or nearly fresh estuarine basin, at the
start of the flood tide ocean water moves upstream with a strong frontal
or salt-wedge structure (Figure \ref{f10_Sxz}a). This salt water
moves upstream and establishes strong longitudinal stratification
(Figure \ref{f10_Sxz}b). With tidal flow reversal and in the presence
of freshwater flow, the longitudinal stratification relaxes to vertical
stratification (Figure \ref{f10_Sxz}c) and a reversal of the pycnoclines
may occur (Figure \ref{f10_Sxz}d). Salt water continues to be removed
from the salt trap through the rest of the ebb (Figure \ref{f10_Sxz}e).



\begin{figure}
\includegraphics[width=\linewidth]{chapter3/figures/figure9_z_t_salt_vel_shear_wcb_edited.pdf}
\protect\caption{Salinity, velocity, and vertical shear with time for three conditions
during the field deployment. The first column (a,c,g) shows conditions
during a spring tide with higher freshwater flow. The second column
(b,e,h) gives neap conditions and an intermediate freshwater flow,
and the third column (c,f,i) gives spring conditions for larger spring
tides than previously and with less freshwater flow into the estuary.
\label{f11_SUtz}}
\end{figure}



Figure \ref{f11_SUtz} depicts the changing salinity and velocity
structure and shear ($\frac{\partial u}{\partial z}$) with time over
two tides for spring conditions with substantial freshwater inflow
to the estuary (a,d,g), neap tidal conditions (b,e,h) and spring conditions
with a larger tidal elevation change and less freshwater (c,f,i). 

The velocities are measured in 15 cm bins, with an ADCP blanking distance
of 50 cm. In this very shallow flow, dynamics are lost by not measuring
the bottom of the water column, but the study was limited by available
instruments. Salinity data from four instruments (three instruments
in the first period) were interpolated onto a grid. CTDs were spaced
unevenly and more concentrated in the upper water column (cf \ref{sub:Data-processing}),
so we expect there to be unresolved gradients. We expect to see a
sharper salinity gradient than appears with our interpolation scheme.
However, relevant salt dynamics still are made apparent with this
method. 

The general structure of salt moving in and out of the estuary follows
the tidal velocities. On the large flood, salt is advected upstream.
Shear in the water column is low and slightly positive, consistent
with unstratified channel flow. In the presence of sufficient freshwater,
surface waters at DC only momentarily become saline (Figure \ref{f11_SUtz}a).
The measurement of shear suggests reduced flood velocities occurred
at the surface due to outflowing freshwater. Comparatively, in the
later spring tide when less freshwater was present, ocean salinity
water is present at the surface for a longer duration and evidence
of two-layer flow on the flood is absent (Figure \ref{f11_SUtz}c,f,i). 

On the ebb, as seen in Figure \ref{f10_Sxz}, the water column vertically
stratifies. Flow out of the estuary is two layer as evidenced by high
negative vertical velocity shear. The mechanism by which salt appears
to be removed from deep regions of the estuary is through shear-induced
mixing. On the smaller of the semidiurnal ebbs, salt is sometimes
retained in the lower reaches of the estuary, and very high shear
is observed as the lower water column velocities approach zero. 

Smaller tidal amplitudes during the neap tide reduce the tidal velocities
throughout the water column. Ocean water is still advected in with
a near-ocean salinity on the flood, but slower ebb velocities may
prevent a complete flushing of the estuarine salt pockets. Very high
shear is observed as lower water is consistently very slow moving,
thus we again see the salt pocket being eroded away by shear-induced
mixing. 

The last time period shown depicts bigger spring tides as well as
the beginning of the transition from the wet season to the dry season.
Reduced vertical shear on the ebb suggests both advection and shear-induced
mixing are responsible for moving salt water downstream. Less freshwater
in the system probably reduced stratification, which also could have
been influenced by higher mixing with the larger tidal velocities. 

In general, over the tidal cycle, salt water moves upstream primarily
due to advection, and then removal of salt water from the estuary
is by advection and shear-induced mixing. Variation due to tidal amplitudes
and freshwater flow alter the specifics of salt dynamics on tidal
timescales. 


\subsubsection{Effects of two-layer flow on frictional control}

Two-layer flow seen on the ebb while salt water is trapped at depth
and freshwater flows out across it will allow surface waters to flow
more quickly than if they were in contact with the rough bed. Given
that frictional control is established long after the depth-averaged
velocities have become negative (Figure \ref{f3_HUHse}\ref{f8_ltHU}),
it is possible that the establishment of frictional control on the
water column corresponds to the point at which flow at the mouth and
in the shallow region of the estuary is one-layer and the only salt
water retained in the system is in the deep salt traps.


\subsubsection{Relation to salt transport in other estuaries}

Estuaries with a deep inlet do not have a discontinuous connection
between the dense ocean and fresh riverine boundaries of the estuary.
In these estuaries, buoyancy differences between these boundaries
drives gravitational circulation, a constant subtidal exchange \parencite{maccready_advances_2010}.
In the Pescadero estuary where the mouth is set above low water level
in the ocean, this continuous exchange flow cannot exist as the dense
water source is disconnected daily. Salt transport is then limited
to processes on tidal and supratidal timescales.



%***********************************************************************************************************************************************************************************************
% CONCLUSIONS


\section{Conclusions}

We have observed dominant hydrodynamic processes within one California
estuary which are likely similar to those present in similar estuaries
along the western coast of the Americas as well as in Australia, South
Africa, and in estuaries in Mediterranean climates on the Atlantic
west coast of Europe, as well as in shallow sandy inlets elsewhere.

We see that in the open state of intermittently open estuaries:
\begin{enumerate}
\item The mouth of the estuary may remain perched above low water in the
ocean.
\item The perched mouth limits tidal velocities and does not allow ocean
gravity waves to enter the estuary while permitting infragravity motions
to pass through the inlet.
\item High velocities induced by these infragravity motions are energetically
important.
\item On the large ebb the ocean retreats from the mouth, nearshore forcing
is cut off, and frictional control describes the velocity.
\item The salt field is tidally advected upstream but disconnects from the
ocean and is removed by shear.
\end{enumerate}
This work has highlighted the strong dependence of hydrodynamics of
small bar-built estuaries on nearshore processes. Future field observations
in these estuaries should include nearshore measurements to better
quantify this connection. Previous work has not described the influence of nearshore forcing on estuarine hydrodynamics, thus the oscillation between nearshore-forced and disconnected has also not been described. Salt movement within the Pescadero estuary reflects similar transport observed elsewhere. 






%**********************************************************************************************************************************************************************************************
% FIGURES

