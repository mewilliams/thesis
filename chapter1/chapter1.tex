\chapter{Introduction - v 3 Aug 2014}
\label{chIntro}

reference cite keys:
Cousins: \cite{cousins_effects_2010,cousins_hydrodynamics_2011}
Elwany \cite{elwany_opening_1998}
Behrens \cite{behrens_characterization_2009,behrens_episodic_2013}
Largier \cite{largier_structure_1986,slinger_evolution_1990,largier_circulation_1991,largier_dynamics_1991,largier_stratified_1992,largier_tidal_1992,largier_seasonally_1997}
Rich and keller \cite{rich_hydrologic_2013}
A. Becker \cite{becker_artificial_2009}
Ranasinghe \cite{ranasinghe_flushing_1998,ranasinghe_circulation_1999,ranasinghe_morphodynamic_1999,ranasinghe_seasonal_2003}
Gale \cite{gale_vertical_2006,gale_processes_2007}
Laudier \cite{laudier_measured_2011} * probably belongs in Pescadero chapter.
Strech (SA) - \cite{stretch_breaching_2006,parkinson_breaching_2007}
ICOLL - \cite{haines_morphometric_2006,baldock_morphodynamic_2008,davidson_simple_2009}
Hanes - Crissy Field. \cite{hanes_waves_2011}
Portugal - \cite{fortunato_morphological_2014}
Spain - \cite{moreno_morphodynamics_2010}
\cite{uncles_infragravity_2014}
Beaches (ICOLL)  \cite{weir_beach_2006}

... Estuaries are important... 
\emph{timescales are short in these estuaries... behrens et al. 2009 daily/tidal sed movement}.



California's coast is dotted with bar-built estuaries.  These estuaries drain 


\section{worldwide:}
Chile \parencite{dussaillant_water_2009}


\emph{Similar type of estuary exists on wave dominated coasts in Mediterranean climates worldwide. The terminology intermittently closed and open lakes and lagoons (ICOLL) is used in Australia, while the South African's refer to their intermittently closed estuaries as temporarily open/closed estuaries (TOCE).  Here we will refer to intermittently closed or open estuaries, but recognize that this naming convention... }

\section{Hydrodynamics of intermittently closed estauaries}
Previous work has described general functioning of intermittently closed estuaries.  
\emph{description of lots of things}

These estuaries exist in many parts of the world, but are restricted to Mediterranean climates on wave-dominated coasts \emph{perhaps a citation}, \emph{which is basically} Spain, Portugal, South Africa, (W?) Australia, \emph{maybe Chile?}. 

Previously studied intermittently open estuaries include those in South Africa (e.g. \cite{largier_dynamics_1991})...


Interest in the hydrodynamics and morphoodynamic of California's bar-built estuaries is \emph{on the up swing}. Older studies include the work of Elwany et al. where X \cite{elwany_opening_1998}. More recent work looking at mouth dynamics on the Russian River estuary \parencite{behrens_characterization_2009, behrens_episodic_2013} and on the mouth of the Carmel River estuary \parencite{rich_hydrologic_2013}... On the hydrodynamics side of things, Cousins et al. \cite{cousins_hydrodynamics_2011} did X in Rodeo Lagoon. 

\section{Morphodynamics, sand transport, inlet closure, accretion, opening}
Davidson et al. \cite{davidson_simple_2009} \emph{says that} sediment infilling in intermittently closed inlets is due to three processes: litoral processes, tidal inlet flow asymmetries, and downslope sediment diffusion.

Closure papers: \cite{ranasinghe_seasonal_2003, elwany_opening_1998, fortunato_morphological_2014, hanes_waves_2011, behrens_characterization_2009, behrens_episodic_2013}.

Increaing berm while closed: \cite{baldock_morphodynamic_2008, laudier_measured_2011, weir_beach_2006} \emph{check Weir reference}. 

Opening/breach papers: \cite{rich_hydrologic_2013, parkinson_breaching_2007, stretch_breaching_2006, elwany_opening_1998, fortunato_morphological_2014}

\section{Hydrodynamic papers}
Closed state hydrodynamics \& salt \& density: \cite{cousins_hydrodynamics_2011, slinger_evolution_1990}

Open state: \cite{largier_dynamics_1991}

Transition/breach: \cite{becker_artificial_2009}  \emph{stretch and parkinson stuff?}


\emph{lack of work in california, not a lot of interest since some work in the 1990s.  Recently beginning to connect hydrodynamics to nearshore environment \parencite{dodet_wave-current_2013, uncles_infragravity_2014}.}



Research Questions:
\begin{itemize}
	\item{How does the Pescadero estuary function during open, closed, and transitioning mouth states?}
	\item{What are the drivers of hydrodynamics in the open state of small intermittently closed bar-built estuaries?}
	\item{How do tides, stratification, infragravity motion forcing, and \emph{marshes} set salt dispersion in a strongly stratified bar-built estuary?}
	\item{How did the 2011 Tohok$\overline{\mathrm{u}}$ tsunami alter flow in Calfifornia's bar built estuaries?}
\end{itemize}




\section{outline of this thesis}

In this dissertation, I will present XXXXX.

Chapter 2 focuses on work specific to the Pescadero estuary. In this chapter, field methods relevant to the work in general are presented, and the conditions affecting dynamics in the system in general are described. These are followed by a description of closed state hydrodynamics and a look at how the Pescadero estuary responds to the opening of the closed mouth. This chapter aims to provide insight to land managers through descriptive XXXXXX of the Pescadero estuary.

In Chapter 3, I examine open-state forcing and hydrodynamics. The connection between waves and nearshore sand transport has long been established, bu tthe nearshore influence on dynamics within bar-built estuaries is XXXXX. \emph{(Sentence on constricted mouth).} This chapter attempts to describe the influence of the nearshore environment unique to bar-built estuaries... \emph{and... soemthing else}. 

Chapters 3 and 4 address salt transprot in the open state, first as XXX in Chapter 3, and then as XXXX in Chapter 4. 

Finally, Chapter 5 describes the response of the Pescadero estuary to the arrival of waves from the tsunami generated by the 2011 Tohok$\overline{\mathrm{u}}$ earthquake. 
