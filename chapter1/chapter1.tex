\chapter{Introduction}
\label{chIntro}

Along the world's coastlines, rivers and streams drain water from all the planet's land area through estuaries.  In many cases estuaries contain unique, productive ecosystems at this confluence of fresh and ocean water. The natural harbor character of many estuaries has also brought large human population centers, international shipping, and industry, leading to contamination, the introduction of invasive species, and the loss of peripheral wetlands. 

An understanding of the dynamics of estuaries is necessary to protect, restore, or limit the negative impacts of human development. Estuarine hydrodynamics are slowly driven by the large density difference between freshwater and ocean water and complicated by tidal forcing, changing riverine discharge, wind, and in large estuaries the effects of Earth's rotation. A complete understanding of these processes and interactions can inform conditions felt by estuarine flora and fauna and transport of contaminants or nutrients, yet, this complete understanding is elusive.

Within the state of California on North America's west coast, the San Francisco Bay drains the Sacramento-San Joaquin watershed, consisting of the California Central Valley and western slope of the Sierra Nevada. California's Coastal Range lies west of the Central Valley, and these coastal watersheds drain the western side of California through small bar-built estuaries. Like larger estuaries, these small systems are historically beneficial environments for aquatic species, host migratory water birds, and provide human recreation space. Coho salmon and steelhead trout pass through California's small estuaries en route to spawn in the headwaters, and those fish who spend time in lagoons perched behind sand barriers grow faster and are more likely to survive in the Pacific Ocean \parencite{hayes_steelhead_2008, bond_marine_2008}. 

Bar-built estuaries are formed when ocean processes form a sand bar at the downstream end of drowned river valleys \parencite{nielsen_coastal_2009}. In California, these estuaries are small and have an intermittent connection to the ocean. The sandy inlet of these estuaries may close for days, seasonally, or may only temporarily open with large flood events. 

%\empH{they exist worldwide:}
Intermittently closed estuaries exist worldwide, primarily on wave-dominated coasts with seasonal rainfall. Those in South Africa, where they are named temporary open/closed estuaries (TOCE), and Australia, where they are called intermittently closed/open lakes and lagoons (ICOLL), have received the most scientific attention. Estuaries with temporarily closed inlets also exist in Chile \parencite{dussaillant_water_2009}, Brazil \parencite{suzuki_effects_1998}, Spain \parencite{moreno_morphodynamics_2010}, Portugal \parencite{fortunato_morphological_2014}, New Zealand \parencite{schallenberg_contrasting_2010}, India and Sri Lanka \parencite{ranasinghe_seasonal_2003}, and likely elsewhere. In this dissertation, the terms intermittently closed and intermittently open are nearly exchangeable, emphasizing estuaries that are more often open to tides as ``intermittently closed'' and those that more often cut off from the ocean by a sand barrier as ``intermittently open.'' Estuarine processes important in intermittently closed estuaries may also be relevant to small bar-built estuaries that do not experience inlet closure. 

Traditional estuarine hydrodynamics research has focused on much larger estuaries, so despite the prevalence of small, intermittently closed estuaries, they remain understudied.  Several factors may play into this. These small estuaries are not often navigable by large ships, limiting their economic importance. Estuarine physics research is usually within the realm of physical oceanography or environmental engineering, and the small size of intermittently closed estuaries contrasts with the scales typically studied by physical oceanographers. Furthermore, some of the research has been performed in the sphere of coastal management and consulting engineering where reports on findings may not be readily accessible or widely distributed.

Some important work on intermittently closed estuaries is highlighted below. Many studies focus on inlet morphodynamics, especially on closure mechanisms, though the ability to predict mouth closure is still elusive. Some work has examined hydrodynamic processes. A brief description of our current understanding follows. When possible, examples from California are cited.

%%%%%%%%%%%%%%%%%%%%%%%%%%%%%%%%%%%%%%%%%%%%%%%%%%%%%%%%%%%%%%%%
\subsubsection{Inlet morphodynamics}

We provide a short summary of sand transport in inlets, highlighting elements important to this dissertation research. Research on morphodynamics tends to focus on inlet closing and conditions setting this closure. A small set of papers addresses how the sand berm grows after inlet closing, and at least a few experimental papers have addressed how the sand berm is eroded upon breach. Much literature analyzes inlet migration and sediment transport in non-closing inlets, and while these processes may also be relevant to closing inlets, their contributions are not important to research contained within this dissertation.

Ranasinghe and Pattiaratchi \parencite*{ranasinghe_seasonal_2003} define two mechanisms for the sand transport required for inlet closure: (1) by longshore currents, or (2) by cross-shore currents causing sandbars to migrate on shore. Either of these mechanisms interacts  with tidal and riverine flow at the mouth. Both tidal and riverine flow act opposite infilling of the inlet. The mechanism responsible for mouth closure may be seasonally varying for the same estuary. Examining a dataset of over 60 years of mouth state records, Behrens et al. \parencite*{behrens_episodic_2013} found closure of the Northern California Russian River mouth likely to be due to cross-shore transport during the summer and early fall, but due to longshore transport during other seasons.

Within California, some examples of conditions for mouth closure have been described. In a tidal marsh in San Francisco's Crissy Field, located inside of the Golden Gate on the San Francisco Bay, the sandy inlet closes either after a period of large waves (typically from the northwest), but also was seen to close during a small wave climate with long period swell from the south or southwest \parencite{hanes_waves_2011}. During low flow conditions characteristic of summer Mediterranean climates, the probability of inlet closure increased with ocean wave height, and streamflow determined the length of closure \parencite{behrens_episodic_2013}. Modeling morphodynamics coupled with hydrology on the Carmel River inlet in Central California, Rich and Keller found an exponential decrease in closure probability as freshwater streamflow increased \parencite*{rich_hydrologic_2013}. Perhaps these studies are best summarized by saying that given enough freshwater flow, the inlet will remain open, and in the absence of enough riverine flow, it might not.

After the estuarine inlet has closed, swash overwash is responsible for berm development \parencite{baldock_morphodynamic_2008}. Two modes have been proposed to grow the berm: vertical berm growth due to wave overtopping the sand berm during spring tides and horizontal berm growth toward the ocean as a smaller berm is formed at the base of the larger berm during neap tides \parencite{weir_beach_2006}. 

In many cases, the breaching of the sand barrier at the mouth is artificial, where the inlet is opened by human intervention (e.g. \cite{fortunato_morphological_2014, behrens_episodic_2013}). Stretch and Parkinson \parencite*{stretch_breaching_2006} investigated breach dynamics by building an experimental lagoon, cutting a small outlet, and observing the resulting outflow, finding a relationship between outflowing volume and breach width. Further work in their experimental lagoon characterized timescales of sand berm breach in relation to outflow volume, hydraulic head, and the breadth of the sand barrier \parencite{parkinson_breaching_2007}. 


%%%%%%%%%%%%%%%%%%%%%%%%%%%%%%%%%%%%%%%%%%%%%%%%%%%%%%%%%%%%%%%%
\subsubsection{Intermittently open estuarine hydrodynamics}

Historically, hydrodynamic observations made in South African then Australian intermittently open estuaries set the framework for understanding general salt dynamics and circulation within these estuaries. Work in the small, South African Palmiet estuary defined the general circulation: salt water enters the estuary as a tidal intrusion front owing to the constricted, sill-like inlet, flows upstream as a density current, and is removed by shear entrainment into fresher (outflowing) surface waters \parencite{largier_stratified_1992}. Field measurements in Wilson Inlet in southwestern Australia show a similar salinity intrusion and trapped salt water in the open state \parencite{ranasinghe_circulation_1999}. Wilson Inlet is a large coastal lagoon that experiences wind forced circulation and the effects of Coriolis, but salinity intrusion similarities suggest that the constricted mouth of small and large intermittently open estuaries allows some processes to remain similar across length scales. 


In the closed-mouth state of intermittently open estuaries, stratification and circulation are most often studied in the context of degraded water quality often accompanying the closed state. Cousins et al. \parencite*{cousins_effects_2010} explored seasonal hydrodynamics in Rodeo Lagoon, a eutrophic, infrequently open estuary north of San Francisco, California. The closed condition of Rodeo Lagoon means that tidal forcing is cut off so wind forcing drives circulation and mixing. Wave overwash provides a salt water source to the lagoon, and resulting density stratification damps turbulence in lower waters. In the summer, wave overwash is rare and wind-mixing converts the lagoon to a well-mixed, brackish state.

Measurements or numerical modeling of the hydrodynamics of the transition from an open to closed state (sand bar breach) are essentially non-existent. In the case of a well-mixed lagoon, one would expect the lagoon draining to start as surface flow out and progress quickly to vertically homogeneous outflow.  In the case of a highly stratified lagoon, stratification could impede uniform flow out. In the case of a fish mortality event triggered by an artificial mouth opening in the Australian Surrey estuary, it was hypothesized that oxygenated (fresh) surface waters flowed out of the estuary during the breach and hypoxic or anoxic higher salinity water was left behind \parencite{becker_artificial_2009}. No velocity measurements were made, leaving this hypothesis unconfirmed.

%%%%%%%%%%%%%%%%%%%%%%%%%%%%%%%%%%%%%%%%%%%%%%%%%%%%%%%%%%%%%%%%
\subsubsection{Pescadero Estuary} 
Morphodynamics of intermittently open estuaries are generally described in terms of wave conditions and freshwater flow, but the influence of the energetic coastal wave environment on estuarine hydrodynamics is not commonly discussed. Salt transport within the open state of intermittently open estuaries has been described in terms of estuarine circulation on tidal timescales. However, work to understand mixing and dispersion commonly quantified in larger estuaries is lacking, owing in part to advancement of the estuarine physics field since measurements were made in the Palmiet estuary in the 1980s. 


%%%%%%%%%%%%%%%%%%%%%%%%%%%%%%%%%%%%%%%%%%%%%%%%%%%%%%%%%%%%%%%%
To attempt to fill in gaps in scientific estuarine knowledge, observations in the Pescadero estuary in Northern California were made during several field campaigns over the years 2010 - 2012. Measurements occurred during the open state, the closed state, and the transitions between these more stable conditions. The Pescadero estuary is just one of many California bar-built estuaries, but one that has been plagued by mortality of federally threatened fish species (steelhead trout, \emph{Oncorhynchus mykiss}). One of the goals of this project was to characterize the Pescadero estuary to inform management decisions. 

Coupled with the current state of understanding, field measurements and analyses were aimed at understanding several processes: dynamics of the estuary breach, high frequency forcing, and salt transport and mixing within the estuary. Furthermore, chance observations during the tsunami generated by the 2011 Tohok$\overline{\mathrm{u}}$ earthquake provided a rare glimse of how small estuaries respond to small tsunamis. 

This thesis will attempt to answer four research questions spanning mouth states, timescales, and forcing: 


\subsubsection{Research Questions:}
\begin{enumerate}
	\item{How does the Pescadero estuary function during open, closed, and transitioning mouth states?}
	\item{What are the drivers of hydrodynamics in the open state of small intermittently closed bar-built estuaries?}
	\item{What processes set salt dispersion in a strongly stratified bar-built estuary?}
	\item{How did the tsunami generated by the 2011 Tohok$\overline{\mathrm{u}}$ earthquake alter flow in California's bar built estuaries?}
\end{enumerate}

To address these general questions, the dissertation will follow the outline described below. The entire dissertation addresses question 1, chapters 3 and 4 focus on question 2, chapter 4 is more detailed and specific to question 3, and chapter 5 describes forcing relevant to question 4.

\subsubsection{Dissertation outline}

This dissertation presents findings from observations and analysis of a bar-built estuary in Northern California, and hopes to push forward understanding of hydrodynamics in intermittently open estuaries while promoting these systems as having many still un-answered and interesting science questions.

Chapter 2 focuses specifically on the Pescadero estuary. This chapter describes the estuary, field methods relevant to the work in general, and the conditions affecting dynamics in the system. These descriptions are followed by an observational discussion of closed state hydrodynamics and a look at how the Pescadero estuary responds to the opening of the closed mouth. This chapter aims to provide insight to land managers through insight into the general functioning of the Pescadero estuary.

Chapter 3 examines open-state forcing and hydrodynamics. The connection between waves and nearshore sand transport is well established, but this chapter will look at observed high velocity oscillations caused by this wave forcing as an energetically important variable in California's bar-built estuaries. The influence of the shallow mouth perched above low water in the ocean is shown to modulate nearshore forcing and alter tidal flow. The influence of this discontinuous forcing on salt transport is then described.

Chapters 3 and 4 address salt transport in the open state, first as general movement into and out of the open estuary in Chapter 3, and then in Chapter 4 looking at dispersion of the salt field during the tidal cycle, which varies at different salinities, probably due to interaction with the marsh surrounding the Pescadero estuary.

Chapter 5 describes the response of the Pescadero estuary to the arrival of waves from the tsunami generated by the 2011 Tohok$\overline{\mathrm{u}}$ earthquake. Early tsunami waves induced high velocity flows within the estuary. As tsunami conditions lingered, the resultant waves raised and lowered the ocean water level, modulating estuarine forcing like short period tides. 

Finally, chapter 6 gives a summary of the content of this dissertation.







