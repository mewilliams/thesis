\chapter{Introduction}
\label{chIntro}

Along the world's coastlines, rivers and streams drain all the planet's land area through estuaries.  These estuaries are in many cases nursery environments XXX, but also with large population centers around often natural harbor XXX, and with international shipping, and ETC, estuaries are also polluted and invaded \emph{infested, overrun, occupied} by \emph{invasive species}. 

Physics of estuaries is \emph{interesting} because where salty ocean water and fresh river water meet, density differences XXXXX. Add to that XXXXX forcing by tidal flow, changing riverine discharge, wind forcing, and in large estuaries the effect of Earth's rotation, and the understanding of estuarine physics is a complicated affair. Characterizing conditions for \emph{bion00b thing}, understanding contaminant transport, ETC, motivates the XXXX study of these environments. 

Within the state of California on North America's west coast, the San Francisco Bay drains the Sacramento-San Joaquin watershed, consisting of the California Central Valley and western slope of the Sierra Nevada. California's Coastal Range lies east of the Central Valley, and these watersheds drain the western side of California through small bar-built estuaries.

Like larger estuaries, these small systems are historically \emph{(steller places)} for fishes, host migratory water birds, and generally support ecosystem function \emph{(that sounds like an ecology word)}. Coho salmon and steelhead trout pass through California's small estuaries en route to spawn in the headwaters, and those fish who spend time in \emph{(something)} rich lagoons perched behind sand barriers grow faster and are more likely to survive their \emph{(part of?)} life in the Pacific Ocean \parencite{hayes_steelhead_2008, bond_marine_2008}. 

Bar-built estuaries are formed when \emph{(XXXXX)}. In California, these estuaries are small and have an intermittent connection to the ocean. The sandy inlet of these estuaries may close for days, seasonally, or may only temporarily open with large flood events. 

%\empH{they exist worldwide:}
Intermittently closed estuaries exist worldwide, primarily on wave-dominated coasts with seasonal rainfall. Those in South Africa, where have been named temporary open/closed estuaries (TOCE), and Australia, where they are called intermittently closed/open lakes and lagoons (ICOLL), have received the most scientific attention. Estuaries with temporarily closed inlets also exist in Chile \parencite{dussaillant_water_2009}, Brazil \parencite{suzuki_effects_1998}, Spain \parencite{moreno_morphodynamics_2010}, Portugal \parencite{fortunato_morphological_2014}, New Zealand \parencite{schallenberg_contrasting_2010}, India and Sri Lanka \parencite{ranasinghe_seasonal_2003}, and likely elsewhere.Estuarine processes important in these estuaries may also be relevant to small bar-built estuaries that do not experience inlet closure and inlets flowing over sand. 

Traditional estuarine hydrodynamics research has focused on much larger estuaries, so despite the prevalence of intermittently closed estuaries, they remain understudied.  Several factors may play into this: these small estuaries are not often navigable by large ships, limiting their economic importance. Estuarine physics work is usually within the realm of physical oceanography or environmental engineering. The small size of intermittently closed estuaries contrasts with the scales typically studied by physical oceanographers..... and, perhaps \emph{more is written as reports by consulting engineers...}

Short of falsely saying that there is a complete dearth of geophysical research on intermittently closed estuaries, some important work is highlighted below. Many studies focus on inlet morphodynamics, especially on closure mechanisms, though the ability to predict mouth closure is still elusive. Some work has focused on hydrodynamics. A brief description of our current understanding follows. When possible, examples from California are cited.

%%%%%%%%%%%%%%%%%%%%%%%%%%%%%%%%%%%%%%%%%%%%%%%%%%%%%%%%%%%%%%%%
\subsubsection{Inlet morphodynamics}
\emph{Three topics/themes: closure papers, how berm gets bigger while closed papers, opening papers. maybe papers on inlet contraction/expansion while open....}


A full review of sand transport in inlets would overwhelm the intended purpose of this introduction, but I will attempt a quick summary relevant to this dissertation research. Research on morphodynamics tends to focus on inlet closing and conditions setting this closure. A small set of papers addresses how the sand berm grows after inlet closing, and limited experimental \emph{and numerical modeling?} papers have addressed how the sand berm is eroded upon breach.

\paragraph{closure papers} 
Ranasinghe and Pattiaratchi \parencite*{ranasinghe_seasonal_2003} define two mechanisms for sand transport for inlet closure: (1) by longshore currents, or (2) by sandbars migrating on shore (cross-shore currents). Either of these mechanisms interacts  with tidal and riverine flow at the mouth. Tidal flow may scour or XXXXX the inlet as XXXXX\emph{asymmetry}. High riverine flow will scour the inlet, although flood conditions may deposit sediment from the watershed to the estuary or ocean.  Davidson et al. \parencite*{davidson_simple_2009} add to these two processes the downslope diffusion of sediment... Littoral drift means .... Tidal flow asymmetries .... \emph{tides usually scour more, waves usually mostly responsible for the big sand transport... can cite Behrens et al 2013}

Some examples of conditions for mouth closure follows. 

In a tidal marsh in San Francisco's Crissy Field, located inside of the Golden Gate on the San Francisco Bay, the sandy inlet closed either after a period of large waves (typically from the northwest), but also was seen to close during a small wave climate with long period swell from the south or southwest \parencite{hanes_waves_2011}. 

\emph{split closure papers to likelihood to close vs. process?}

Closure of the Russian River estuary varies on seasonal timescales. During low flow conditions characteristic of summer Mediterranean climates, the probability of inlet closure increased with ocean wave height \parencite{behrens_episodic_2013}. \emph{CST closes in summer, early fall, LST in late fall, winter, spring... River conditions set \# of days closed. mention extensive dataset}

Closure papers: \parencite{ranasinghe_seasonal_2003, elwany_opening_1998, hanes_waves_2011, behrens_characterization_2009, behrens_episodic_2013}.

\paragraph{berm increase}
After the estuarine inlet has closed, swash overwash is responsible for berm development \parencite{baldock_morphodynamic_2008}. Two modes have been proposed to grow the berm: vertical berm growth due to X and horizontal berm growth due to X \parencite{weir_beach_2006}. 
Increasing berm while closed: \parencite{baldock_morphodynamic_2008, laudier_measured_2011, weir_beach_2006} \emph{check Weir reference}. 
Laudier et al. \parencite*{laudier_measured_2011} measured the lagoon water level at the Carmel River lagoon in Central California and ... \emph{only talked about wave overwash... i think no connection to berm development}


In many cases, closed inlet opening is artificial (e.g. \cite{fortunato_morphological_2014, behrens_episodic_2013})
Opening/breach papers: \parencite{rich_hydrologic_2013, parkinson_breaching_2007, stretch_breaching_2006, elwany_opening_1998, fortunato_morphological_2014}
Stretch and Parkinson built an experimental lagoon, cut a small inlet, and made measurements of the resulting breach (\cite*{stretch_breaching_2006}; \cite{parkinson_breaching_2007}).



%%%%%%%%%%%%%%%%%%%%%%%%%%%%%%%%%%%%%%%%%%%%%%%%%%%%%%%%%%%%%%%%
\subsubsection{Hydrodynamic papers}
\emph{Three topics/themes: closed state dynamics, open state dynamics, transition dynsmics}


The influence of stratification or mouth state is most often discussed in a hydrodynamic context in relation to water quality concerns.

Cousins et al. \parencite*{cousins_effects_2010} explored seasonal hydrodynamics in Rodeo Lagoon, an infrequently open estuary north of San Francisco, California. The closed condition of Rodeo Lagoon means that tidal forcing is cut off and wind forcing drives circulation and mixing there. Wave overwash provides a salt water source to the lagoon, and resulting density stratification damps turbulence in lower waters and ...\emph{WQ?}... In the summer, wave overwash is rare and wind-mixing converts the lagoon to a well-mixed, brackish state.

Gale et al. \parencite*{gale_vertical_2006} also looked at seasonal timescales in two Australian ICOLLs. \emph{stratification salt dependent but episodic}. 




 Gale et al. ........ in Australia..... \parencite*{gale_vertical_2006}, and Slinger ........ in South Africa \parencite*{slinger_evolution_1990}. Differences between thermal stratification and salinity stratification... 

In the open state of these estuaries,... \emph{probably cite everything john wrote in his phd}. Open state: \parencite{largier_dynamics_1991}, and at least R\&P 1999. 


And, in the breaching state, \emph{nada}. Becker et al. published a ... of a fish kill in .... that proposed a mechanism for removing the oxygenated surface layer from .... estuary, but their hypothesis is in no way substantiated \parencite*{becker_artificial_2009}.




%%%%%%%%%%%%%%%%%%%%%%%%%%%%%%%%%%%%%%%%%%%%%%%%%%%%%%%%%%%%%%%%
\subsubsection{Where that background leaves us...}

Missing from this work is (a) much of anything in California, (b) short timescales, (c) thinking about estuarine physics in the context of that field which has been defined for larger systems (e.g. dispersion)... in these.. This dissertation aims to begin to fill in some of these holes. 

%%%%%%%%%%%%%%%%%%%%%%%%%%%%%%%%%%%%%%%%%%%%%%%%%%%%%%%%%%%%%%%%
Observations in the Pescadero estuary in Northern California took place in several field campaigns over 2010 - 2012, and included open state, closed state, and transitions between these more stable conditions. Measurements were being made when the small tsunami generated by the 2011 Tohok$\overline{\mathrm{u}}$ earthquake reached California. The Pescadero estuary is just one of many California bar-built estuaries, but one that has been plagued by mortality of federally threatened fish species. One of the goals of this project was to characterize the Pescadero estuary to inform management decisions there. 

Measurements made in the Pescadero estuary coupled with the current state of understanding... prompted field measurements to tackle a couple of topics: dynamics of the estuary breach \emph{b/c fish kill}. high frequency forcing \emph{bc obs of IG fluctuations}, and salt dispersion with in the estuary \emph{because it's a thing in bigger systems, trying hard to be a real estuary here}. Furthermore, chance observations during a tsunami provided a rare glimse of how small estuaries respond to small tsunamis. 

This thesis will attempt to address four research questions spanning mouth states, timescales, and forcing: 


\subsubsection{Research Questions:}
\begin{enumerate}
	\item{How does the Pescadero estuary function during open, closed, and transitioning mouth states?}
	\item{What are the drivers of hydrodynamics in the open state of small intermittently closed bar-built estuaries?}
	\item{How do tides, stratification, and \emph{marshes} set salt dispersion in a strongly stratified bar-built estuary?}
	\item{How did the 2011 Tohok$\overline{\mathrm{u}}$ tsunami alter flow in California's bar built estuaries?}
\end{enumerate}

\emph{to address these questions, the dissertation will follow the outline described below... q1: chapter 2, but also 3-5, q2: ch3, q3: ch 3\&4, q4: ch 5, but built on understanding developed in the previous chapters.}


\subsubsection{Dissertation outline}

In this dissertation, I will present XXXXX.

Chapter 2 focuses on work specific to the Pescadero estuary. In this chapter, field methods relevant to the work in general are presented, and the conditions affecting dynamics in the system in general are described. These are followed by a description of closed state hydrodynamics and a look at how the Pescadero estuary responds to the opening of the closed mouth. This chapter aims to provide insight to land managers through descriptive XXXXXX of the Pescadero estuary.

In Chapter 3, I examine open-state forcing and hydrodynamics. The connection between waves and nearshore sand transport is well established, but \emph{no one has really begun to } study these waves and their influence on estuarine...... nearshore influence on dynamics within bar-built estuaries is XXXXX. \emph{(Sentence on constricted mouth).} This chapter attempts to describe the influence of the nearshore environment unique to bar-built estuaries... \emph{and... something else}. 

Chapters 3 and 4 address salt transport in the open state, first as XXX in Chapter 3, and then as XXXX in Chapter 4. 

Finally, Chapter 5 describes the response of the Pescadero estuary to the arrival of waves from the tsunami generated by the 2011 Tohok$\overline{\mathrm{u}}$ earthquake. 

In concluding remarks, Chapter 6 will X. 




%%%%%%%%%%%%%%%%%%%%%%%%%%%%%%%%%%%%%%%%%%%%%%%%%%%%%%%%%%%%%%%%%%%%%%%%%%%%%%%%%%%%%%%%%%%%%%%%%%%%%%%%%%%%%%%%%%%%%%%%%%%%%%%%%%%%%%%%%%%%%%%%%%%%%%%%%%%%%%%%%%%%%%%%%%%%%%%%%%%%%%%%%%% END OF CHAPTER ONE %%%%%%%%%%%%%%%%%%%%%%%%%%%%%%%%%%%%%%%%%%%%%%%%%%%%%%%%%%%%%%%%%%%%%%%%%%%%%%%%%%%%%%%%%%%%%%%%%%%%%%%%%%%%%%%%%%%%%%%%%%%%%%%%%%%%%%%%%%%%%%%%%%%%%%%%%%%%%%%%%%%%%%%%%%%%%%%%%%%%%%%%%%%%







reference cite keys:
Cousins: \cite{cousins_effects_2010,cousins_hydrodynamics_2011}
Elwany \cite{elwany_opening_1998}
Behrens \cite{behrens_characterization_2009,behrens_episodic_2013}
Largier \cite{largier_structure_1986,slinger_evolution_1990,largier_circulation_1991,largier_dynamics_1991,largier_stratified_1992,largier_tidal_1992,largier_seasonally_1997}
Rich and Keller \cite{rich_hydrologic_2013}
A. Becker \cite{becker_artificial_2009}
Ranasinghe \cite{ranasinghe_flushing_1998,ranasinghe_circulation_1999,ranasinghe_morphodynamic_1999,ranasinghe_seasonal_2003}
Gale \cite{gale_vertical_2006,gale_processes_2007}
Laudier \cite{laudier_measured_2011} * probably belongs in Pescadero chapter.
Strech (SA) - \cite{stretch_breaching_2006,parkinson_breaching_2007}
ICOLL - \cite{haines_morphometric_2006,baldock_morphodynamic_2008,davidson_simple_2009}
\cite{uncles_infragravity_2014}
Beaches (ICOLL)  \cite{weir_beach_2006}







