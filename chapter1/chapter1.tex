\chapter{Introduction}
\label{chIntro}

\emph{estuaries!:}
The land-sea interface is one of the most biologically productive regions on the planet, while being heavily populated. Along the world's coastlines, rivers and streams drain XXXXX watersheds through X\# estuaries.  These estuaries are in many cases nursery environments XXX, but also with large population centers around often natural harbor XXX, and with international shipping, and ETC, estuaries are also polluted and invaded by \emph{invasive species}. 

Physics of estuaries is \emph{interesting} because where salty ocean water and fresh river water meet, density differences XXXXX. Add to that XXXXX forcing by tidal flow, changing riverine discharge, wind forcing, and in large estuaries the effect of Earth's rotation, and the understanding of estuarine physics is a complicated affair. Characterizing conditions for \emph{bion00b thing}, understanding contaminant transport, ETC, motivates the XXXX study of these environments. 

Within the state of California on North America's west coast, the San Francisco Bay drains the Sacramento-San Joaquin watershed, accounting for X acres in CA, \emph{OR, NV?}. The Colorado River XXXXXXXX. The rest of the state's X acres are drained through rivers and streams in the California Coastal Range \emph{X mountain ranges spread from X to X}. These waterways meet the ocean through California's X bar-built estuaries \emph{(and X coastal inlets).}

\emph{paragraph on ecology of these things, small size, easily altered? salmonids.}

Bar-built estuaries are morphologically defined as XXXXX. In California, these estuaries are small and have an intermittent connection to the ocean. The sandy inlet of these estuaries may close for days, seasonally, or may only temporarily open with large flood events. Similar estuaries in South Africa have been named temporary open/closed estuaries (TOCE) and in Australia are intermittently closed/open lakes and lagoons (ICOLL). While some variation in terminology, including classifying regularly and irregularly closed lagoons where ''regularly closed lagoons'' are  those that close seasonally \parencite{davidson_simple_2009}, here we will not differentiate between ... and refer usually to intermittently closed estuaries, simply because the focus of field observations in this dissertation was open more than closed during the first two years of field measurements. \emph{man, this section doesnt matter at all}. 

%\empH{they exist worldwide:}
Similar estuaries exist worldwide. Those in South Africa () and Australia () have received the most scientific attention, but estuaries with temporarily closed inlets also exist in Chile \parencite{dussaillant_water_2009}, Brazil \parencite{suzuki_effects_1998}, Spain \parencite{moreno_morphodynamics_2010}, Portugal \parencite{fortunato_morphological_2014}, New Zealand \parencite{schallenberg_contrasting_2010}, India and Sri Lanka \parencite{ranasinghe_seasonal_2003}, and likely elsewhere. Many estuarine processes important in these estuaries may also be relevant to small bar-built estuaries that do not experience inlet closure. 

Traditional estuarine hydrodynamics research has focused on much larger estuaries, so despite the prevalence of intermittently closed estuaries, ...........  Several factors may play into this: these small estuaries are not often navigable by large ships, limiting their economic importance. Large estuaries .... physical oceanographers, .... and, perhaps \emph{more is written as reports by consulting engineers...}

Short of falsely saying that there is a complete dearth of geophysical research on intermittently closed estuaries... \emph{here is some work...} Many studies focus on the morphodynamics of inlet conditions/changes (e.g. ...), though .... mouth closure is still elusive. Some work has focused on hydrodynamic conditions (e.g. ...), but XXXXX. A brief description of our current understanding follows.

%%%%%%%%%%%%%%%%%%%%%%%%%%%%%%%%%%%%%%%%%%%%%%%%%%%%%%%%%%%%%%%%
\subsubsection{Morphodynamics, sand transport, inlet closure, accretion, opening}
\emph{Three topics/themes: closure papers, how berm gets bigger while closed papers, opening papers. maybe papers on inlet contraction/expansion while open....}

The topic of sand transport in inlets is \emph{broad, and big, and ...}, but I will attempt a quick summary here. Generally, research on morphodynamics tends to focus on inlet closing and conditions setting this closure. A small set of papers addresses how the sand berm grows after inlet closing, and limited experimental \emph{and numerical modeling?} papers have .... how the sand berm is eroded upon breach.

\paragraph{closure papers} 
Ranasinghe and Pattiaratchi \parencite*{ranasinghe_seasonal_2003} define two mechanisms for sand transport for inlet closure: (1) by longshore currents, or (2) by sandbars migrating on shore (cross-shore currents). Either of these mechanisms compete with tidal and riverine scour at the mouth. \emph{sentecne on how tidal flow doesn't necessarily scour}
Inlet closure typically thought to be determined by the interaction of two processes: littoral drift which forces sand into a dynamic inlet, and tidal inlet flow (CITE). Davidson et al. \parencite*{davidson_simple_2009} also consider for the downslope diffusion of sediment... Littoral drift means .... Tidal flow asymmetries .... 

\emph{ranasinghe and pattiaratchi 2003 says either longshore drift, or movement of sand bars on shore}.

In a tidal marsh in San Francisco's Crissy Field, located inside of the Golden Gate on the San Francisco Bay, the sandy inlet closed either after a period of large waves (typically from the northwest), but also was seen to close during a small wave climate with long period swell from the south or southwest \parencite{hanes_waves_2011}. 

\emph{split closure papers to likelihood to close vs. process?}

Closure papers: \parencite{ranasinghe_seasonal_2003, elwany_opening_1998, fortunato_morphological_2014, hanes_waves_2011, behrens_characterization_2009, behrens_episodic_2013}.

\paragraph{berm increase}
After the estuarine inlet has closed, swash overwash is responsible for berm development \parencite{baldock_morphodynamic_2008}. Two modes have been proposed to grow the berm: vertical berm growth due to X and horizontal berm growth due to X \parencite{weir_beach_2006}. 
Increasing berm while closed: \parencite{baldock_morphodynamic_2008, laudier_measured_2011, weir_beach_2006} \emph{check Weir reference}. 

Opening/breach papers: \parencite{rich_hydrologic_2013, parkinson_breaching_2007, stretch_breaching_2006, elwany_opening_1998, fortunato_morphological_2014}



%%%%%%%%%%%%%%%%%%%%%%%%%%%%%%%%%%%%%%%%%%%%%%%%%%%%%%%%%%%%%%%%
\subsubsection{Hydrodynamic papers}
\emph{Three topics/themes: closed state dynamics, open state dynamics, transition dynsmics}
Closed state hydrodynamics \& salt \& density: \parencite{cousins_effects_2010, slinger_evolution_1990}

Open state: \parencite{largier_dynamics_1991}

Transition/breach:\parencite{becker_artificial_2009} \emph{a stretch to call that hydrodynamics though}.  \emph{Stretch and Parkinson stuff?}

%%%%%%%%%%%%%%%%%%%%%%%%%%%%%%%%%%%%%%%%%%%%%%%%%%%%%%%%%%%%%%%%
\subsubsection{Where that background leaves us...}

Missing from this work is (a) much of anything in California, (b) short timescales, (c) thinking about estuarine physics in the context of that field which has been defined for larger systems (e.g. dispersion)... in these.. This dissertation aims to begin to fill in some of these holes. 

%%%%%%%%%%%%%%%%%%%%%%%%%%%%%%%%%%%%%%%%%%%%%%%%%%%%%%%%%%%%%%%%
Observations in the Pescadero estuary in Northern California took place in several field campaigns over 2010 - 2012, and included open state, closed state, and transitional XXXX conditions. Measurements were being made when the small tsunami generated by the 2011 Tohok$\overline{\mathrm{u}}$ earthquake reached California. The Pescadero estuary is interesting/important because XXXXX, and one of the goals of this project was to characterize the Pescadero estuary to inform management decisions there. 

Measurements made in the Pescadero estuary coupled with the current state of understanding... prompted field measurements to tackle a couple of topics: dynamics of the estuary breach \emph{b/c fish kill}. high frequency forcing \emph{bc obs of IG fluctuations}, and salt dispersion with in the estuary \emph{because it's a thing in bigger systems, trying hard to be a real estuary here}. Furthermore, chance observations during a tsunami \emph{means we should report what we saw}. 

This thesis will attempt to address four research questions spanning mouth states, timescales, and forcing: 


\subsubsection{Research Questions:}
\begin{enumerate}
	\item{How does the Pescadero estuary function during open, closed, and transitioning mouth states?}
	\item{What are the drivers of hydrodynamics in the open state of small intermittently closed bar-built estuaries?}
	\item{How do tides, stratification, and \emph{marshes} set salt dispersion in a strongly stratified bar-built estuary?}
	\item{How did the 2011 Tohok$\overline{\mathrm{u}}$ tsunami alter flow in California's bar built estuaries?}
\end{enumerate}

\emph{to address these questions, the dissertation will follow the outline described below... q1: chapter 2, but also 3-5, q2: ch3, q3: ch 3\&4, q4: ch 5, but built on understanding developed in the previous chapters.}


\subsubsection{Dissertation outline}

In this dissertation, I will present XXXXX.

Chapter 2 focuses on work specific to the Pescadero estuary. In this chapter, field methods relevant to the work in general are presented, and the conditions affecting dynamics in the system in general are described. These are followed by a description of closed state hydrodynamics and a look at how the Pescadero estuary responds to the opening of the closed mouth. This chapter aims to provide insight to land managers through descriptive XXXXXX of the Pescadero estuary.

In Chapter 3, I examine open-state forcing and hydrodynamics. The connection between waves and nearshore sand transport has long been established, but the nearshore influence on dynamics within bar-built estuaries is XXXXX. \emph{(Sentence on constricted mouth).} This chapter attempts to describe the influence of the nearshore environment unique to bar-built estuaries... \emph{and... something else}. 

Chapters 3 and 4 address salt transport in the open state, first as XXX in Chapter 3, and then as XXXX in Chapter 4. 

Finally, Chapter 5 describes the response of the Pescadero estuary to the arrival of waves from the tsunami generated by the 2011 Tohok$\overline{\mathrm{u}}$ earthquake. 

In concluding remarks, Chapter 6 will X. 




%%%%%%%%%%%%%%%%%%%%%%%%%%%%%%%%%%%%%%%%%%%%%%%%%%%%%%%%%%%%%%%%%%%%%%%%%%%%%%%%%%%%%%%%%%%%%%%%%%%%%%%%%%%%%%%%%%%%%%%%%%%%%%%%%%%%%%%%%%%%%%%%%%%%%%%%%%%%%%%%%%%%%%%%%%%%%%%%%%%%%%%%%%% END OF CHAPTER ONE %%%%%%%%%%%%%%%%%%%%%%%%%%%%%%%%%%%%%%%%%%%%%%%%%%%%%%%%%%%%%%%%%%%%%%%%%%%%%%%%%%%%%%%%%%%%%%%%%%%%%%%%%%%%%%%%%%%%%%%%%%%%%%%%%%%%%%%%%%%%%%%%%%%%%%%%%%%%%%%%%%%%%%%%%%%%%%%%%%%%%%%%%%%%







reference cite keys:
Cousins: \cite{cousins_effects_2010,cousins_hydrodynamics_2011}
Elwany \cite{elwany_opening_1998}
Behrens \cite{behrens_characterization_2009,behrens_episodic_2013}
Largier \cite{largier_structure_1986,slinger_evolution_1990,largier_circulation_1991,largier_dynamics_1991,largier_stratified_1992,largier_tidal_1992,largier_seasonally_1997}
Rich and Keller \cite{rich_hydrologic_2013}
A. Becker \cite{becker_artificial_2009}
Ranasinghe \cite{ranasinghe_flushing_1998,ranasinghe_circulation_1999,ranasinghe_morphodynamic_1999,ranasinghe_seasonal_2003}
Gale \cite{gale_vertical_2006,gale_processes_2007}
Laudier \cite{laudier_measured_2011} * probably belongs in Pescadero chapter.
Strech (SA) - \cite{stretch_breaching_2006,parkinson_breaching_2007}
ICOLL - \cite{haines_morphometric_2006,baldock_morphodynamic_2008,davidson_simple_2009}
\cite{uncles_infragravity_2014}
Beaches (ICOLL)  \cite{weir_beach_2006}







