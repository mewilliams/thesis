\chapter{The Pescadero estuary} \label{chPescadero}

The Pescadero estuary is located at the confluence of Pescadero Creek and Butano Creek in southern San Mateo County on the California coast (Figure~\ref{fig:ccrp_2010}). The estuary drains a watershed of 210 km\textsuperscript{2} of the Santa Cruz mountains, located in California's coastal range. The coastal range has a rainfall dominated hydrology, and freshwater stream flow on the creeks is flashy and seasonal.



A history of the Pescadero marsh was compiled as the M.A. dissertation work of Viollis \parencite*{viollis_evolution_1979}. The marsh used to be bigger, has a history of being farmed, drained, duck hunting ground... biggest changes in the 20\textsuperscript{th} century (as written in 1979) because of construction of levees and roads, logging, increased agriculture in the watershed as well as increased water use. The marsh underwent further alterations in the 1990s when the Highway 1 bridge was replaced and \emph{culverts were installed between the Pescadero creek and channel to the North Pond in an attempt to create habitat for X SPECIES... Can be read about in ESA document? \parencite{esa_pescadero-butano_2004}}.

California's intermittently closed estuaries have been seen to play an important role in the lifecycle of juvenile salmonids. Steelhead rearing in the estuary and lagoon habitat of Central California's Scott Creek exhibited much faster growth rates than those smolts rearing primarily in upper reaches of the estuary \parencite{hayes_steelhead_2008}. Subsequent survival in the ocean of the \emph{larger} fishes also increases \parencite{bond_marine_2008}. A similar \emph{predator limited and nurtrient rich environment should exist in the Pescadero estuary.} Unfortunately, massive fish mortality has been observed with the opening of the mouth of the estuary in late fall in many of the past X years. This problem has led to scientific study \parencite{sloan_ecological_2006, smith_inorganic_2009} as well as lawsuits \parencite{scheck_tiny_2012}, but to date a conclusive explanation for the triggers of fish mortality has not been defined. 

Work to understand hydrodynamic and salt transport processes in intermittently closed estuaries in general and in the Pescadero estuary in specific aims to develop a framework for understanding these understudied systems. 

%***********************************************************************
%***********************************************************************
%*************************************************
% SITE STUFF
\section{Measurements made in the Pescadero estuary} \label{measurementslabel}

Measurements were made over the course of three years to quantify dominant forcing during the open and closed states of the estuary. A description of measurement methods precedes a description of general observations and functioning of this estuary.

Field measurements were taken in the Pescadero estuary during four field deployments:

\begin{enumerate}
	\item March - May 2010 
	\item September 2010 
	\item November 2010 - May or June 2011 
	\item August 2011 - June 5, 2012
\end{enumerate}

 Measurements were not always continuous because instrument batteries sometimes died before being replaced and instruments were sometimes buried, failed, or damaged during the deployment. Later deployments utilized more instruments and so had better resolution in space.  The frequency of measurements was also increased in later deployments. 

\subsubsection{Bathymetry} \label{sssec:bathymetry}
Bathymetry measurements were made using the bottom track feature of an acoustic Doppler current profiler (ADCP; RDI Workhorse Monitor 1200kHz) pulled on a raft behind a canoe on 17 November, 2010. The bathymetry map (Figure \emph{copy from chapter 3}) shows that the lower embayment of the estuary is shallow and becomes deep in the narrow channel upstream of the lagoonal area and downstream of the confluence of the Pescadero and Butano Creeks. The mouth was not because measurements were made in the closed-mouth state and because the geometry of the mouth is constantly changing. Bedrock sits below the mouth, setting a minimum elevation for the mouth, but the bedrock is usually covered with sand. 

Sensors were located according to bathymetry. Early attempts to collect measurements in the inlet resulted in buried and broken instruments, even after 24-hour deployments, so long term measurements were limited to the parts of the estuary with less rapid sand transport. 

\subsubsection{Depth} \label{sssec:depth}
Depth measurements were made using the pressure transducer on conductivity, temperature, depth (CTD) sensors
attached to milk crates (RBR XR-420 CTD). The pressure values on the instruments were adjusted to
changing atmospheric pressure using air pressure measurements from the
Half Moon Bay airport (KHAF). An approximate density of $\rho$ = 1021 kg
m\textsuperscript{-3} is used with the hydrostatic pressure equation ${p
= \rho g h}$, slightly overestimating the depth when the estuary is
fresh and density is lower than this value. Error in the pressure
measurement using this assumption may be up to 2.5\%. Further error is
introduced comparing pressure measurements in a quiescent water column
to a quickly moving flow. Stagnation pressure p is actually p = 0.5
$\rho$ g h +  v\textsuperscript{2} (cite general fluid mechanics
textbook). Under quiescent flow, pressure values measured are
hydrostatic pressure. Under small velocities, error in this assumption
is small. If flow reaches 1 m s\textsuperscript{-1} in a 1 m water
column, the pressure reading will increase by 5\% therefore depth
calculated from these measurements not accounting for dynamic pressure
will be slightly overestimated.

We observe error in comparison of recorded water level (according to the
staff gauge on the highway 1 bridge) while the estuary is draining in
February and March 2012 and expect this error to be due to dynamic
pressure readings. Thus, to set the water level in the estuary to a
NAVD88 or MLLW datum, we use measurements taken during the closed water
state and assume quiescent flow.  Further error is introduced as
moorings were weighted to the ground but not fastened in place and some
movement of the bed or mooring may have occurred. \emph{The staff gauge on the Highway 1 bridge is X ft above X ft NAVD 88, and that translates to X ft MLLW or X m MLLW.}

\subsubsection{Salinity and Temperature} \label{sssec:SandT}
Salinity and temperature were also measured by moored CTDs. The instruments were attached to weighted milk crates, floating lines, and floating buoys. In some cases, subsurface buoys maintained taught lines.  Shorter, taught lines seemed to be more prone to being dragged away from their original location so later deployments allowed sensors to be more free floating. \emph{INSERT IMAGE OF CTD MOORING.}

\subsubsection{Velocity} \label{sssec:velmeas}
Velocity measurements were made with moored ADCPs as well as with moored acoustic Doppler velocimeters (ADV; Nortek Vector). ADCPs were attached to flat plates and placed on the bed of the estuary. The ADCPs used are typically used in deeper flow, and were not always successful in measuring velocities. The ADVs, usually attached downward facing from a sawhorse frame, were able to measure velocity, but instruments fixed near the mouth were susceptible to damage by floating debris. In one instance, the titanium probe of an ADV was bent after being deployed only eight hours, probably by a car wheel found the next day in the intertidal zone floating on quickly moving infragravity bores. Driftwood logs are also ubiquitous along the shores of the Pescadero estuary.  \emph{INSERT IMAGES OF ADCP AND ADV MOORINGS.}


\subsubsection{Wind measurements} \label{sssec:windmeas}
Wind measurements were made using a \emph{[Campbell Scientific Met Station]} during only the Fall 2011 - Spring 2012 field deployment. The wind vane was mounted approximately 3 m above the water level in the marsh adjacent to the estuary \emph{(See location on a figure, and include a photograph?)}.  Measurements show that wind direction through the marsh is almost entirely bi-directional (Figures~\ref{fig:metstn_pdo_ws_wdir} and~\ref{fig:metstn_pdo_windrose}). Wind direction of between 300 and 360 degrees is blowing from the ocean and wind blowing from 100 to 170 degrees is land-sourced wind. This constrained directionality is attributed to topography of the marsh.  The inlet is protected by bluffs to the south and historical sand dunes reinforced for the Highway 1 bridge to the north. The marsh itself is located in a low valley, further constricting wind flow paths. 

Windspeeds from the ocean are higher than those from the land (Figures~\ref{fig:metstn_pdo_ws_wdir} and~
\ref{fig:metstn_pdo_windrose}). 


% FIGURE THAT COMPARES KHAF WIND MEASUREMENTS TO PESCADERO MEASUREMENTS.
% 
% 
%***********************************************************************
%***********************************************************************
%*************************************************
% CONDITIONS from observations etc. 
\section{Climatic conditions} \label{conditions_label}

The years 2010-2012 encompassed \emph{[? El ninho, la ninha, drought].} Freshwater flow in water year 2010 (1 October, 2009 to 30 September, 2010) was above the 61 year median, and flow in water year 2012 was below the 61 year median. \emph{Do wave conditions change with ENSO, PDO? I think they do}.



\subsubsection{Freshwater}
Freshwater streamflow into the Pescadero estuary is estimated based on a United States Geological Survey (USGS) gauge located on Pescadero Creek 8.5 km upstream from the mouth of the estuary (USGS 11162500). The gauge is downstream of 57 \% of the watershed, so to estimate freshwater flow into the estuary the gauged discharge measurement is multiplied by $Q_R = 1.76Q_{R,G}$. Unaccounted for freshwater diversions are thought to exist downstream of
the gauge which likely reduce freshwater flow into the estuary, however we do not have a way to quantify these diversions and thus will report freshwater flow as if they do not exist. Flow into the Pescadero estuary was highest in the early part of the calendar year. \emph{Flows as high as X m\textsuperscript{3} s\textsuperscript{-1} and as low as X m\textsuperscript{3} s\textsuperscript{-1} were observed during the years of our field deployments (Figure~\ref{fig:QTH_2010_2012}a).} 

The Mediterranean climate of California is highlighted by seasonal precipitation. Freshwater flow in the rainfall-dominated Pescadero Creek watershed follows this trend. Figure~\ref{fig:Q_1951_2012} shows median gauged creek flow at the USGS Pescadero gauge including data from 1951 to 2012 as well as the freshwater discharge curves for the water years of our study. 



\subsubsection{Tides}
The nearest ocean tide gauges to the Pescadero estuary are operated by the National Oceanic and Atmospheric Administration (NOAA) at Crissy Field in San Francisco, California (Station 9414290) as well as in Monterey Harbor in Monterey, California (Station 9413450). Water level in both locations are recorded every six minutes, and one minute tsunami-capable unverified data is also recorded at these stations. The Northern Californian coast experiences a semidiurnal tide with a neap tide range of under 1 m and a spring tide range up to \emph{almost} 3 m (Figure~\ref{fig:QTH_2010_2012}b). 

\subsubsection{Ocean wave climate}
Ocean wave conditions are obtained through NOAA National Data Buoy Center (NDBC). NDBC buoy 46012 is located approximately 40 km WNW of the Pescadero estuary at 208.8 m depth. This buoy reports ocean significant wave height every hour. Deep water wave heights are larger than wave heights experienced at the coast, but larger waves 40 km from shore will result in larger waves at the coast so we use this value as a proxy for coastal ocean conditions. Maximum wave heights in the ocean from 2010 - 2012 were over 7 m, and minimum were under 1 m (Figure~\ref{fig:QTH_2010_2012}c). Ocean wave heights also follow a seasonal trend with typically smaller waves in the summer months transitioning to larger wave climates in the fall, winter and spring months when storms on the Pacific Ocean generate large swell.

\subsubsection{Wind}
Wind conditions are approximated by measurements at the Half Moon Bay airport (KHAF) during periods before the wind vane was installed in the Pescadero estuary.  There is good agreement in [Wind Speed? Direction], as shown in [FIGURE].

\subsubsection{Tsunami}
The 2011 Tohoku earthquake generated a Pacific Ocean-wide tsunami. A description of the ocean state and the response of the Pescadero estuary to the forcing of the small amplitude tsunami that reached California is found in Chapter~\ref{chTsunami}.



%*************************************************
% OPEN & CLOSED, etc.
\section{Two main states: open and closed, plus some intermediate stuff}
\label{betterlabelmaybe}

Two main states exist for the Pescadero estuary: open and closed.  The inlet continuously expands and contracts with nearshore sand transport competing with freshwater and ebb tidal scour. \emph{Open} is characterized by any period in which the estuarine water level falls with the outgoing tide, and a \emph{closed} state is highlighted by the lack of outflowing water thus no falling depth. Within the open state, tides in the estuary may be an attenuated version of the ocean tides \emph{[Figure]}, and this state we will consider to be fully open. An extensive description of relevant hydrodynamic and salt transport processes in the fully open state is found in Chapter 3 and 4. \emph{Some words later on interannual variability to open state? How much freshwater it takes to freshen the estuary?}.

%*********************

%------------------------

Because the estuarine inlet is in a constantly unstable condition, an intermediate state between fully open and closed is also common in the Pescadero estuary.  In this state, the sandy inlet is constricted. Here, the estuarine water level rises and falls in response to ocean tides, but may only fill with the diurnal high high tide \emph{[Figure]} or be severely attenuated \emph{[Figure]}. We consider this condition to be a highly-constricted open state. Realistically, the open state of the estuary is a continuum. Sand moves in and out of the inlet with tidal flow and waves. High freshwater discharge will scour the mouth.

The closed state of the estuary is characterized by tidal depth oscillations being shut off. This does not mean that the ocean tides cease to influence water level in the estuary but that the water is not outflowing through the channel. Variation in forcing within the closed
state depends on length of closure, to be discussed in section~\ref{ssec:ClosedDynamics}.

Some discussion of mouth closure in the Pescadero estuary following the 2011 Tohoku tsunami is found in chapter~\ref{chTsunami}, but the subject is largely untreated here. Work in the Russian River \emph{(Cite Behrens phd, Behrens et al 2013)} as well as in the Carmel Lagoon \emph{(cite Rich and Keller, 201?)}, both California estuaries, discuss processes for mouth closure relevant to the Pescadero estuary. The specific north facing geometry of the mouth of the Pescadero inlet may affect the direction from which waves must arrive to cause nearshore sand transport to close the mouth in comparison to other California estuaries, but this is a topic for further research.

The transition from the closed state to open state in the Pescadero estuary naturally occurs only when increased freshwater streamflow overtops the sand bar at the mouth. It has been suggested in other estuaries that storm waves overtopping the bar may breach the estuary from the ocean side [CITE], but the mouth of the Pescadero estuary is
somewhat protected by the southern end cliff, and wave conditions observed during our years of study make this unlikely. Given a storm surge similar to that carried by Superstorm Sandy which created new inlets in Barnegat Bay [CITE], this could be possible, but these are very unlikely to occur in California's climate.

A description of estuarine dynamics during the closed and transitioning states follows in this chapter, with the description of dynamics in the open state to follow in the proceeding chapters.


%*************************************************
% CLOSED STATE
\subsection{Closed State} \label{ssec:ClosedDynamics}

\emph{Interplay of slow changes: Qfw, water filling lagoon, small wind forcing, and episoidic bigger events.}
[FIGURE Needed: water level: closure through breach]


\subsubsection{Closed estuarine water levels} \label{cl_wl}

Immediately after closure, the mouth of the Pescadero estuary is still low in elevation (Figure~\ref{fig:mouth_fb_20110825}). Throughout the closed period, we expect the height of the sand berm to be equal to or slightly above the height of the estuary water surface.Initial lagoon infilling occurs primarily to ocean water overtopping the low sand berm. Figure~\ref{fig:depthclosuref11} shows the estuarine water level and ocean water level for the nine days following the mouth closure on August 23, 2011.  

There are at least two mechanisms by which ocean water enters the Pescadero estuary during the closed mouth state. (1) Waves overtop the sand berm directly into the estuary. (2) Waves overtop the crest of the steep bermed beach, and gravity causes salt water to flow down the beach and into the estuary in a river of sea water (Figure~\ref{fig:beachriver} and~\ref{fig:beachriver_after}). The second mechanism is less frequent, but more dramatic, with the one event observed corresponding with a \emph{(X cm)} rapid increase in lagoon water level on that day (Figure~\ref{fig:depthclosuref11}).

Wave overwash carries salt water into the estuary and grows the sand berm. A study on the Belongil Beach and Belongil Creek inlet in New South Wales, Australia showed that \emph{XXXX} [Baldock et al., 2008]. Our observations suggest that wave overtopping in the Pescadero 
estuary occurs on an infragravity timescale, consistent with forcing seen in the open state of the estuary \emph{(cf chapter~\ref{ch3})}.

Due to the hypsometry of the marsh, initial wave overwash is most relevant to raising the water level in the closed lagoon. After a certain point, the surface of the lagoon begins to spill across the marsh and at this point filling becomes much more gradual. Because nearshore forcing is primarily responsible for sand transport causing accretion of the sand berm at the mouth of the estuary, it is unlikely that a condition exists where the mouth becomes high enough that ocean water cannot enter. \emph{[Sand transport by wind may also shape the mouth... I read once]}




\subsubsection{Closed estuary density structure} \label{cl_strat}


In the closed state, the water column remains stratified (Figures~\ref{fig:profNov2010} and~\ref{fig:profFall2011}). The density is set primarily by salinity of the water column, and a fresh or well-mixed brackish state was never present in our closed-state measurements. The density structure appears to be maintained by freshwater slowly increasing the thickness of the fresh surface mixed layer, and by episodic inputs of salt water coupled by limited mixing allowing the stratification to persist. 

Event-based mixing does occur...  Occasionally the wind blows \emph{(hard enough and/or long enough)} to upwell the pycnocline. If this occurs, the wind stress can act directly on the pycnocline (cf Section~\ref{sssec:WindMixNov10}). \emph{(see wind mixing event description below)}.

\subsubsection{Temperature response to stratification} \label{sssec:TempResStrat}

Since salinity sets stratification in the Pescadero estuary, the temperature structure observed is a dependent on that stratification rather than altering it.Observations show that after the estuary has closed and a freshwater layer is present, water temperature in the mixed surface waters follow a diurnal trend of warming and cooling in accordance with solar heating. The salt-stratified lower layer of the estuary does not \emph{feel} this diurnal heating, and may be much warmer than the surface water. \emph{[Figure: temperature during closed state].}

\subsubsection{Closed estuary velocity description} \label{cl_vel}
While the estuary is open to tidal influence, tides are the main driver of flow (in the absence of very large freshwater discharge). After the inlet closes, tidal forcing is cut off. Velocity measurements show that flow within the closed lagoon follows wind forcing, usually with slow movement, but occasionally with higher velocities driven by high windspeeds. A description of one such event follows.

%-------------------------------------------------------%

\subsubsection{Wind mixing event - November 2010} \label{sssec:WindMixNov10}

An observed example of wind setup and mixing occurred just prior to the sand bar breach on November 24, 2010 (Figure~\ref{fig:closed_UVwindsalt}). In situ velocity and salinity measurements combined with wind measurements at Half Moon Bay airport are used to understand this event: 
On November 23, wind forcing was seaward, small surface velocities in the Pescadero estuary were consistent with this forcing, and the water column was vertically stratified. Three layer flow apparent in this state may be due to the bathymetry of the estuary where the lower end of the estuary is very shallow. Wind shear may drive surface flow downstream, and the lower water column will respond by flowing upstream.
At the sensors, \emph{[sentence on trapped lower water]}.

From this point, the wind direction slowly shifted to become landward, and on November 23 at (X pm) GMT, the wind picked up to 10 m s\textsuperscript{-1}. The water column responded immediately to this increased wind shear with surface flow in line with the wind and the  deeper water column compensating with flow in the other direction. Three-layer flow may not be seen because the upper estuary is deeper and the end of the estuary [has a sharp wall instead of a shallow lagoon thing]. Salinity measurements suggest that the strong wind forcing caused slight upwelling of saltier water at the DS mooring, and significant downwelling of fresher water at US as the deeper sensor there went from measuring [15-20] PSU water to [5] PSU water in [X] hours.

At [XX:XX] on November 24, the wind falls down to 5 m
s\textsuperscript{-1} and the water column relaxes, as evidence by flow
reversal. The surface mixed layer at the ADCP is shallower as seen by
comparing the depth of highest shear at the setup event several hours
earlier to the relaxation event, as well as by looking at the salinity
sensors. As the relaxation occurs, the [surface mixed layer becomes
thicker], but mid-column sensors end up fresher than prior to the mixing
event.

It is upon this slightly more well-mixed background that the lagoon
mouth breach occurs, and it is difficult to determine an exact moment of
estuarine breaching...

\emph{Three (?) things drive mixing in this state. The first: direct input of wind shear at the surface creating shear in the water column.  The second is shear by interfacial drag mid-column. And a third process may be movement of the lower water column against the bed creating [... more shear].}


\emph{episodic wind events can rapidly alter salinity strucutre, but observations show stratification always persists.}

%***********************************************************************
%***********************************************************************
%*************************************************
% BREACH STATE
\subsection{Breach.. } \label{breach_dynamics}
Understanding dynamics of the response of the Pescadero estuary to the breach of the mouth sandbar is pressing because of fish mortality events that often occur in conjunction with the opening. Breach events occurred on the following days within our measurement period:
\begin{enumerate}
	\item November 24, 2010
	\item November 11, 2011
	\item January 20-21(?), 2012
	\item Feburary 20(?), 2012
	\item March 3, 2012
\end{enumerate}

Velocity measurements were made during all but theJanuary 2012 breach. \emph{note on data: jan. 20 adcp data is sitting in raw form on harddrive at home...}

No measurements on top of marsh and few measurements were made in
creeks upstream of the confluence, so data shows what passes by sensors in the lower estuary. High water
levels allow salt to extend farther upstream than measurements were made.

Based on measurements as the sand berm at the mouth breached, two types of events occurs: one where the entire (lower) estuary is freshened \emph{and mixed} (Figures~\ref{fig:ctdBreachNov11} and~\ref{fig:ctdBreachJan12}, and the second where mixing occurs in the fresh surface layer and the stratified lower layer, but salt water is retained in the depths of the estuary (Figure~\ref{fig:ctdBreachFeb12} \emph{also March 2012, but data need some quality control}).

Salt measurements:
In all measurements made during the closed mouth state of the estuary, salt water was present in the estuary and the water column was stratified. In most (all?) cases, the water column had two or more layers, generally characterizable as salt water and fresh water, but more accurately a stratified lower layer and a fresh upper layer. 

\emph{RE-WRITE...} Of the four Fall 2011 - Spring 2012 mouth bar breaches observed, the early two resulted in a fully mixed water column while the later two saw maintenece of a stratified state. The primary difference between these two events was the length of the draining of the estuary. In November 2011 and January 2012, the drawdown of water levels was very fast, suggesting high flows. In these cases the lagoon levels were very high, and the January 2012 breach was accompanied by high freshwater flows. In February 2012 and March 2012, draining of the lagoon occured slowly over X-X hours compared to X-X hours in the earlier events. The February 2012 breach was induced by [guys digging] (some dude at the beach, pers. comm.) and occurred in the absence of increased freshwater flows. [Something about later breach events occurred with lower water surface elevation].

\emph{Can probably still answer:}
\begin{itemize} 
	\item is the estuary flushed or mixed?
	\item how?
\end{itemize}

	









%------------ figure --------------
%%%%%%% breach figure
%------------ figure --------------

\section{megan's summary of all the things} \label{backofch2}

\emph{Mark's comments: take out advice, repace with 2 paragraph with annual/interannual variability.. restate. fill in placeholders. }

\emph{summary section - say here is my take on what the estuary does.  may pull from datasets that don't exist.  there is enough in the chapter to touch on states and transitions... can add as necessary.}

The largest variability observed year to year was in the timing of mouth closure and opening in the Pescadero estuary. The estuary while open to the ocean in September 2010 was highly marine, while during the same days the following year, the mouth was closed and the character of the water column was completely different. 

Low rainfall in water year 2012 limited the freshwater outflow available to maintain an open mouth, and so the mouth repeatedly closed and opened during Winter 2011-2012 and Spring 2012 in what was a seemingly uncharacteristic way. The drought currently occurring in California as of summer 2014 has allowed the mouth to close earlier and stay closed longer, suggesting perhaps a new status quo. 



% putting figures back at the end. %%%%%%%%%%%%%%




% INSERT CALIFORNIA COASTAL RECORDS PROJECT PHOTO OF THE PESCADERO
% ESTUARY HERE. 
% 
\begin{figure}
\includegraphics[width=\textwidth]{chapter2/figures/CCRP_201008354_small.jpg} \caption{Photograph of the Pescadero Beach and Natural Marsh Preserve. Photograph taken Sept. 25,
2010. Copyright \textcopyright  2002-2013 Kenneth \& Gabrielle
Adelman, California Coastal Records Project,
www.californiacoastline.org.} \label{fig:ccrp_2010} \end{figure}
% ---------------------- figure end ------------------------------------------------------------------------------------------


\begin{figure} %\centering
	\includegraphics[width=\textwidth]{chapter2/figures/wind_vs_time_20111027_20120510} \caption{Wind measurements in the Pescadero marsh. (a) is wind speed [m s\textsuperscript{-1}] at 3 m above the water surface. (b) is wind direction in degrees.}
\label{fig:metstn_pdo_ws_wdir} 
\end{figure}

 \begin{figure}
 	\begin{center}
 		\includegraphics[height=10cm]{chapter2/figures/windrose_20111027_20120510} \caption{These are wind measurements in the Pescadero marsh.} \label{fig:metstn_pdo_windrose} 
 	\end{center}
 \end{figure}




% INSERT Q vs water years figure here.
% ---------------------- figure start ----------------------------------------------------------------------------------------------------------
\begin{figure}
	\begin{center}
		\includegraphics[width=\textwidth]{chapter2/figures/Q_1951_to_2012_withstudyyears.pdf} \caption{Gauged freshwater discharge at the USGS Pescadero gauge (11162500) for the years of data collection as well as the median freshwater flow for 1951 - 2012.}\label{fig:Q_1951_2012}
	\end{center}
\end{figure}
% ---------------------- figure end ----------------------------------------------------------------------------------------------------------



% INSERT EXTERNAL FORCING CONDITIONS FIGURE HERE.
% ---------------------- figure start ----------------------------------------------------------------------------------------------------------
\begin{figure}
	\begin{center}
		\includegraphics[width=\textwidth]{chapter2/figures/QTH_2010_2012.pdf} \caption{Estimated freshwater discharge into the Pescadero estuary (a), tidal water level at San Francisco's Crissy Field (b), and ocean significant wave height (c) for calendar years 2010, 2011, and 2012.}\label{fig:QTH_2010_2012}  
	\end{center}
\end{figure}
% ---------------------- figure end ----------------------------------------------------------------------------------------------------------

\begin{figure}
	\begin{center}
		\includegraphics[height=10cm]{chapter2/figures/mouth_fb_20110825_recentlyclosed.jpg} \caption{Photograph of the mouth of the recently-closed Pescadero estuary on August 25, 2011.}
	\end{center}
\label{fig:mouth_fb_20110825} \end{figure}


%%%% FIGURE %%%%%%%
\begin{figure}
	\begin{center}
		\includegraphics[height=10cm]{chapter2/figures/depth_closure_fall2011_zoomed.pdf} \caption{Water depth at station XX for the 9 days following mouth closure on 23 August, 2011. Early increases in lagoon depth are rapid and attributed to high high ocean tide wave overwash. More gradual depth increases are attributed to freshwater flow into the lagoon. The large water level increase on 31 August is from wave overwashing over the length of Pescadero Beach forming a salt water river running into the estuary (See figures~\ref{fig:beachriver} \&~\ref{fig:beachriver_after}).}
	\end{center}
\label{fig:depthclosuref11} \end{figure}
%%%% FIGURE %%%%%%%


\begin{figure}
	\begin{center}
		\includegraphics[height=10cm]{chapter2/figures/beach_andmouth_20110831_beachriver.jpg} \caption{Looking north up the Pescadero beach on 31 August, 2011, it is apparent that water flowed into the Pescadero estuary traveling as a river of salt water down the beach and into the closed estuary. Highway 1 is to the right of the photograph, the Pacific Ocean is to the left and the mouth of the estuary is behind the photographer.}
	\end{center} \label{fig:beachriver} 
\end{figure}

\begin{figure}
	%\centering
		\includegraphics[width=\textwidth]{chapter2/figures/beach_fm_20110831_beachriver_aftertideout.jpg} \caption{Ocean water flowed into the Pescadero estuary on 31 August, 2011 both through waves overtopping the closed mouth and through waves overtopping the steep beach and traveling as a river of salt water down the beach and into the closed estuary. Evidence that there was a lot of flow...} \label{fig:beachriver_after} 
\end{figure}
%------------ figure --------------
%%%%%%% november 2010: closed estuary %%%%%%%%%

\begin{figure}[t]
	\centering
	\includegraphics[width=.9\linewidth]{chapter2/figures/wind_event_nov2010.pdf}
\caption{Wind event... (a) CTD depth, (b) Salinity at sensors, (c) East Velocity [range -50 to 50 cm/s], (d) North Velocity [range -50 to 50 cm/s], (e) windspeed [m/s] at KHAF, (f) wind direction [deg] at KHAF}. \emph{I think those letters match}.
\label{fig:closed_UVwindsalt}
\end{figure}

%------------ figure --------------
%%%%%%% november 2010: closed estuary %%%%%%%%%
% ---------------------- figure start ----------------------------------------------------------------------------------------------------------
\begin{figure}
	\begin{center}
		\includegraphics[height=10cm]{chapter2/figures/kite20100415mouth} 
	\end{center}
\caption{The mouth of the Pescadero estuary in the open state on 15 April, 2010. Photograph taken from a camera attached to a kite, so the kite string is visible. Photograph credit: Rusty Holleman}\label{fig:kite_photos} \end{figure}

\begin{figure}
	\begin{center}
		\includegraphics[height=10cm]{chapter2/figures/kite20101119mouth} 
	\end{center}
\caption{The mouth of the Pescadero estuary in the closed state on 11 November, 2010.  Photograph credit: Rusty Holleman}\label{fig:kite_photos} \end{figure}
% ---------------------- figure end ----------------------------------------------------------------------------------------------------------

%------------ figure --------------
%%%%%%% november 2010: closed estuary %%%%%%%%%

\begin{figure}[t]
%\begin{subfigure}{.5\textwidth}
	%\centering
	\includegraphics[width=.9\linewidth]{chapter2/figures/STprofileNov2010}
	\caption{Fall 2010, (a) Salinity, (b) Temperature profiles}
	\label{fig:profNov2010}
%\end{subfigure}
\end{figure}
\begin{figure}
%\begin{subfigure}{.5\textwidth}
	%\centering
	\includegraphics[width=.9\linewidth]{chapter2/figures/STprofileFall2011}
	\caption{Fall 2011, (a) Salinity, (b) Temperature profiles}
%\end{subfigure}
%\caption{(a)(b)}
\label{fig:profFall2011}
\end{figure}

%------------ figure --------------
%%%%%%% november 2010: closed estuary %%%%%%%%%

\begin{figure}
	%\centering
	\includegraphics[width=.9\linewidth]{chapter2/figures/breach_2day_nov2011} \label{fig:ctdBreachNov11}
\end{figure}


\begin{figure}
	%\centering
	\includegraphics[width=.9\linewidth]{chapter2/figures/breach_2day_jan2012} \label{fig:ctdBreachJan12}
\end{figure}


\begin{figure}
	%\centering
	\includegraphics[width=.9\linewidth]{chapter2/figures/breach_2day_feb2012} \label{fig:ctdBreachFeb12}
\end{figure}

