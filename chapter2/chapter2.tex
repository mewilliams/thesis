\chapter{The Pescadero estuary} \label{chPescadero}

The Pescadero estuary is located at the confluence of Pescadero Creek and Butano Creek in southern San Mateo County on the California coast (Figure~\ref{fig:ccrp_2010}). The estuary drains a watershed of 210 km\textsuperscript{2} located in the Santa Cruz mountains in California's coastal range. The coastal range is rainfall dominated, and freshwater streamflow on the creeks is flashy and seasonal. The mouth of the estuary closes by sand transport during dry months and opens with the seasonal increase of freshwater flow. 


% CCRP Photograph of Pescadero estuary
\begin{figure}
\includegraphics[width=\textwidth]{chapter2/figures/CCRP_201008354_small.jpg} \caption{Photograph of the Pescadero Beach and Natural Marsh Preserve. The Highway 1 bridge that crosses the shallow inlet of the estuary. Butano Creek arrives to the confluence of two creeks from the right of the image through wetlands. Pescadero Creek is more wooded and channelized. Photograph taken Sept. 25, 2010. In 2010 the inlet closed in October. Copyright \textcopyright  2002-2013 Kenneth \& Gabrielle Adelman, California Coastal Records Project, www.californiacoastline.org.} \label{fig:ccrp_2010} \end{figure}
%%%

A history of the Pescadero marsh was compiled as the M.A. dissertation work of Viollis \parencite*{viollis_evolution_1979}. Historically, the marsh was bigger, and it has a history of being farmed, drained, and leveed. Large changes in the 20th century included impacts from the construction of levees and roads, logging, increased agriculture in the watershed, and increased water use. The marsh underwent further alterations in the 1990s when the Highway 1 bridge was replaced and culverts were installed between the Pescadero creek and channel to the North Pond in an attempt to create habitat for threatened and endangered species \parencite{esa_pescadero-butano_2004}.

California's intermittently closed estuaries play an important role in the life cycle of juvenile salmonids. Studies in Central California's Scott Creek showed that steelhead reared in the estuary and lagoon habitat exhibited much faster growth rates than those smolts that reared primarily in upper reaches of the estuary \parencite{hayes_steelhead_2008}. Subsequent survival in the ocean of the larger fishes also increases \parencite{bond_marine_2008}. A similar nutrient rich environment should exist in the Pescadero estuary and be beneficial to the threatened steelhead trout. Unfortunately, massive fish mortality has been observed with the opening of the mouth of the estuary in late fall in many recent years. This problem has led to scientific study \parencite{sloan_ecological_2006, smith_inorganic_2009} as well as lawsuits \parencite{scheck_tiny_2012}, but to date a conclusive explanation for the triggers of fish mortality has not been defined. 

Work to understand hydrodynamic and salt transport processes in intermittently closed estuaries in general and in the Pescadero estuary in specific aims to develop a framework for understanding these understudied systems. This chapter outlines observation methods from field studies in the Pescadero estuary, describes the forcing seen to be dominant in the system, and finally describes hydrodynamics and salt transport in the closed state and during the rapid transition from closed to open. 

%***********************************************************************
%***********************************************************************
%*************************************************
% SITE STUFF
\section{Measurements made in the Pescadero estuary} \label{measurementslabel}

Measurements were made intermittently over the course of three years to quantify dominant forcing during the open and closed states of the estuary. A description of measurement methods precedes a description of general observations and functioning of this estuary.

Field measurements were made in the Pescadero estuary during four field deployments:

\begin{enumerate}
	\item March - May 2010 
	\item September 2010 
	\item November 2010 - May or June 2011 
	\item August 2011 - June 5, 2012
\end{enumerate}

 Measurements were not always continuous because instrument batteries sometimes died before being replaced and instruments were sometimes buried, failed, or damaged during the deployment. Later deployments utilized more instruments and so had better resolution in space.  The frequency of measurements was also increased in later deployments. 

% Bathymetry map
\begin{figure}[h]
\centering
	\includegraphics[width=.9\textwidth]{chapter2/figures/bathymetry_map_w_20112012_sensors} \caption{Bathymetry map of the Pescadero estuary. Locations of moorings from September 2011 - June 2012 are shown on this map. Mooring locations for November 2010 - May 2011 deployments are shown in Chapter 5. In the 2010-2011 deployment, DC and DS are in approximately the same location, and an upstream (US) station was located about 70 m upstream from DC/DS.}\label{fig:BathyCh2}
\end{figure}
% end Bathymetry map

\subsubsection{Bathymetry} \label{sssec:bathymetry}
Bathymetry measurements were made using the bottom track feature of an acoustic Doppler current profiler (ADCP; RDI Workhorse Monitor 1200kHz) pulled on a raft behind a canoe on 17 November, 2010 while the estuarine inlet was closed. The bathymetry map (Figure~\ref{fig:BathyCh2}) shows that the lower embayment of the estuary is shallow and becomes deep in the narrow channel upstream of the lagoonal area. The mouth was not measured because measurements were made in the closed-mouth state. While open, the geometry of the mouth is constantly changing. Bedrock sits below the mouth, setting a minimum elevation for the mouth, but the bedrock is usually covered with sand. 

Sensors were located according to bathymetry. Early attempts to collect measurements in the inlet resulted in buried and broken instruments, even after short 24-hour deployments, so long term measurements were limited to regions of the estuary with less rapid sand transport. 

\subsubsection{Depth} \label{sssec:depth}
Depth measurements were made using the pressure transducer on conductivity, temperature, depth (CTD) sensors attached to weighted milk crates (RBR XR-420 CTD, Figure~\ref{fig:CTDphoto}). The pressure values on the instruments were adjusted to changing atmospheric pressure using air pressure measurements from the Half Moon Bay airport (KHAF). An approximate density of $\rho$ = 1021 kg m\textsuperscript{-3} is used with the hydrostatic pressure equation ${p = \rho g h}$, slightly overestimating the depth when the estuary is fresh and density is lower than this value. Error in the pressure measurement using this assumption may be up to 2.5\%. Further error is introduced comparing pressure measurements in a quiescent water column to a quickly moving flow. Stagnation pressure \emph{p} is given by \emph{p = $\rho$ g h +  0.5 $\rho$ v\textsuperscript{2}}, where \emph{h} is water depth above the reference location, \emph{v} is velocity, \emph{g} is gravity and \emph{$\rho$} is water density \parencite{young_brief_2004}. Under quiescent flow, pressure values measured are hydrostatic pressure, so under small velocities, error in this assumption is small. If flow reaches 1 m s\textsuperscript{-1} in a 1 m water column, the pressure reading will increase by 5\% therefore depth calculated from these measurements not accounting for dynamic pressure will be slightly overestimated.

% CTD mooring photograph
\begin{figure}[h!]
\centering
		\includegraphics[width=.6\textwidth]{chapter2/figures/PB110014_CTD_mooring} \caption{A mooring consisting of several CTDs and buoys attached to a weighted milk crate. Bed CTDs were attached 0.2 m above the bottom of the milk crate.}\label{fig:CTDphoto}

\end{figure}
% end CTD mooring photograph


A staff gauge on the Highway 1 bridge has been surveyed as recording 2.87 ft (or 0.875 m) above the North American Vertical Datum (NAVD88) (C. Hammersmark, \emph{pers. comm.}). Mean low low water (MLLW) is 0.0303 m above NAVD88 on the coast by the Pescadero estuary. We observe error in comparison of recorded water level (according to the staff gauge on the highway 1 bridge) while the estuary is draining in
February and March 2012 and expect this error to be due to dynamic pressure readings. Thus, to set the water level in the estuary to a NAVD88 or MLLW datum, we use measurements taken during the closed water state when available and assume quiescent flow.  Further error is introduced as
moorings were weighted to the ground but not fastened in place and some movement of the bed or mooring may have occurred.  

\subsubsection{Salinity and Temperature} \label{sssec:SandT}
Salinity and temperature were also measured by moored CTDs. The instruments were attached to weighted milk crates, floating lines, and floating buoys (Figure~\ref{fig:CTDphoto}). In some cases, subsurface buoys maintained taut lines.  Shorter, taut lines seemed to be more prone to being dragged away from their original location so later deployments allowed sensors to be more free floating. Salinity data presented is matched with its depth measurement. 

% Velocity measurements
\subsubsection{Velocity} \label{sssec:velmeas}
Velocity measurements were made with moored ADCPs as well as with moored acoustic Doppler velocimeters (ADV; Nortek Vector, Figure~\ref{fig:ADVADCPphoto}). ADCPs were attached to flat plates and placed on the bed of the estuary. The ADCPs are designed for use in deeper flow, and were not always successful in measuring velocities. Some data dropouts are attributed to interference by reflection from the surface. The ADVs, usually attached downward facing from a sawhorse frame (Figure~\ref{fig:ADVADCPphoto}), were consistently able to measure velocity, but instruments fixed near the mouth were susceptible to damage by floating debris. In one instance, the titanium probe of an ADV was bent after being deployed only eight hours, probably by a car wheel found the next day in the intertidal zone floating on quickly moving infragravity bores (cf chapters 3 and 5). Driftwood logs are also ubiquitous along the shores of the Pescadero estuary.

ADCP measurements presented in this chapter are from the DC location. ADV data presented in chapters 3 and 4 are from the near mouth (NM) station. 


% ADV and ADCP mooring photograph subplots
\begin{figure}[t]
\begin{subfigure}{.5\textwidth}
	%\centering
	\includegraphics[width=\linewidth]{chapter2/figures/IMG_0153_adv_downward_20110921}
\end{subfigure}
\begin{subfigure}{.5\textwidth}
	\includegraphics[width=\linewidth]{chapter2/figures/IMG_9471_adcp_onstopsign}
\end{subfigure}
\caption{Photographs of downward facing ADVs and an upward facing ADCP used in field deployments.} \label{fig:ADVADCPphoto}
\end{figure}
% end ADV and ADCP mooring photograph subplots


% Wind measurements
\subsubsection{Wind measurements} \label{sssec:windmeas}
Wind measurements were made using an anemometer installed during the Fall 2011 - Spring 2012 field deployment (Model \#05106, RM Young). The wind vane was mounted approximately 3 m above the water level in the marsh adjacent to the estuary (See location on figure~\ref{fig:BathyCh2}).  Measurements show that wind direction through the marsh is almost entirely bi-directional (Figures~\ref{fig:metstn_pdo_ws_wdir} and~\ref{fig:metstn_pdo_windrose}). Wind direction of between 300 and 360 degrees is blowing from the ocean and wind blowing from 100 to 170 degrees is land-sourced wind. This constrained directionality is attributed to topography of the marsh.  The inlet is protected by bluffs to the south and historical sand dunes reinforced for the Highway 1 bridge to the north. The marsh itself is located in a low valley, further constricting wind flow paths. 



% Figure - Met station photograph
\begin{figure} %\centering
	\includegraphics[width=\textwidth]{chapter2/figures/P1020756_met_cropped} \caption{An anemometer made wind measurements approximately 3 m above the water level in the Pescadero estuary.}
\label{fig:metstnPhoto} 
\end{figure}
% End Figure - Met station photograph



Windspeeds from the ocean are higher than those from the land (Figures~\ref{fig:metstn_pdo_ws_wdir} and~\ref{fig:metstn_pdo_windrose}). 


% Figure - Met station windspeed and direction in Pescadero 
\begin{figure}[h!] %\centering
	\includegraphics[width=\textwidth]{chapter2/figures/wind_vs_time_20111027_20120510} \caption{Wind measurements made in the Pescadero marsh. (a) is wind speed [m s\textsuperscript{-1}] at 3 m above the water surface. (b) is wind direction in degrees.}
\label{fig:metstn_pdo_ws_wdir} 
\end{figure}
% End Figure - Met station windspeed and direction in Pescadero 

% Figure - wind rose, pescadero data
 \begin{figure}[h]
 	\begin{center}
 		\includegraphics[height=10cm]{chapter2/figures/windrose_20111027_20120510} \caption{A wind measurement histogram shows that wind blows in two directions and faster velocity wind coems from the ocean direction (blowing SE).} \label{fig:metstn_pdo_windrose} 
 	\end{center}
 \end{figure}
 % end figure - wind rose, pescadero data

% CONDITIONS from observations etc. 
\section{Climatic conditions and external forcing} \label{conditions_label}
The years 2010-2012 encompassed El Ni\~{n}o and La Ni\~{n}a years, and drought conditions in California were present for much of the study. Increased storm activity in the Pacific Ocean with El Ni\~{n}o years has been attributed to an increased wave climate seen on the coast of California \parencite{seymour_influence_1984}. Freshwater flow in water year 2010 (1 October, 2009 to 30 September, 2010) was above the 61 year median, at least after April 2010, and flow in water year 2012 was below the 61 year median. 


\subsubsection{Freshwater}
Freshwater streamflow into the Pescadero estuary is estimated based on a United States Geological Survey (USGS) gauge located on Pescadero Creek 8.5 km upstream from the mouth of the estuary (USGS 11162500). The gauge is downstream of 57 \% of the watershed, so to estimate freshwater flow into the estuary the gauged discharge measurement is multiplied by a scaling factor ($Q_R = 1.76Q_{R,G}$). Unaccounted for freshwater diversions are thought to exist downstream of the gauge which likely reduce actual freshwater flow into the estuary. We do not have a way to quantify these diversions and thus will report freshwater flow as if they do not exist. Flow into the Pescadero estuary was highest in the early part of the calendar year. The highest flow recorded at the USGS Pescadero gauge during our field deployments was Q$_{R,G}$ = 169 m$^{3}$ s$^{-1}$ in March 2011, and freshwater flows under 0.1 m$^{3}$ s$^{-1}$ were common for late summer and early autumn (Figure~\ref{fig:QTH_2010_2012}a).


% figure tide, Q, swh
\begin{figure}[hp]
	\begin{center}
		\includegraphics[width=\textwidth]{chapter2/figures/QTH_2010_2012.pdf} \caption{External conditions relevant to Pescadero estuarine dynamics include estimated freshwater discharge into the Pescadero estuary (a), tidal water level at San Francisco's Crissy Field (b), and ocean significant wave height (c). These values are given for calendar years 2010, 2011, and 2012.}\label{fig:QTH_2010_2012}  
	\end{center}
\end{figure}
% end figure tide, Q, swh


The Mediterranean climate of California is highlighted by seasonal precipitation. Freshwater flow in the rainfall-dominated Pescadero Creek watershed follows this trend. Figure~\ref{fig:Q_1951_2012} shows median gauged creek flow at the USGS Pescadero gauge including data from 1951 to 2012 and the freshwater discharge curves for the water years of our study. 

% figure Q stacked all years with study years.
\begin{figure}[p]
	\begin{center}
		\includegraphics[width=\textwidth]{chapter2/figures/Q_1951_to_2012_withstudyyears.pdf} \caption{Gauged freshwater discharge at the USGS Pescadero gauge (11162500) for the years of data collection compared to the median freshwater flow for water years 1951 - 2012.}\label{fig:Q_1951_2012}
	\end{center}
\end{figure}
% end figure Q stacked all years with study years.


\subsubsection{Tides}
The nearest ocean tide gauges to the Pescadero estuary are operated by the National Oceanic and Atmospheric Administration (NOAA) at Crissy Field in San Francisco, California (Station 9414290) and in Monterey Harbor in Monterey, California (Station 9413450). Water level in both locations are recorded every six minutes, and one minute tsunami-capable unverified data is also recorded. The Northern Californian coast experiences a semidiurnal tide with a neap tide range of under 1 m and a spring tide range up to almost 3 m (Figure~\ref{fig:QTH_2010_2012}b). High and low tide levels in San Francisco during the deployment years were 2.34 m and -0.64 m MLLW, respectively. 




% figure comparison of pdo and khaf winds
\begin{figure}[h!]
\centering
	\includegraphics[width=\linewidth]{chapter2/figures/pdo_vs_khaf_windmeas} \caption{Wind measurements in Half Moon Bay (black) and in Pescadero (red). Wind from the northwest blows off the ocean and agreement is better between KHAF and Pescadero measurements. High windspeeds show good agreement regardless of the direction.} \label{fig:PDOvsKHAFwind}
\end{figure}
% end figure comparison of pdo and khaf winds

\subsubsection{Ocean wave climate}
Ocean wave conditions are obtained through the NOAA National Data Buoy Center (NDBC). NDBC buoy 46012 is located approximately 40 km WNW of the Pescadero estuary at 208.8 m depth. This buoy reports ocean significant wave height every hour. Deep water wave heights are larger than wave heights experienced at the coast, but larger waves 40 km from shore will cause larger waves at the coast thus we use this value as a proxy for coastal ocean conditions. Maximum wave heights in the ocean from 2010 - 2012 were over 7 m, and minimum were under 1 m (Figure~\ref{fig:QTH_2010_2012}c). Ocean wave heights also follow a seasonal trend with smaller waves in the summer months transitioning to larger wave climates in the fall, winter and spring months when storms on the Pacific Ocean generate large swell.


\subsubsection{Wind}
Wind conditions are approximated by measurements at the Half Moon Bay airport (KHAF) during periods before the wind anemometer was installed in the Pescadero estuary.  There is good agreement in wind speed and direction between KHAF and our measurement (Figure~\ref{fig:PDOvsKHAFwind}). Wind direction measurements agree better when wind speeds are high.

\subsubsection{Tsunami}
The 2011 Tohoku earthquake generated a Pacific Ocean-wide tsunami. A description of the ocean state and the response of the Pescadero estuary to the forcing of the small amplitude tsunami that reached California is found in chapter 5.



%*************************************************
% OPEN & CLOSED, etc.
\section{Estuarine state as determined by the condition of the inlet}
\label{betterlabelmaybe}

Two inlet conditions exist for the Pescadero estuary: open and closed (Figure~\ref{fig:kite_photos_open} and~\ref{fig:kite_photos_closed}).  These conditions set the hydrodynamic state of the estuary. The inlet continuously expands and contracts as nearshore sand transport competes with freshwater and ebb tidal scour. An open estuary is characterized by any period in which the estuarine water level falls with the outgoing tide (Figure~\ref{fig:depthclosuref11}, prior to 23 August), and a closed estuarine state is highlighted by the lack of outflowing water thus no falling depth (Figure~\ref{fig:depthclosuref11}, prior to 23 August). Within the open state, tides in the estuary may be an attenuated version of the ocean tides (Figure~\ref{fig:depthclosuref11}, prior to 20 August), and this state we will consider to be fully open. An extensive description of relevant hydrodynamic and salt transport processes in the fully open state is found in chapters 3 and 4.

% figure: photograph mouth kite open
\begin{figure}[h!]
	\begin{center}
		\includegraphics[height=10cm]{chapter2/figures/kite20100415mouth} 
	\end{center}
\caption{The mouth of the Pescadero estuary in the open state on 15 April, 2010. The photograph was taken from a camera attached to a kite, and the kite string is visible in the image. Photograph credit: Rusty Holleman}\label{fig:kite_photos_open} \end{figure}
% end figure: photograph mouth kite open

% figure: water level with closure (CTD)
\begin{figure}
\centering
	\includegraphics{chapter2/figures/depth_closure_fall2011} \caption{Water surface elevation in the Pescadero estuary in August 2011 (a) and significant wave height in the ocean (b). This graphic shows fully open, partially open, and closed mouth estuarine water levels. Early increases in lagoon depth are rapid and attributed to high high ocean tide wave overwash. More gradual depth increases are attributed to freshwater flow into the lagoon. The large water level increase on 31 August is from wave overwashing over the length of Pescadero Beach forming a salt water river running into the estuary while ocean wave conditions are large (See figures \ref{fig:beachriver}).}
	\label{fig:depthclosuref11} 
\end{figure}
% end figure: water level with closure (CTD)

Because the estuarine inlet is in a constantly unstable condition, an intermediate state between fully open and closed is also common in the Pescadero estuary.  In this state, the sandy inlet is constricted. Here, the estuarine water level rises and falls in response to ocean tides, but may only fill with the diurnal high high tide (Figure~\ref{fig:depthclosuref11}, 20-23 August) or be severely attenuated. We consider this condition to be a highly-constricted open state. The open state of the estuary is a continuum. Sand moves in and out of the inlet with tidal flow and waves. High freshwater discharge will scour the mouth.

The closed state of the estuary is characterized by tidal depth oscillations being shut off. This does not mean that the ocean tides cease to influence water level in the estuary but just that the water is not outflowing through the channel.

Some discussion of mouth closure in the Pescadero estuary following the 2011 Tohoku tsunami is found in chapter 5, but the subject is largely untreated here. Work in the Russian River \parencite{behrens_russian_2012, behrens_episodic_2013} as well as in the Carmel Lagoon \parencite{rich_hydrologic_2013}, both California estuaries, discuss processes for mouth closure relevant to the Pescadero estuary. The specific north facing geometry of the mouth of the Pescadero inlet may affect the direction from which waves must arrive to cause nearshore sand transport to close the mouth in comparison to other California estuaries, but this is a topic for further research.

The transition from the closed state to open state in the Pescadero estuary naturally occurs when increased freshwater streamflow overtops the sand bar at the mouth and may artificially occur with human intervention. 

A description of estuarine dynamics during the closed and transitioning states follows in this chapter, with the description of dynamics in the open state to follow in the proceeding chapters.


%*************************************************
% CLOSED STATE
\subsection{Closed State} \label{ssec:ClosedDynamics}

The dominant forcing in open estuaries is due to tidally driven flows. When the mouth of intermittently closed estuaries chokes closed, tidal momentum and buoyancy fluxes are cut off. Thus, we look to other estuarine processes to set hydrodynamics within the closed Pescadero estuary. 

% figure: photograph mouth kite closed
\begin{figure}[h!]
	\begin{center}
		\includegraphics[height=10cm]{chapter2/figures/kite20101119mouth} 
	\end{center}
\caption{The mouth of the Pescadero estuary in the closed state on 11 November, 2010.  Photograph credit: Rusty Holleman}\label{fig:kite_photos_closed} \end{figure}
% end figure: photograph mouth kite closed



\subsubsection{Closed estuarine water levels} \label{cl_wl}
Immediately after closure, the mouth of the Pescadero estuary is still low in elevation (Figure~\ref{fig:mouth_fb_20110825}). Throughout the closed period, we expect the height of the sand berm to be equal to or slightly above the height of the estuary water surface. Initial lagoon infilling occurs primarily to ocean water overtopping the low sand berm. Figure~\ref{fig:depthclosuref11} shows the estuarine water level and ocean water level for the nine days following the mouth closure on August 23, 2011. Water levels in the estuary rapidly increase at high tide for the days immediately after mouth closure.  


% figure - photograph mouth 25 aug 2011 - recently closed
\begin{figure}
\centering
		\includegraphics[height=10cm]{chapter2/figures/mouth_fb_20110825_recentlyclosed.jpg} \caption{Photograph of the mouth of the recently-closed Pescadero estuary on August 25, 2011.} \label{fig:mouth_fb_20110825}
\end{figure}
% end figure - photograph mouth 25 aug 2011 - recently closed



There are at least two mechanisms by which ocean water enters the Pescadero estuary during the closed mouth state to cause rapid lagoon filling: (1) waves overtop the sand berm directly into the estuary, and (2) waves overtop the crest of the steep bermed beach, and gravity causes salt water to flow down the beach and into the estuary in a river of sea water (Figure~\ref{fig:beachriver}). The second mechanism is less frequent, but more dramatic, with the one event observed corresponding with a rapid increase in lagoon water level on 31 August, 2011, just before instruments were pulled out to be redeployed (Figure~\ref{fig:depthclosuref11}).

% figure: photograph beach river (1) 
\begin{figure}
\begin{subfigure}{.5\textwidth}
		\includegraphics[width=\linewidth]{chapter2/figures/beach_andmouth_20110831_beachriver.jpg}
\end{subfigure}
\begin{subfigure}{.5\textwidth}
		\includegraphics[width=\linewidth]{chapter2/figures/beach_fm_20110831_beachriver_aftertideout.jpg}
\end{subfigure}   \caption{Photographs of sea water flowing into the closed Pescadero estuary as waves overtop the entire beach. (a) Ocean water flowed into the Pescadero estuary on 31 August, 2011 both through waves overtopping the closed mouth and through waves overtopping the steep beach and traveling as a river of salt water down the beach and into the closed estuary. (b) Looking north up the Pescadero beach on 31 August, 2011 after the tide retreated, it is apparent that water flowed into the Pescadero estuary traveling as a river of salt water down the beach and into the closed estuary. Highway 1 is to the right of the photograph, the Pacific Ocean is to the left and the mouth of the estuary is behind the photographer.} 		 \label{fig:beachriver} 
 \end{figure}
 % end figure: photograph beach river (1) 

Wave overwash carries salt water into the estuary and grows the sand berm. A study on the Belongil Beach and Belongil Creek inlet in New South Wales, Australia showed fast berm growth can be attributed to swash overwash \parencite{baldock_morphodynamic_2008}. Our observations suggest that wave overtopping in the Pescadero estuary occurs on an infragravity timescale, consistent with forcing seen in the open state of the estuary (cf chapter 3).

Due to the hypsometry of the marsh, initial wave overwash is most relevant to raising the water level in the closed lagoon. After a certain point, the surface of the lagoon begins to spill across the marsh and lagoon filling becomes much more gradual. Because nearshore forcing is primarily responsible for sand transport causing accretion of the sand berm at the mouth of the estuary, it is unlikely that a condition exists where the mouth becomes high enough that ocean water cannot enter. 


\subsubsection{Closed estuary density structure} \label{cl_strat}

With an episodic flux of salt water from the ocean into the Pescadero estuary and a flux of freshwater into the estuary from the creeks, conditions are set to allow the estuary to remain stratified in the closed state (Figures~\ref{fig:profNov2010} and~\ref{fig:profFall2011}). The density is set by salinity of the water column, and a fresh or well-mixed brackish state was never observed in our closed-state measurements. The density structure is maintained by freshwater slowly increasing the thickness of the fresh surface mixed layer, and by episodic inputs of salt water coupled with limited mixing allowing the stratification to persist. 


% FIGURE: S and T vs depth - Nov 2010
\begin{figure}
\centering
	\includegraphics[width=.9\linewidth]{chapter2/figures/profile_nov2010}
	\caption{November 2010, (a) Salinity, (b) Temperature profiles with CTD depth. Measurements were made with a profiling CTD at many locations within the closed estuary.}
	\label{fig:profNov2010}
\end{figure}
% end FIGURE: S and T vs depth - Nov 2010


% FIGURE: S and T vs depth - Fall 2011
\begin{figure}
\centering
	\includegraphics[width=.9\linewidth]{chapter2/figures/profile_fall2011}
	\caption{Fall 2011, (a) Salinity, (b) Temperature profiles with CTD depth. Measurements were made with a profiling CTD at many locations within the closed estuary.}
\label{fig:profFall2011}
\end{figure}
% end figure: S and T vs depth - Fall 2011

\subsubsection{Temperature response to stratification} \label{sssec:TempResStrat}

Since salinity sets stratification in the Pescadero estuary, the temperature structure observed is a product of that stratification as opposed to its driver as is common in lakes and oceans. Observations show that after the estuary has closed and a freshwater layer is present, water temperature in the mixed surface waters follow a diurnal trend of warming and cooling in accordance with solar heating (Figure~\ref{fig:temp_and_strat}). The salt-stratified lower layer of the estuary does not feel the diurnal heating, and may be much warmer than the surface water. 

% FIGURE: S and T vs time - Fall 2011 - solar heating example
\begin{figure}
\centering
	\includegraphics[width=.9\linewidth]{chapter2/figures/temp_salt_week_closed_nov2010}
	\caption{Sensors vertically distributed during the closed state in November 2010 show that the fresh upper water column (purple lines) responds to solar heating while the lower, stratified water column (blue, green, red lines) does not.} \label{fig:temp_and_strat}
\end{figure}
% end IGURE: S and T vs time - Fall 2011 - solar heating example

\subsubsection{Closed estuary velocity description} \label{cl_vel}
While the estuary is open to tidal influence, tides are the main driver of flow (in the absence of very large freshwater discharge). After the inlet closes, tidal forcing is cut off. Velocity measurements show that flow within the closed lagoon is driven by wind forcing. The stratified water column reacts to surface wind stress from gentle wind forcing with two or three layer flow, and with upwelling in the presence of strong sustained winds. Large wind events which trigger upwelling of the saline lower waters in the closed Pescadero estuary cause some vertical mixing. A description of one such event follows.

\subsubsection{Wind driven flow} \label{sssec:WindMixNov10}

An observed example wind setup and mixing in the Pescadero estuary occurred just prior to the sand bar breach on November 24, 2010 (Figure~\ref{fig:closed_UVwindsalt}). In situ velocity and salinity measurements combined with wind measurements at Half Moon Bay airport are used to understand this event: 
On November 23, wind forcing was seaward, small surface velocities in the Pescadero estuary were consistent with this forcing, and the water column was vertically stratified. Three layer flow is apparent which may be related both to the vertical stratification and the influence of shallow bathymetry between the ADCP and closed mouth. Wind shear drives surface flow downstream, and the lower water column will respond by flowing upstream. At the deep DC sensors, a third recirculation region may exist (Figure~\ref{fig:schematic3lf}).


% figure: wind event nov. 2010 - 6 subplots
\begin{figure}[p]
	\centering
	\includegraphics[width=.9\linewidth]{chapter2/figures/wind_event_nov2010ef}
\caption{Observations of wind setup and the inlet breach on 24-25 November, 2010. Plotted are wind measurements at KHAF (a) and (b), ADCP velocity measurements color range $\pm $0.5 m s$^{-1}$ (c) and (d) and the DS (solid line) and US (dashed line) CTD mooring measurements (e) and (f). Observations show a wind setup event on 23 November, and a rapid draining of the estuary which resulted in full mixing of the water column at the DS and US moorings.}
\label{fig:closed_UVwindsalt}
\end{figure}
% end figure: wind event nov. 2010 - 6 subplots


% figure: schematic - 3 layer flow.
\begin{figure}[h!]
	\centering
	\includegraphics[width=.5\linewidth]{chapter2/figures/closedthreelayerflow}
\caption{Three layer flow may occur with wind stress on the surface of the closed lagoon coupled with the effects of stratification and bathymetry.} \label{fig:schematic3lf}
\end{figure}
% end figure: schematic - 3 layer flow.


From this point, the wind direction slowly shifted to become landward, and on November 23 at 18:20 GMT, the wind picked up to above 7 m s\textsuperscript{-1}. The water column responded immediately to this increased wind stress with surface flow in line with the wind and the  deeper water column compensating with flow in the other direction (Figure~\ref{fig:schematicupwelling}). Three-layer flow may not be seen because the upper estuary is deeper and the leeward end of the estuary has a steep wall compared to the shallow lagoonal embayment, or because upwelling occurred rapidly. Salinity measurements suggest that the strong wind forcing caused upwelling of saltier water at the DS mooring (Figure~\ref{fig:closed_UVwindsalt}e,f solid line), and downwelling of fresher water at US (Figure~\ref{fig:closed_UVwindsalt}e,f dashed line) as the deeper sensor there went from measuring 18 PSU to less than 5 PSU in 13 hours while the sensors remained vertically fixed.


% figure: schematic - upwelling
\begin{figure}[h!]
\centering
	\includegraphics[width=.5\linewidth]{chapter2/figures/upwellingschematic}
\caption{High, persistent winds push surface waters and may upwell the more dense bottom waters.} \label{fig:schematicupwelling}
\end{figure}
% end figure: schematic - upwelling

Around 06:00 on November 24, wind speeds slowed to under 5 m s\textsuperscript{-1} and the water column relaxed, as evidence by flow reversal. ADCP measurements suggest that the surface mixed layer at the instrument is shallower because the depth of highest shear (where flow reverses) at the setup event is deeper than at the start of relaxation. Salinity measurements agree. Whether this is evidence of vertical mixing or longitudinal or lateral movement of water masses is not necessarily distinguishable from this data. Mid-column sensors do become fresher as the relaxation occurs, suggesting mixing. 

Large, wind-forced mixing events where persistent high wind upwells the halocline occur with some frequency. These events are significant as a mechanism of destratifying the water column, although observations suggest that at best the mid-column or deep water becomes a few PSU fresher. 

It is upon this more well-mixed water column that the mouth bar is breached in November 2010, an event possibly induced by the relaxation of wind and subsequent slosh back of setup waters. 

%*************************************************
% BREACH STATE
\subsection{Closed inlet breach and rapid transition} \label{breach_dynamics}
Our measurements can shed some light on the hydrodynamic response of the lower estuary to the inlet opening. Breach events occurred on the following days within our measurement period:
\begin{itemize}
	\item 24 November, 2010
	\item 10-11 November, 2011
	\item 20-21 January, 2012
	\item 20-21 Feburary, 2012
	\item 3-4 March, 2012
\end{itemize}
The mouth remained open after the November 2010 breach until the following August. Following the November 2011 inlet opening, the mouth again closed and continued to oscillate between an open and closed state until March 2012 at which point it remained open for the duration of our field deployment which ended on June 5, 2012. Of the events measured, 20 February, 2012 is known to be an artificial breach, triggered by the mouth being manually excavated. Whether other bar breaches were instigated by natural or artificial means is not known, although the data inform speculations that the November 2011 breach was also trigged by human intervention. Our data shows what passes by sensors in the lower estuary, but does not include measurements on top of the submerged wetlands or many measurements in the creeks upstream of the confluence.

\subsubsection{Two salt responses to inlet opening}
In general, there appear to be two manifestations of breach dynamics in the Pescadero estuary.  In one, the water column freshens and destratifies (Figures~\ref{fig:closed_UVwindsalt},~\ref{fig:breach_nov2011}, and~\ref{fig:breach_jan2012}). In a second formation, the water column does not freshen, and salt water is retained in the depths of the estuary (Figure~\ref{fig:breach_feb2012} and~\ref{fig:breach_mar2012}). In this case mixing does occur within the surface layer and within the stratified lower layer, but not between the two.

This discussion will look at the first type of breach event to generally understand what happens, and then point out differences in the breach events with limited mixing and hypothesize what causes these differences. 

Figure~\ref{fig:closed_UVwindsalt} depicts east and north velocities at DS as well as salinity at DS and US during the 24 November, 2010 breach. The lagoon water level reached a maximum at 08:20 GMT on 24 November and began to fall, slowly at first, and then rapidly from 14:45 until the water level equilibrated with the incoming tide at 07:00 the next day. In this event, the estuary water surface dropped over 1.5 m. Maximum outflowing velocity magnitudes were 0.4 m s\textsuperscript{-1}. CTDs report that by 05:30 GMT on 25 November, 2010, the water column is well-mixed at 8-9 PSU. This intermediate salinity suggests mixing has occurred. A uniform salinity of 3.5 PSU or less (the surface salinity pre-breach) could indicate that the lower estuary was flushed by fresher waters upstream, rather than indicate a mixing event.

The dramatic breach event is obvious when the lagoon was quickly draining, but velocity data and depth data suggest that there was a gradual lead up to the fast part of the breach. Data seem to suggest that relaxation of the wind-induced setup of the lagoon water surface instigated flow downstream which overtopped the sand bar. Comparing across other breach events sheds some light on processes involved in triggering the breach. 



% figure conditions QTH nov breaches
\begin{figure}[h]
\centering
	\includegraphics[width=.9\linewidth]{chapter2/figures/QTH_nov_breach}
\caption{Freshwater streamflow (a,b) into the estuary, tidal conditions (c,d) and ocean wave conditions (d,f) surrounding breach events in November 2010 and November 2011. The red line gives the time of maximum water level preceding the water draw down.} \label{fig:QbreachNov}
\end{figure}
% end figure conditions QTH nov breaches


%figure conditions QTH spring 2012 breaches
\begin{figure}[h]
\centering
\includegraphics[width=.9\linewidth]{chapter2/figures/QTH_sp2012_breach}
\caption{Freshwater streamflow (a) into the estuary, tidal conditions (b) and ocean wave conditions (c) surrounding breach events in January, February, and March 2012. The red line gives the time of maximum water level preceding the water draw down.} \label{fig:QbreachSpring2012}
\end{figure}
% end figure conditions QTH spring 2012 breaches


\subsubsection{Inlet bar breach triggers}
Streamflow measurements show an elevated freshwater discharge in conjunction with three breach events: November 2010, January 2012, and March 2012 (Figures~\ref{fig:QbreachNov}a and~\ref{fig:QbreachSpring2012}). The February 2012 breach occurred in the absence of precipitation and we know that breach to be artificial. Low streamflow in November 2011 suggests it may also have been created by human intervention (Figures~\ref{fig:QbreachNov}b). 


% figure nov 2011 breach
\begin{figure}[p]
	%\centering
	\includegraphics[width=.9\linewidth]{chapter2/figures/breach_nov2011ef} \caption{Observations of the inlet breach on 10-11 November, 2011. Plotted are wind measurements in the Pescadero marsh (a) and (b), velocity [color range $\pm$0.5 m s$^{-1}$] (c) and (d) and the DC CTD mooring measurements (e) and (f). Measurements show a rapid draining of the estuary which resulted in full mixing of the water column at the DC mooring.} \label{fig:breach_nov2011}
\end{figure}
% end figure nov 2011 breach

% figure jan 2012 breach
\begin{figure}[p]
	\centering
	\includegraphics[width=\linewidth]{chapter2/figures/breach_jan2012ef} 
	\caption{Observations of the inlet breach on 21-22 January, 2012. No velocity data is available for this event. Wind measurements in the Pescadero marsh (a) and (b) and the DC CTD mooring measurements (c) and (d) show a wind setup event early on 21 January, and a rapid draining of the estuary which resulted in full mixing of the water column at the DC mooring.} \label{fig:breach_jan2012}
\end{figure}
% end figure jan 2012 breach

% figure feb 2012 breach
\begin{figure}[p]
	\centering
	\includegraphics[width=\linewidth]{chapter2/figures/breach_feb2012ef} 
	\caption{Observations of the inlet breach on 21-22 February, 2012. Wind measurements in the Pescadero marsh (a) and (b), velocity [color range $\pm$0.5 m s$^{-1}$] (c) and (d) and the DC CTD mooring measurements (e) and (f) show slow draining of the estuary. Mixing in the deep water occurred, but complete vertical mixing is absent during this event.} \label{fig:breach_feb2012}
\end{figure}
% end figure feb 2012 breach


% figure mar 2012 breach
\begin{figure}[p]
	\centering
	\includegraphics[width=\linewidth]{chapter2/figures/breach_mar2012ef} 
	\caption{Observations of the inlet breach on 3-4 March, 2012. Plotted are wind measurements in the Pescadero marsh (a) and (b), velocity [color range $\pm$0.5 m s$^{-1}$] (c) and (d) and the DC CTD mooring measurements (e) and (f). Observations show slow draining of the estuary. Mixing in the deep water occurred, but complete vertical mixing is absent during this event. The bed CTD conductivity cell was malfunctioning during this window and so those salinity data are not shown.} \label{fig:breach_mar2012}
\end{figure}
% end figure mar 2012 breach


Within the events hypothesized to be naturally occurring, only the January 2012 inlet opening occurred with very high freshwater flows. The sand bar seems able to hold some flooding, possibly with help from wave overwash carried sand transport in conjunction with stormy conditions.  A complementary mechanism is suggested by data in the case of the November 2010 and March 2012 events. In both instances, a wind transition occurred a few hours before maximum draining of the lagoon (Figures~\ref{fig:closed_UVwindsalt}a,b and~\ref{fig:breach_mar2012}a,b). In the case of November 2010, strong winds upwelled the pycnocline and then shutoff as discussed in the previous section. In the case of March 2012, significant upwelling is not visible in CTD data measurements, but the surface sensors are temporarily slightly saltier, suggestive of movement of the stratified water masses. Similar velocity structure occurred in March 2012 with less dramatic setup and setdown followed by flow downstream of surface waters and eventually a dramatic drop in the water surface elevation as the sand bar gives way. 

In both cases, the wind shifts directions and greatly slows. Wind shear high enough to upwell the pynocline will do so via a pileup of fresher waters in the lee side of the lagoon. In the case of wind blowing from the ocean, flow is driven upstream, and gravity will cause this piled up water to return downstream when the wind shuts off. Coupled with increased freshwater flows, the slosh back may instigate the natural breach of the closed estuarine inlet. 
If this is the case, the early breach is characterized by surface flow out, and stagnant or reverse flow in the lower water column upstream, probably bathymetrically determined, but little water level drop. Slow erosion of the sandbar at the mouth the occurs while flow accelerates to the point where the breach develops rapidly. It is in the rapid fall of the lagoon water level that the highest velocities are seen and full mixing of the isopycnals occurs. 

In November 2011 and February 2012 breach events no wind setup (and setdown) preceded the breach(Figures~\ref{fig:breach_nov2011} and~\ref{fig:breach_feb2012}. In January 2012 there was was wind-induced movement of the stratified water masses, but strong wind was from the land (Figure~\ref{fig:breach_jan2012}). Wind stress in the direction of the mouth could have aided in pushing water downstream and triggering the breach. No velocity data is available from that event to shed insight on this speculation.

Not yet mentioned is the influence of the ocean wave climate on the inlet opening (Figures ~\ref{fig:QbreachNov}e,f, and~\ref{fig:QbreachSpring2012}). Large ocean waves conditions may result in sand bar overtopping from the ocean side at high tide. Several of the breach events occurred in the presence of large ocean wave climates. However, storm conditions which bring precipitation also result in large waves. Whether mouth opening is aided by ocean waves is not obvious from these data. 

\subsubsection{Mixing vs. non-mixing breach events}

Full mixing occurs when the lagoon drains quickly. Based on our limited number of observations, this occurs when:
\begin{itemize}
	\item the lagoon water level is very high, and/or
	\item freshwater discharge from the creeks is very high.
\end{itemize}
November breaches after several months of closed estuarine conditions are characterized by massive flooding of the marsh complex. In November 2011, the marsh between lower Butano Creek and the narrow estuary just below the confluence of the two creeks was flooded to the point that it was easy passable by canoe. January 2012 water levels were also high, and the draining outflow was aided by high creek discharge. The two slow drains in February and March 2012 occurred when water levels were not as high and/or freshwater streamflow was low.

General breach dynamics show that in some breach events surface flows across the entire lagoon start as a result of wind forcing, and as the sand bar erodes, flow out of the estuary rapidly transitions from surface flow to flow throughout the entire water column. This was the case in November 2010 and March 2012 (Figures~\ref{fig:closed_UVwindsalt} and~\ref{fig:breach_mar2012}). Flow structure in November 2011 seems to abruptly become nearly uniformly outflowing (Figure~\ref{fig:breach_nov2011}. With no spin up by wind, this is probably the result of manual removal of part of the sand bar. Here, flow is siphoned from the end and velocities at the ADCP (approximately 400 m upstream of the mouth) reflect flow that has developed as a drain near the mouth. Flow in the February 2012 breach is somewhat between these two cases: sudden flow starts over a large part of the upper water column, but does not penetrate to deep water (Figure ~\ref{fig:breach_feb2012}). 


\section{Annual and interannual variability} \label{sec:annvar}

The largest variability observed year to year was in the timing of mouth closure and opening in the Pescadero estuary. The estuary while open to the ocean in September 2010 was highly marine, while during the same days the following year, the mouth was closed and the character of the water column was completely different. Late summer open state conditions transform the lower estuary into an environment where marine flora and fauna have been seen and the water is very clear. Late summer closed estuary conditions produce a more turbid estuary with high vertical stratification. 

Low rainfall in water year 2012 limited the freshwater outflow available to maintain an open mouth, and so the mouth repeatedly closed and opened during Winter 2011-2012 and Spring 2012 in what was a seemingly uncharacteristic way. The drought currently occurring in California as of summer 2014 has allowed the mouth to close earlier and stay closed longer. In this way, the effect of changing hydrologic conditions being multiplied by state of the inlet to maybe completely change the status quo hydrodynamic state within the Pescadero estuary. 


\section{Conclusions} \label{backofch2}
The small size of the Pescadero estuary, as well as other bar-built estuaries in California allows them to respond quickly to external forcing. Closed-state dynamics in the Pescadero estuary are set by the stratifying inputs of freshwater and salt water and destratifying inputs from wind stress. This wind forcing may play a role in starting the closed mouth breach. 

Wind, wave, freshwater and tidal forcing in the Pescadero estuary are all dependent on the state of the inlet.  While closed, tides bring water near the top of the berm and waves allow salt water to overtop the berm, growing the sand berm and supplying a salt water source to the estuary. While open tides contribute momentum, and move salt in and out of the estuary. Freshwater inputs in the open state are often small compared to tidal exchange (except after large storm events, which were infrequent during our study). Late-summer freshwater flow may be negligible if the inlet is open, but if the inlet is closed these inputs slowly accumulate and become a thick fresh surface layer. This layer impedes mixing by wind and limits the exchange of thermal heating and oxygen into stratified or saline bottom waters. Wind drives flow while the estuary is closed, but its effect during the open state is not necessarily noticeable in the face of dominant tidal forcing. Predictive capabilities in estuaries such as the Pescadero will not be possible without an understanding of inlet dynamics superior to our current state of knowledge.  Processes at work in determining inlet conditions include the nonlinear interactions between waves, swash, tides, river flow and their influence on sediment resuspension, transport, and deposition. And, in return the changing morphology alters these processes, creating a dynamic and difficult interplay to understand. 
