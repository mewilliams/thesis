\chapter{The Pescadero estuary}
\label{chPescadero}

Background on the estuary

* big picture things: location, size of stuff *

The Pescadero estuary is located at the confluence of Pescadero Creek and Butano Creek in southern San Mateo County on the California coast. The estuary drains a watershed of 210 km\textsuperscript{2} of the Santa Cruz mountains, located in California's coastal range. The coastal range has a rainfall dominated hydrology, and freshwater streamflow on the creeks is flashy and seasonal. 

* other research specific to this estuary * 

% INSERT CALIFORNIA COASTAL RECORDS PROJECT PHOTO OF THE PESCADERO ESTUARY HERE. 


A - Historical Context
A history of the Pescadero marsh was compiled as the M.A. dissertation work of [Viollis, 1979]. The marsh used to be bigger, has a history of being farmed, drained, duck hunting ground... biggest changes in the 20\textsuperscript{th} century (as written in 1979) because of construction of levees and roads, logging, increased agriculture in the watershed as well as increased water use. 


B - Current problems (fish kills?)
- cite WSJ article. 2 MS thesis from SJSU
In its altered state, the Pescadero estuary has experienced fish kill events during the fall breach of the inlet for many of the past X years. This fish mortality has drawn studies (Sloan, 2006) as well as [anger] from local residents {cite WSJ}. To date a conclusive explanation for the triggers of fish mortality does not exist. 


C - The scope of what I can say. (and maybe some disclaimers on what I can't). 

%***********************************************************************************************************************************************************************************************
% SITE STUFF
\section{Measurements}
\label{measurementslabel}



--------------------------------------
* Needs site description: bathymetry, depth of mouth, bedrock there. 
* MLLW to water level conversion. 

Our observations: (not measuring sand)
* measurement setup (lead weight, milk crates, line, subsurface buoys,...). [PICTURES]. Problems with early setup. Suggestions for anyone in the future making measurements there. (Comment on driftwood and car wheels). 

Depth measurements were made using the pressure transducer on CTDs attached to milk crates... The RBR pressure values were then adjusted to changing atmospheric pressure using air pressure measurements from the Half Moon Bay airport (KHAF). An approximate density of $\rho$ = 1021 kg m\textsuperscript{-3} is used with the hydrostatic pressure equation ${p = \rho g h}$, slightly overestimating the depth when the estuary is fresh and density is lower than this value. Error in the pressure measurement using this assumption may be up to 2.5\%. Further error is introduced comparing pressure measurements in a quiescent water column to a quickly moving flow. Stagnation pressure p is actually p = 0.5 $\rho$ g h +  v\textsuperscript{2} (cite general fluid mechanics textbook). Under quiescent flow, pressure values measured are hydrostatic pressure. Under small velocities, error in this assumption is small. If flow reaches 1 m s\textsuperscript{-1} in a 1 m water column, the pressure reading will increase by 5\% therefore depth calculated from these measurements not accounting for dynamic pressure will be slightly overestimated.

We observe error in comparison of recorded water level (according to the staff gauge on the highway 1 bridge) while the estuary is draining in February and March 2012 and expect this error to be due to dynamic pressure readings. Thus, to set the water level in the estuary to a NAVD88 or MLLW datum, we use measurements taken during the closed water state and assume quiescent flow.  Further error is introduced as moorings were weighted to the ground but not fastened in place and some movement of the bed or mooring may have occurred. 

Field measurements were taken in the Pescadero estuary during X field deployments. 

1 - March - May 2010
2 - September 2010
3 - November 2010 - May or June 2011
4 - August 2011 - June 5, 2012
5 - October 2012 (but not really)

Later deployments utilized more instruments and thus had better resolution in space.  The frequency of measurements was also increased in later deployments with the realization that infragravity forcing was present. A meteorological station measuring wind speed and direction was installed in {September 2011} and remained in place until June 2012.

Summary of range of conditions in all data collected:
Salinity : 0 to [highest]
Temperature: cold to hot
Windspeed and direction: usually two directions, probably because of geometry of marsh. Fast to slow. 
Water depth? [Depends on how much I can get set to a datum]. 


%% Wind measurements paragraph:

Wind measurements were made using a [Campbell Scientific Met Station thingy] during the Fall 2011 - Spring 2012 field deployment. The wind vane was mounted approximately 3 m above the water level in the marsh adjacent to the estuary (See location on a figure).  Measurements show that wind direction through the marsh is almost entirely bi-directional (Figures winddir vs. time and wind rose). Wind direction of [360-ish degrees] is wind blowing from the ocean and from the other direction [180ish degrees] is land-sourced wind (is that called land breeze?). This constrained directionality is attributed to topography of the marsh.  Landward wind from the ocean is constrained by high topography on the South end of the estuary and marsh. On the northern end of the estuary inlet, sand dunes and constraint by the [stuff ] supporting the Highway 1 bridge further constains flow, essentially funnelling wind flow through a tunnel.  In the opposite direction, we hypothesize that the high topography in the south constrains flow, and since the marsh is [sorta] located in a valley, [flow is two-directional]. 

Windspeeds from the ocean are higher than those from the land (Figure wind rose, wind vs. time). 

\begin{figure}
  \includegraphics[width=0.8\textwidth]{chapter2/figures/wind_vs_time_20111027_20120510.eps}
    \caption{These are wind measurements in the Pescadero marsh. (a) is wind speed [m s\textsuperscript{-1}] at 3 m above the water surface. (b) is wind direction in degrees.}
 \label{metstn_pdo_ws_wdir}
 \end{figure}
 
 \begin{figure}
  \includegraphics[width=0.5\textwidth]{chapter2/figures/windrose_20111027_20120510.eps}
    \caption{These are wind measurements in the Pescadero marsh.}
 \label{metstn_pdo_windrose}
 \end{figure}


Wind stress values based on wind speed measurements are calculated by....:  (Mary - friction velocity using Yelland and Taylor values for windspeed based drag coefficients). 

% FIGURE THAT COMPARES KHAF WIND MEASUREMENTS TO PESCADERO MEASUREMENTS.


%***********************************************************************************************************************************************************************************************
% CONDITIONS from observations etc. 
\section{Conditions... 2010-2012}
\label{conditions_label}

The years 2010-2012 encompassed [? El ninho, la ninha, drought].

Tides: 
The nearest ocean tide gauges to the Pescadero estuary are operated by the National Oceanic and Atmospheric Administration (NOAA) at Crissy Field in San Francisco, California (Station XXXX) as well as in Monterey Harbor [???] in Monterey, Califoria (Station XXXX). Water level in both locations are recorded every six minutes, and one minute tsunami-capable unverified data is also recorded at these stations. 

Freshwater:
Freshwater streamflow into the Pescadero estuary is estimated based on a United States Geological Survey (USGS) gauge located on Pescadero Creek [X km] upstream from the mouth of the estuary (USGS XXXXXXXX). This gauge records water height and calculates freshwater discharge every [X min]. The gauge is downstream of [X] \% of the watershed, so to estimate freshwater flow into the estuary the gauged discharge measurement is multiplied by [X = 1.76, eu acho]. Unaccounted for freshwater diversions are thought to exist downstream of the gauge which likely reduce freshwater flow into the estuary, however we do not have a way to quantify these diversions and thus will report freshwater flow as if they do not exist.

The Mediterranean climate of California [comes with] seasonal precipitation. [X] \% of coastal California's [X cm] of annual precipitation falls between the months of [X and X], and freshwater flow in the Pescadero Creek follows this trend. [Insert Figure: Q vs. t for all t (median) and 2010- 2012 overlaid]. 

Ocean waves:
Ocean wave conditions are obtained through NOAA National Data Buoy Center (NDBC). NDBC buoy 46012 is located approximately 40 km WNW of the Pescadero estuary in 208.8? m of water. This buoy reports ocean significant wave height ever [X min]. [Note on deep water vs. coastal]. Maximum wave heights in the ocean from 2010 - 2012 were [X m], and minimum were [X m], while the range of ocean significant wave heights while our sensors were recording in the Pescadero estuary was [X m to X m]. 

Wind: 
Given that wind measurements were not made until the last field deployment from September 2011 - June 2012, wind conditions in the estuary during other periods are approximated by XXXXX. There is approximate aggreement in [Wind Speed? Direction], as shown in [FIGURE]. 

Tsunami:
Furthermore, the 2011 Tohoku earthquake generated a Pacific Ocean-wide tsunami (also known as a teletsunami). A description of the Pescadero estuary's response to the forcing of the small amplitude tsunami that reached California is found in Chapter 5. 


%***********************************************************************************************************************************************************************************************
% OPEN & CLOSED, etc.
\section{Two main states: open and closed, plus some intermediate stuff}
\label{betterlabelmaybe}

With these observations, we have identified two main conditions for the estaury: open and closed.  The inlet continuously contracts and [gets bigger] with nearshore sand transport. Within the open state, tides in the estuary may be an attenuated version of the ocean tides [Figure], and this state we will consider to be fully open. An extensive description of relevant hydrodynamic and salt transport processes in the fully open state is found in Chapter 3 and 4. {Some words later on interannual variability to open state? How much freshwater it takes to freshen the estuary?}. 

Because the estuarine inlet is in a constantly unstable condition, an intermediate state between fully open and closed is also common in the Pescadero estuary.  In this state, the sandy inlet is constricted, probably due to accretion of sand at the bed of the inlet. Here, the estuarine water level rises and falls in response to ocean tides, but may only fill with the diurnal high high tide [Figure] or be severly attenuated [Figure]. We consider this condition to be a highly-constricted open state. Realistically, the open state of the estuary is a continuum. Sand moves in and out of the inlet with tidal flow and waves. High freshwater discharge will scour the mouth. 

The closed state of the estuary is characterized by tidal depth oscillations being shut off. This does not mean that the ocean tides cease to influence water level in the estuary but that the water is not outflowing through the channel. Variation in forcing within the closed state depends on length of closure, to be discussed in section {X}. 

Transitions between these two states are rapid. 

[A note on closure.] 

And, the transition from the closed state to open state in the Pescadero estuary naturally occurs only when increased freshwater streamflow overtops the sand bar at the mouth. It has been suggested in other estuaries that storm waves overtopping the bar may breach the estuary from the ocean side [CITE], but the mouth of the Pescadero estuary is somewhat protected by the southern end cliff, and wave conditions observed during our years of study make this unlikely. Given a storm surge similar to that carried by Superstorm Sandy which created new inlets in Barnegat Bay [CITE], this could be possible, but these are very unlikely to occur in California's climate. 

A description of estuarine dynamics during the closed and transitioning states follows in this chapter, with the description of dynamics in the open state to follow in the proceeding chapters. 

%*********************
% 2.3.1 Open state
\subsection{"Open"}
\label{opench2}



%***********************************************************************************************************************************************************************************************
% How the estuary closes - 2.3.2
\subsection{Closing estuary note}
\label{closingest}
A closed lagoon is defined based on no significant flow out of the estuary as the ocean tide recedes. Depth measurements are used to determine flow out. [FIGURE]

%***********************************************************************************************************************************************************************************************
% CLOSED STATE - 2.3.3
\subsection{Closed State}
\label{closed dynamics}


[FIGURE Needed: water level: closure through breach]

\begin{figure}
  \includegraphics[width=\textwidth]{chapter2/figures/mouth_fb_20110825_recentlyclosed.jpg}
    \caption{This is a photo of the mouth of the Pescadero estuary on August 25, 2011.}
 \label{mouth_fb_20110825}
 \end{figure}


The mouth of the Pescadero estuary closes when \emph{...[basically waves so long as freshwater is low]...} Details about lagoon closure are outside of the scope of this research, but relevant discussions on the conditions necessary for lagoon closure can be found in [Dane's thesis and his geomorphology paper]... The mouth of the Pescadero estuary is somewhat unique in its protection by a cliff to the south side [Image - mouth of Pescadero]. This cliff as well as the Highway 1 bridge piers and \emph{[ground/dune stabilization]} prevent the migration of inlets seen in some other California intermittently closed estuaries [CITE - Dane's GRL paper?]. 

Immediately after closure, the mouth of the Pescadero estuary is still low in elevation [IMAGE: 8/25ish/2011]. Throughout the closed period, we expect the height of the sand berm to be equal to or slightly above the height of the estuary water surface.

Initial lagoon infilling occurs primarily to ocean water overtopping the low sand berm. Figure {FIGURE} shows the estuarine water level and ocean water level for [PERIOD OF TIME]. 

There are at least two mechanisms by which ocean water enters the Pescadero estuary during the closed mouth state.
(1) Waves overtopping berm directly into the estuary.
(2) Waves overtopping the steep (bermed?) beach, running down the beach and into the estuary: creates river of sea water. 

The second mechanism has only been observed once by the author (on DAY), and on the day it occurred the lagoon water level raised [X cm]. A day [a few days earlier] with a smaller wave climate only raised the estuary water level by [X cm]. 

It is assumed that wave overwash carries salt water into the estuary and grows the sand berm. A study on the [X estuary] in [somewhere] Australia showed that \emph{XXXX} [Baldock et al., 2008]. 



Due to the hypsometry of the marsh, intial wave overwash is most relevant to raising the water level in the closed lagoon. After a certain point, the surface of the lagoon begins to spill across the marsh and at this point filling becomes much more gradual. [Does the berm get high enough to prevent ocean water from getting in?]. Because nearshore forcing is primarily responsible for sand transport accreting the sand berm at the mouth of the estuary, it is highly unlikely that a condition exists where the mouth becomes high enough that ocean water cannot enter. [Sand transport by wind may also shape the mouth...]

[some data to look into:]



VELOCITY
While the estuary is open to tidal influence, tides are the main driver of flow (in the absence of very large freshwater discharge). After the inlet closes, tidal forcing is cut off. Velocity measurements show that: {What do velocity measurements show?}

STRATIFICATION:
SALT
* Evolution of stratification (saltier, freshwater surface layer). 
* Wedderburn number. The Wedderburn number is a nondimensional number used to compare wind stress (?) to stratification and is an indicator of [what it would take to mix a stratified water column]. The number is a comparsion of [quick derivation here]. 


* Maybe a potential density anomaly?


* Mixing events:
The high Wedderburn number means that the freshwater sitting on the surface of the estuary acts to protect salt-stratified lower water from mixing due to wind stress induced shear... Occasionally the wind blows (hard enough and/or long enough) to upwell the pycnocline. [Figures - wind, salt]. If this occurs, the wind stress can act directly on the pycnocline. [Descriptions of the figures here].  THis appears to be the primary mechanism of significant mixing during the closed state.  [February 13, 2012]. 



Hypothesize on mixing of intruding salt. Some parameterization?

TEMPERATURE. (Important to fish).
Stratification in the Pescadero estuary is always dependent on salt. [Because equation of state]. Temperature in the estuary in the closed state{HERE} depends on the salt stratification. Observations show that after the estuary has closed and a freshwater layer is present, water temperature in the mixed surface waters follow a diurnal trend of warming and cooling in accordance with solar heating. The salt-stratified lower layer of the estuary does not 'feel' this diurnal heating, and in fact is much warmer than the surface water. [Figure: temperature during closed state]. 


An observed example of wind setup and mixing occurred just prior to the sand bar breach on November 24, 2010. In situ velocity and salinity measurements combined with wind measurements at Half Moon Bay airport are used to understand this event:

On November 23, wind forcing was seaward, small surface velocities in the Pescadero estuary were consistent with this forcing, and the water column was vertically stratified. Three layer flow apparent in this state may be due to the bathymetry of the estuary where the lower end of the estuary is very shallow. Wind shear may drive surface flow downstream, and the lower water column will respond by flowing upstream. At the sensors, [sentence on trapped lower water]. 

From this point, the wind direction slowly shifted to become landward, and on November 23 at (X pm) GMT, the wind picked up to 10 m s\textsuperscript{-1}. The water column responded immediatedly to this increased wind shear with surface flow in line with the wind and the deeper water column compensating with flow in the other direction. Three-layer flow may not be seen because the upper estuary is deeper and the end of the estuary [has a sharp wall instead of a shallow lagoon thing]. Salinity measurements suggest that the strong wind forcing caused slight upwelling of saltier water at the DS mooring, and significant downwelling of fresher water at US as the deeper sensor there went from measuring [15-20] PSU water to [5] PSU water in [X] hours. 

At [XX:XX] on November 24, the wind falls down to 5 m s\textsuperscript{-1} and the water column relaxes, as evidence by flow reversal. The surface mixed layer at the ADCP is shallower as seen by comparing the depth of highest shear at the setup event several hours earlier to the relaxation event, as well as by looking at the salinity sensors. As the relaxation occurs, the [surface mixed layer becomes thicker], but mid-column sensors end up fresher than prior to the mixing event. 

It is upon this slightly more well-mixed background that the lagoon mouth breach occurs, and it is difficult to determine an exact moment of estuarine breaching...

***
Three (?) things drive mixing in this state. The first: direct input of wind shear at the surface creating shear in the water column.  The second is shear by interfacial drag mid-column. And a third process may be movement of the lower water column against the bed creating [... more shear]. 

These effects may be parameterized in the following way: 



%***********************************************************************************************************************************************************************************************
% BREACH STATE
\subsection{Breach.. }
\label{breach_dynamics}


[FIGURE Needed: water level: closure through breach]

The mouth of the Pescadero estuary closes when ...\emph{basically waves so long as freshwater is low}... Details about lagoon closure are outside of the scope of this research, but relevant discussions on the conditions necessary for lagoon closure can be found in [Dane's thesis and his geomorphology paper]... The mouth of the Pescadero estuary is somewhat unique in its protection by a cliff to the south side [Image - mouth of Pescadero]. This cliff as well as the Highway 1 bridge piers and {ground/dune stabilization} prevent the migration of inlets seen in some other California intermittently closed estuaries [CITE - Dane's GRL paper?]. 

Immediately after closure, the mouth of the Pescadero estuary is still low in elevation [IMAGE: 8/25ish/2011]. Throughout the closed period, we expect the height of the sand berm to be equal to or slightly above the height of the estuary water surface.

Initial lagoon infilling occurs primarily to ocean water overtopping the low sand berm. Figure [FIGURE] shows the estuarine water level and ocean water level for [PERIOD OF TIME]. 

There are at least two mechanisms by which ocean water enters the Pescadero estuary during the closed mouth state.
(1) Waves overtopping berm directly into the estuary.
(2) Waves overtopping the steep (bermed?) beach, running down the beach and into the estuary: creates river of sea water. 

The second mechanism has only been observed once by the author (on DAY), and on the day it occurred the lagoon water level raised [X cm]. A day [a few days earlier] with a smaller wave climate only raised the estuary water level by [X cm]. 

It is assumed that wave overwash carries salt water into the estuary and grows the sand berm. A study on the [X estuary] in [somewhere] Australia showed that ... [Baldock et al., 2008]. 



Due to the hypsometry of the marsh, intial wave overwash is most relevant to raising the water level in the closed lagoon. After a certain point, the surface of the lagoon begins to spill across the marsh and at this point filling becomes much more gradual. [Does the berm get high enough to prevent ocean water from getting in?]. Because nearshore forcing is primarily responsible for sand transport accreting the sand berm at the mouth of the estuary, it is highly unlikely that a condition exists where the mouth becomes high enough that ocean water cannot enter. [Sand transport by wind may also shape the mouth...]

[some data to look into:]



VELOCITY
While the estuary is open to tidal influence, tides are the main driver of flow (in the absence of very large freshwater discharge). After the inlet closes, tidal forcing is cut off. Velocity measurements show that: {What do velocity measurements show?}

STRATIFICATION:
SALT
* Evolution of stratification (saltier, freshwater surface layer). 
* Wedderburn number. The Wedderburn number is a nondimensional number used to compare wind stress (?) to stratification and is an indicator of [what it would take to mix a stratified water column]. The number is a comparsion of [quick derivation here]. 


* Maybe a potential density anomaly?


* Mixing events:
The high Wedderburn number means that the freshwater sitting on the surface of the estuary acts to protect salt-stratified lower water from mixing due to wind stress induced shear... Occasionally the wind blows (hard enough and/or long enough) to upwell the pycnocline. [Figures - wind, salt]. If this occurs, the wind stress can act directly on the pycnocline. [Descriptions of the figures here].  THis appears to be the primary mechanism of significant mixing during the closed state.  [February 13, 2012]. 



Hypothesize on mixing of intruding salt. Some parameterization?

TEMPERATURE. (Important to fish).
Stratification in the Pescadero estuary is always dependent on salt. [Because equation of state]. Temperature in the estuary in the closed state{HERE} depends on the salt stratification. Observations show that after the estuary has closed and a freshwater layer is present, water temperature in the mixed surface waters follow a diurnal trend of warming and cooling in accordance with solar heating. The salt-stratified lower layer of the estuary does not 'feel' this diurnal heating, and in fact is much warmer than the surface water. [Figure: temperature during closed state]. 

-----------------------------------
breach data: 
salt response
velocity...? (backward order?)


BREACH:
Instruments recorded 5 (I think) opening events in the Pescadero estuary. The date of the breach, maximum Q in the [24?] hours preceding the breach, water level before the breach (maybe), ocean wave climate, and [tide something] are given in table [TABLE OF BREACH DATA THAT SEEMS RELEVANT TO ME]. 

{Disclaimer: No measurements on top of marsh, few measurements in creeks... data can show what passes by sensors in lower estuary. When I say estuary, I mean estuary where measurements were made. High water level means salt can extend farther upstream than normal... didn't measure there}. 

Based on measurements as the sand berm at the mouth breached, two types of events occurs: one where the entire (lower) estuary is freshened, and the second where mixing occurs in the fresh surface layer and the stratified lower layer, but salt water is retained in the depths of the estuary. [Could maybe do a HOW MUCH ENERGY TO FLUSH BOTTOM WATERS estimation]. 

Here I present data of salt and/or velocity from as many breach events as I can present data for.

1 - November 2010
2 - November 2011
3 - January 2012
4 - Feburary 2012
5?

%***********************************************************************************************************************************************************************************************
% FIGURES

\begin{figure}
  \includegraphics[width=\textwidth]{chapter2/figures/depth_closure_fall2011_zoomed.eps}
    \caption{Test caption.}
 \label{depthclosuref11}
 \end{figure}


%**********************************************************************************************************************************************************************************************
\section{Anything else?}
\label{backofch2}

\emph{field measurement suggestions? how to answer things i can't answer?}


