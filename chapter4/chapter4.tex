\chapter{Estuarine dispersion}
\label{chSalt}

\emph{(alternate title suggestions accepted)}

\section{Introduction}

Salt transport in small, bar-built estuaries tends to be frontal. Sharp gradients exist between freshwater and salt water, both vertically and horizontally. In the open state of the Pescadero estuary, salt moves in and out as described in chapter 3. This general description treats salt water and freshwater as relatively immiscible, a character deemed untrue by decades of estuarine physics research and millenia of cooking pasta. Here, then, I set out to characterize salt mixing and dispersion in the Pescadero estuary. 

Mixing and dispersion in estuaries is due to a combination of vertical and lateral processes. Shear dispersion,... tidal straining \parencite{Simpson:1990aa}, tidal trapping,.. \emph{also baroclinic processes on subtidal timescales that don't exist in these little estuaries}

In estuaries with peripheral channels or margins, tidal trapping increases longitudinal dispersion processes \parencite{okubo_effect_1973}. 

\parencite{Ralston:2005aa}: \emph{Similar thing, different estuary}
Ralston and Stacey \cite{Ralston:2005aa} used a 1-d framework to quantify frontal mixing within the intertidal zone of mudflats in San Francisco Bay. They found \emph{K$_x$} of 10 m\textsuperscript{2} s\textsuperscript{-1} of the frontal salinity structure...

Salt is a conservative tracer, making transport and mixing of salinity a quantifiable term. Implications of salinity transport and mixing can help to characterize similar effects on other terms such as nutrients, temperature, plankton, or suspended sediment. 

\section{Methods} \label{sec:ch4methods}

%% 1D Advection-diffusion
\subsection{1D advection-diffusion} \label{ssec:1dadvdif}

We start with the depth-averaged 1-d advection-diffusion of salinity,
\begin{eqnarray}
\frac{\partial S}{\partial t} + U\frac{\partial S}{\partial x} = K_x\frac{\partial^2S}{\partial x^2} \label{eq:1dadvdiff}
\end{eqnarray}
where \emph{S} is salinity, \emph{U} is depth-averaged velocity, and \emph{K$_x$} is the longitudinal dispersion coefficient. In an estuary, the salt structure is advected upstream and downstream with the tides. Moving the frame of reference along with these tides, \emph{U=0} relative to the reference frame and equation~\ref{eq:1dadvdiff} becomes
\begin{eqnarray}
\frac{\partial S}{\partial t} = K_x\frac{\partial^2S}{\partial x^2} \label{eq:1ddiff}
\end{eqnarray}
Assuming salinitiy follows a step funtion at time $t=0$ and that the boundaries of the system remain at ocean salinity, $S=S_0$ and freshwater salinity $S=0$, an analytical solution to equation~\ref{eq:1ddiff} can be obtained,
\begin{eqnarray}
S(x,t) = \frac{S_0}{2}\left(1+erf\left(\frac{x-x_c}{\sqrt{4K_xt}}\right)\right) \label{eq:S}
\end{eqnarray}
where \emph{x$_c$} is the center of the salt front. Using observations of the salinity field, the value of \emph{K$_x$} can be obtained from this analytical solution.


%%% Observations --> Analytical framework
\subsection{Using observations in an analytical framework} \label{ssec:ObsInto1DAdvDiff}
To solve for \emph{K$_x$} in equation~\ref{eq:S}, the salinity structure in time and space within an estuary must be known. Salinity measurements were made in the Pescadero estuary with good temporal resolution and some longitudinal distribution during an April - June 2012 field campaign. However, longitudinal salinity gradients in the Pescadero estuary as well as in other small bar-built estuaries are very sharp, so locating instruments close enough to each other to directly measure the longitudinal gradient and getting spatial coverage to understand dynamics of a whole system would be nearly impossible. So, we resort to recreating the salinity structure in time using measurements from a collocated ADV and CTD. 

The sensors give a time series of salinity and velocity (0.2 m above the bed at NM). Velocity is the rate of change of distance with time,

\begin{eqnarray}
U = \frac{dx}{dt} \label{eq:uEdxdt}\\
\end{eqnarray}
so integrating velocity,
\begin{eqnarray}
\int{Udt} = \int{dx} = X \label{eq:intuEx}
\end{eqnarray}
transforms the time series of point measurements of velocity and salinity (Figure~\ref{fig:UandSvsTch4}) into a spatial representation of salinity ($S(x)$, Figure~\ref{fig:SvsXall}).

%%%% Data Processing Specifics
\subsection{Data processing} \label{ssec:DataProcessing}
To make this dataset, velocity measurements were burst-averaged (the instruments collected data for 512 s out of every 900 s) and an 800 s moving average was applied to the CTD measurements.  Averaging was intended to remove the infragravity oscillation component of velocity and salinity measurements to focus on tidal processes. 

To get \emph{K$_x$} from the recreated \emph{S(x)}, equation~\ref{eq:S} is rearranged into the linear form:
\begin{eqnarray}
erf^{-1}\left(\frac{2S}{S_0}-1\right) = \frac{x-x_c}{\sqrt{4K_xt}} \label{eq:linfitS}
\end{eqnarray}
and a linear fit ((y = ax + b).. a = $\frac{1}{\sqrt{4K_xt}}$ and b = $\frac{-x_c}{\sqrt{4K_xt}}$) is applied. Given a linear regression fit value for \emph{a}, to solve for $K_x$, a value of \emph{t} must be determined. In this case it is appropriate to use the time between when the front floods past the sensors and subsequent ebb of said front. This method assumes an infinitely sharp front at time $t_0$. Setting $t_0$ as the time at which the front floods past the sensors will overestimate \emph{K$_x$}. Thus, an estimation on the initial time from a sharp front to the flood front is made. By setting a step function at the mouth, and approximating advection velocities U = 0.1 m s$^{-1}$ and a length from the ocean to NM of 100 m, initial time, t$_0$ is 1000 s. This time is then added to the time between a selected isopycnal passing by the sensor. Specific considerations in selecting the isopycnal to follow, and influence of $t_0$ will be discussed in section~\ref{sec:Sensitivity}.



\section{Observations} \label{sec:ObsCh4}
Time series of collocated velocity and salinity measurements at NM are shown in figure~\ref{fig:UandSvsTch4}. Gaps in measured data occur when the water level is very low and sensors were out of the water. In response to coastal California's mixed semidiurnal tide, the time series shows that salt water does not remain in the system on the low high tide  as long as it does on the high high tide suggesting limited tidal excursion of the salt field. Here we focus only on the tidal cycle associated with high high water level, referred to in this test as the \textquotedblleft large tide.''

\begin{figure}[h]
	\includegraphics[width=\textwidth]{chapter4/figures/UandSvsT} 
\caption{Velocity (a) and salinity (b) records 20 cm above the bed at station NM. Velocity measurements have been averaged across their X s burst and rotated into the direction of principle flow. An 800 s moving average has been applied to salinity measurements. Averaging was done to remove higher frequency oscillations.} \label{fig:UandSvsTch4}
\end{figure}

The time series plots of salinity and velocity (Figure~\ref{fig:UandSvsTch4}b) show that the transition from fresh to ocean salinity water occurs rapidly on the flood and the return to fresh water on ebb is a longer duration process. Also, the ebb salinity field passes the sensors in two distinct segments. On ebb, lower salinities demonstrate a much more relaxed salinity structure than higher salinities. The flood salinities do not exhibit this feature. To investigate whether this observation is an actual transition in salinity and not just a reduction in water velocity, as well as to approach a quantification of \emph{K$_x$}, the time series is transformed into a spatial record as described in section~\ref{sec:ch4methods}. 

\begin{figure}[hp]
	\includegraphics[width=\textwidth]{chapter4/figures/SvsXfloodebb} 
\caption{S(x) for four floods - April 2012, 20cm above bed, Near Mouth (a) and four ebbs (b). In both figures, the profiles are centered (x=0) around the S = 15 PSU pycnocline.} \label{fig:SvsXall}
\end{figure}

Figure~\ref{fig:SvsXall} shows the salt front on flood (a) and on ebb (b) for the four large tides on 21 - 25 April, 2012. Spatial mapping of the salinity field shows that the estuary transforms  the salinity front from a smooth, sharp structure on flood to a two-part relaxed structure on ebb, where $S(x)$ is much more relaxed among lower salinities than among higher salinities. This mirrors the time series salinity and implies that there is a physical change in the salt field between flood and ebb. This non uniform structure requires the calculation of two different values of \emph{K$_x$} for the higher and lower salinities.

By fitting the error function curve to the two ebb salinity fields determined by the salinity of the transition point ($S_T$), values of \emph{K$_x$} are thus calculated for the higher ($K_{x,S_T\rightarrow S_0}$) and lower salinities ($K_{x,0\rightarrow S_T}$) in the four tidal cycles (Table~\ref{tab:Kxtable} and Figures~\ref{fig:Kx421}-~\ref{fig:Kx424}). Averaged across the four days, \emph{K$_x$} = 4.1 m$^2$s$^{-1}$ at higher salinity and \emph{K$_x$} = 45.4 m$^2$s$^{-1}$ at lower salinity. Values of \emph{t} for \emph{K$_x$} calculations are taken to be the time between when the 8 PSU and 20 PSU isopycnals flood and ebb past the sensor for lower and higher salinity \emph{K$_x$} values, respectively. 

Values of \emph{K$_x$} show that dispersion is much greater at lower salinities than at higher salinities in the Pescadero estuary.  Values of \emph{K$_x$} are an order of magnitude higher in lower salinities than in higher salinities. Dispersion coefficients in lower salinity water (\emph{K$_{x,S=0 \rightarrow S_T}$}), are higher in the first two days than in the last two. Dispersion coefficients in higher salinity water (\emph{K$_{x,S=S_T \rightarrow S_0}$}) are lower in the first two days than in the second two days. 

% --------------TABLES ---------------------------------------------------------------%
\begin{table}[h]
\renewcommand{\arraystretch}{1.3}

	\begin{center}
		\begin{tabular}{| l || c | c | c | c | c | c |}
		\hline
		Date & S$_{T}$ & K$_{x,S=0 \rightarrow S_T}$ &  K$_{x,S=S_T \rightarrow S_0}$ & x$_{c,fresher}$ & x$_{c,fresher}$\\
		\hline \hline
		21 Apr & 11.5 & 48.7 m$^2$ s$^{-1}$ & 2.3 m$^2$ s$^{-1}$ & -591 m & -9.4 m\\ 
		22 Apr & 10.5 & 65.1 m$^2$ s$^{-1}$ & 3.1 m$^2$ s$^{-1}$ & -713 m & -60.2 m\\ 
		23 Apr & 14.4 & 31.0 m$^2$ s$^{-1}$ & 5.2 m$^2$ s$^{-1}$ & 34.2 m & 18.95 m\\ 
		24 Apr & 14.1 & 31.6 m$^2$ s$^{-1}$ & 5.9 m$^2$ s$^{-1}$ & -137.3 m & -3.9 m\\  \hline 
	\end{tabular}
	\caption{Values of \emph{K$_x$} for the large flood and ebb on four days  in April 2012.}\label{tab:Kxtable}
\end{center}
\end{table}
% -------------------------------------------------------------------------------------%


Comparison to theory, hypothesized mechanisms of dispersion, reasons for the two-phase structure of \emph{S(x)}, and changes with physical forcing are discussed below. 
% Kx figures
\begin{figure}[hp]
\centering
	\includegraphics[width=.75\textwidth]{chapter4/figures/SvXwKx_124_165} 
\caption{Spatially reconstructed salinity field and \emph{K$_x$} values for the large tidal cycle on 21 April, 2012. } \label{fig:Kx421}
\end{figure}

\begin{figure}[hp]
\centering
	\includegraphics[width=.75\textwidth]{chapter4/figures/SvXwKx_183_263} 
\caption{Spatially reconstructed salinity field and \emph{K$_x$} values for the large tidal cycle on 22 April, 2012. } \label{fig:Kx422}
\end{figure}

\begin{figure}[hp]
\centering
	\includegraphics[width=.75\textwidth]{chapter4/figures/SvXwKx_279_357} 
\caption{Spatially reconstructed salinity field and \emph{K$_x$} values for the large tidal cycle on 23 April, 2012. } \label{fig:Kx423}
\end{figure}

\begin{figure}[hp]
\centering
	\includegraphics[width=.75\textwidth]{chapter4/figures/SvXwKx_376_454} 
\caption{Spatially reconstructed salinity field and \emph{K$_x$} values for the large tidal cycle on 24 April, 2012.} \label{fig:Kx424}
\end{figure}
% end Kx figures




\section{Comparison to theoretical ways to estimate $K_x$}

Longitudinal dispersion caused by vertical shear \emph{(in an unstratified water column?)} was derived by Elder to be
\begin{eqnarray}
K_x = 5.93hu_* \label{eq:Kshear}
\end{eqnarray}
where \emph{h} is the height of the water column and \emph{u$_*$} is the shear velocity defined by $u_*=\sqrt{\tau_0/\rho}$ where $\tau_0$ is a bed stress and $\rho$ is water density \parencite*{elder_dispersion_1959}. The friction velocity is parameterized as $u_* = c_{_D}^{1/2}U$. In the Pescadero estuary, \emph{h} ranges from 0.2 m to 2.7 m depending on location in the estuary and tidal conditions, \emph{U} is $\pm$ 0.4 m s\textsuperscript{-1} with spring tides, and surges due to infragravity motions may result in positive velocities of 0.6 m s\textsuperscript{-1}. Estimating c$_D$ as 0.01, the largest \emph{K$_x$} by this formulation would be \emph{K}$_x =$ 0.96 m$^2$ s$^{-1}$ using the maximum value of \emph{h} and the infragravity velocity of $U = 0.6$ m s$^{-1}$. More likely values of $h = 1 m$ and $U = 0.3 m s^{-1}$ result an estimated \emph{K}$_x = 0.18$ m$^2$ s$^{-1}$. In either case, this value of \emph{K$_x$} is well below observations.

\emph{longitudinal dispersion via tidal trapping...}
\begin{eqnarray}
K = \frac{K^{'}}{1+r} + \frac{ru_0^2}{2k(1+r)^2(1+r+\omega / k)} \label{eq:KxOkubo}
\end{eqnarray}

\emph{If u0 is .3, r = .5, k and omega are what are in fischer... the second term is 34.5 (first is .66K')... If I figure out what that longitudinal diff term is, can get r variation with tide? Except I expect more trapping with spring tide, and my values from the 4 days suggest opposite...}

\section{Discussion} \label{sec:discCh4}

Spatial mapping of the salinity field from time series of velocity relies on the assumption of pure advection ($X = \int{U}dt$). This allows a simple paramerization of salinity mixing for the entire estuary. This method lacks the ability to inform where mixing is occuring, whether it is a continuous process or controlled by a bathymetric feature elsewhere in the estuary. An informed understanding of the Pescadero estuary and measurements elsewhere in the estuary help to shed light on the likely mechanisms of salinity dispersion.

\subsection{Why this two-part structure?}

Calculation of \emph{K$_x$} shows that at higher salinities ($S_T$ to $S_{ocean}$) dispersion is an order of magnitude less than dispersion in lower salinities ($S_{creek}$ to $S_T$). There are two ways to think salinity dependent mixing: as a longitudinal or a vertical process. In most larger estuaries, movement of salt occurs on tidal and subtidal timescales. In small bar-built estuaries, tidally discontinuous ocean forcing (cf chapter 3) and short residence times may all the estuary to reset as fresh on a tidal or daily timescale, thus eliminating subtidal gravitational circulation. This is the case in the Pescadero estuary.

\begin{figure}[h]
\centering
	\includegraphics[]{chapter4/figures/WellMixedEstuary}
	\caption{A well-mixed estuary is characterized by vertical isopycnals.} \label{fig:WellMixedEstuary}
\end{figure}

In a well mixed estuary (Figure~\ref{fig:WellMixedEstuary}), fresh water is found on the upstream side of the estuary, salt water on the ocean end, and isopycnals (which are also isohalines in the case of salinity dependent density) are vertical. In this case, the lowest salinity water is located the farthest upstream. 

\begin{figure}[h]
\centering
	\includegraphics[]{chapter4/figures/SaltWedgeEstuary}
	\caption{Highly stratified estuaries, like salt-wedge type estuaries, have both longitudinal and vertical stratification.} \label{fig:SaltWedgeEstuary}
\end{figure}

In a salt-wedge estuary (Figure~\ref{fig:SaltWedgeEstuary}), isopycnals are slanted. The lowest salinity water is still the farthest upstream in an estuary, and in an estuary that freshens tidally, the lowest salinity water will also travel farther than saltier water. Additionally, density effects mean that in vertically stratified estuaries, even if salt water intrudes far up estuary, fresher water will still be on the surface. In an idealized homogeneous channel estuary, whether water reaches farther upstream or has access to the surface does not matter. But, natural estuaries are not spatially homogeneous. In the Pescadero estuary, the upstream ends of the estuary are in two creeks, and probably more importantly, the estuary sits amid a marsh complex. Given the salt-wedge type salinity structure, fresher water is given access to shallow wetlands which the saltier water cannot access. 

Knowing that the Pescadero estuary is always vertically stratified, we begin to look at vertical differences and processes to explain the differences in dispersion among high and low salinities. 

\subsection{Vertical processes (mixing)}

To understand the vastly different values of \emph{K$_x$} in higher and lower salinities, we look to time series of salinities at different depths in the estuary. Figure~\ref{fig:SDCApr2012} shows data from three CTDs vertically distributed in the water column at the upstream DC station, approximately 200 m upstream from the NM station (location shown on maps in chapters 2 and 3). The first subplot (a) shows the depth of each sensor. The bed sensor was fixed 20 cm above the bed and the surface sensor was floating on a buoy approximately 20 cm below the surface. The middle sensor was hanging from a buoy such that at most times its line was slack, but at high enough water levels the line went taught and the sensor was pulled down. The second subplot (b) gives salinity at these sensors.

Some features of vertical salinity differences are apparent in this plot. The bed sensor (blue lines on figure~\ref{fig:SDCApr2012}) shows that during this period of time salt completely leaves the estuary daily on the large ebb, but does not freshen completely on the small ebb. (Figure~\ref{fig:UandSvsTch4} confirms salt advects out, but as described in chapter 3, salt can become trapped in the deep region where the DC mooring is located). The mid-column sensor (blue lines on figure~\ref{fig:SDCApr2012}) in (b) shows a similar structure to the CTD collocated with the ADV at NM (Figure~\ref{fig:UandSvsTch4}). The surface salinity at DC on the diurnal high high tide has two peaks (Figure ~\ref{fig:SDCApr2012}b, green line). The first peak is attributed to advection of the salt field upstream with the flood tide, and the fast drop should be the movement of saline surface waters back downstream with the coupled effect of the ebbing tide and buoyancy-driven restratification of the flow. A second peak in surface salinity occurs as the high salinity at depth begins to lower. We hypothesize that this peak represents water that had been trapped upstream now moving past the lower estuary sensors. 

At these sensors, the two-phase salinity time series of the middle sensor looks to be induced by vertical mixing in the upper water column between the second pulse of salty water and the mid-column water. High vertical shear is observed on ebb tides (cf Chapter 3), which will induce vertical mixing. 

The structure seen at the NM collocated ADV and CTD is thus evidence of mixing in the upper water column between this second pulse of salty surface water and salinities at upper and middle depth.

\subsection{Variation in two-phase K$_x$ structure with tides}
There is a marked difference in the transition salinity (\emph{S$_T$}) of ebb tides on 21 \& 22 April compared to those on 23 \& 24 April (Table~\ref{tab:Kxtable} and figures~\ref{fig:Kx421}-~\ref{fig:Kx424}). Figure~\ref{fig:SsurfLT} shows surface salinity at DC. Early in the time series, the estuary is in a spring tide (Figure~\ref{f2_QTHs}). Lower values of \emph{S$_T$} on 21 and 22 April correspond with second pulses of salt water with lower salinity as shown by the green line in figure~\ref{fig:DC_salt_421_425}c. Higher values on 23 and 24 April correspond with pulses of salt water of higher salinity. The longer term figure shows that this second pulse of salt water follows the spring tides and shuts down during the neap tide. This variation points toward trapping of water in shallower regions because as the water level gets higher during the spring tide, higher salinity water will have access to trapping within shallower environments.

Differences between the salinity magnitude of the first and second spring tides is attributed to two factors: the first is that the second spring tide had higher water levels than the first spring tide (Figure~\ref{f2_QTHs}), and the second is that lower freshwater discharge was present during the second spring.  

\subsection{Longitudinal processes (trapping, dispersion?)}

Several mechanisms may allow the trapping and release of water at shallow depths that induces the second phase of salinity mixing. Surface salinity trapping probably occurs in the shallower upper reaches of the estuary. Sensors were located primarily in the lower estuary, and no surface sensor measurements were made upstream of DC, so an exact location or mechanism of trapping is not verifiable using our data. At least three features of the Pescadero estuary could contribute to the bimodal surface salinity spike seen in conjunction with the ebbing high high tide:

\subsubsection{Confluence of two creeks}
Approximately 750 m upstream from the mouth of the Pescadero estuary, two creeks form a confluence (Figure~\ref{fig:geMarsh}). Salt water carried by the flood tide may move either up the Pescadero Creek or up the Butano Creek. (In a third process described below, it may also move through culverts into a channel to the shallow North Pond.) Water that flows up perfectly equal branching channels should ebb down these channels so that two parcels of water split by the branch arrive back together. Variation in natural channels may cause divergence of these hypothetical parcels as the balance of momentum and friction in each channel differs. Phasing of tidal flow into these channels may cause interactions which set up a delayed pulse of salt water downstream. 

Bed sensors in each creek were approximately 100 m upstream of the confluence. Data at these sensors (Figure~\ref{fig:PCBC421:425}) confirm that these are not completely uniform channels. Less salt water moves up Pescadero Creek than Butano Creek. This could be due to higher freshwater flow from the Pescadero Creek, which has a larger watershed. The Pescadero sensor may have been approximately 50 m farther upstream from the confluence than the Butano Creek sensor (the mooring was dragged from it's initial location), and so it is possible that this difference played a big role in whether salt reached the mooring. Further differences between the two creek channels are discussed below.

\subsubsection{Marsh}
The Pescadero Creek flows into the lower estuary through a relatively constricted, channelized creek bed.  Meanwhile, the Butano Creek flow through an extensive salt marsh. This marsh floods significantly with a closed mouth (cf Chapter~\ref{chPescadero}), but as is the nature of salt marshes, some flooding occurs tidally. The marsh has various sloughs, and the historic Butano Creek bed has undergone massive accretion with sediments from the logged watershed (Figure~\ref{fig:geMarsh}), so the main channel has shifted, adding further complexity to Butano marsh and creek flow paths.

Bed salinity measurements in the Pescadero and Butano Creeks suggest that within the Pescadero Creek (upstream of the confluence of both the Butano and the culverts to the North Pond) do not demonstrate the same structure seen downstream of these controls. The time series of salinity is very sharp and mostly uniform on flood and ebb (Figures~\ref{fig:PCBC421:425} and~\ref{fig:PCBC511:516}). Bed salinity measurements in the Butano Creek show a slightly more relaxed flood salinity structure, and hint at a two-phase ebb salinity structure. The later time series (Figure~\ref{fig:PCBC511:516}) actually shows two pulses of salty water on the ebb, similar to surface measurements downstream. These measurements seem to implicate marsh processes in setting the two-phase salinity structure.

\subsubsection{Culverts and flow to North Pond}
The North Pond in the Pescadero marsh is connected to the Pescadero creek through a channel and dilapidated culverts (Figures~\ref{fig:geNP} and~\ref{fig:photoCulverts}). The culverts through a levee were designed to be able to regulate flow to the North Pond, but sluice gates and culverts have since rusted and fallen into disrepair and culverts remain perennially open, allowing slow flooding and draining of the shallow North Pond. 

\section{Sensitivity to inputs/assumptions} \label{sec:Sensitivity}

Values of $K_x$ in saltier and fresher salinities remain an order of magnitude apart, but are sensitive to assumptions made in the analysis.

\subsection{Time dependence}

Figure~\ref{fig:tvsS} ~\ref{fig:Kxvst}

Using equation~\ref{eq:S} to get $K_x$ values required providing a value of time $t$. The time used was $t=t_0 + t_{isopycnal}$, where $t_{isopycnal}$ was the time between the flooding and ebbing of each isopycnal past the ADV and CTD at NM. Because different $K_x$ values were calculated from higher and lower salinities, the 20 PSU and 8 PSU isopycnals were selected as representative of the two separate water masses. 

Figure~\ref{fig:tvsS} gives the time between each isopycnal for the four tides. Time is not constant, so selecting a value introduces bias to the $K_x$ calculation.

Time $t_0$, the time from an infinitely sharp interface to the flood \emph{S(x)} structure was estimated to be 1000 s based on physical parameters within the estuary (cf section~\ref{ssec:DataProcessing}). $K_x$ values are also sensitive to this value.

A longer $t$ lessens the dispersion coefficient $K_x$. The influence of selecting different isopycnals on the calculation of $K_x$ is shown in figure~\ref{fig:tvsS}. 

\subsection{Influence of salt trap}
Near the mouth, the water column was seen to freshen on the semidiurnal tide (Figure~\ref{fig:UandSvsTch4}) but in the deep channel upstream, freshening sometimes only occurred on the large ebb (Figure~\ref{fig:DC_salt_421_425}). We assumed $t$ based on the passage of the salt front past the collocated ADV and CTD at NM, a time of less than one tidal cycle. However, some of the high salinity water passing the NM sensors on the large ebb will have been retained for an extra tidal cycle. Dispersion may be lower at high salinity than estimated here if the time high salinity water is retained is double our estimates. This may explain the different values of high salinity $K_x$ on 21 \& 22 April compared to on 23 \& 24 April. 


\section{Conclusions}

Mixing and dispersion of the salt field in the Pescadero estuary has been approached via simplifying the system into the one-dimensional advection-diffusion equation for salinity, creating a spatial field with collocated velocity and salinity measurements, and fitting the dispersion coefficient, $K_x$, to the resulting curves. The resulting values show that mixing is higher at low salinities in the estuary than at high salinities. This appears to be a function of vertical mixing of salt in the upper water column on the ebb, brought on by the \emph{presence} of a second pulse of salinity in the surface, dependent on the spring-neap tidal cycle.  


% -------------- FIGURES -------------- %




\begin{figure}
	\includegraphics[width=\textwidth]{chapter4/figures/DC_salt_421_425}
	\caption{Sensor depth (a) and salinity measurements (b) at DC for the four days where $K_x$ is computed by sensors at NM. The midcolumn sensor (red) has a similar structure to that seen at the CTD collocated with the ADV, and the surface sensor (green) suggests that this structure is an effect of mixing a second pulse of saline water on ebb.} \label{fig:SDCApr2012} \label{fig:DC_salt_421_425}
\end{figure}


\begin{figure}
	\includegraphics[width=\textwidth]{chapter4/figures/DC_surfaceonly_421_513}
	\caption{A longer record of the the surface salinity sensor at DC (green line in figure~\ref{fig:SDCApr2012}. The magnitude of the second pulse of saltier water on the ebb at the surface follows the spring-neap cycle.} \label{fig:SsurfLT}
\end{figure}


\begin{figure}
	\includegraphics[width=\textwidth]{chapter4/figures/PCBC_salt_421_425}
	\caption{Bed CTD sensor measurements in the Pescadero (PC - blue line) and Butano (BC  - red line) creeks for the four dacs of collocated ADV and CTD measurements at NM. (a) gives the sensor depth and (b) gives the salinity. The salt structure appears to be more relaxed in Butano Creek than in Pescadero Creek, both on flood and ebb. The truncated depth at low tide at PC suggests salt water does not reach the Pescadero Creek sensor on the small flood tide because of bathymetric controls. Higher freshwater streamflow may also play a role in limiting salt intrusion in the Pescadero Creek.} \label{fig:PCBC421:425}
\end{figure}



\begin{figure}
	\includegraphics[width=\textwidth]{chapter4/figures/PCBC_salt_516_520}
	\caption{Bed CTD sensor measurements in the Pescadero (PC - blue line) and Butano (BC - red line) creeks 25 days after the measurements in figure~\ref{fig:PCBC421:425}. (a) gives the sensor depth and (b) gives the salinity. Here, the bed sensor in Butano creek measures two pulses of more saline water as well as an altered dS/dt similar to downstream sensors - suggesting the mechanism for bifurcated K$_x$ calculations near the mouth occurs upstream of the Butano sensor. The less truncated behavior of the PC depth record compared to figure~\ref{fig:PCBC421:425} could be from bed movement or movement of the mooring. A faulty CTD at BC only recorded for X weeks at a time, so data are missing between X and X.}  \label{fig:PCBC511:516}
\end{figure}


\begin{figure}
\centering
	\includegraphics[height=.4\textheight]{chapter4/figures/timefrontpassagebysalt}
	\caption{Time \emph{t} used to compute K$_x$ for the two-phase salinity structure. Time at salinity = 20 was used for the higher salinity calculation and time at salinity = 10 was used for the lower salinity calculation.}  \label{fig:tvsS}
\end{figure}



\begin{figure}
\centering
	\includegraphics[height=.6\textheight]{chapter4/figures/Kx_vs_t_both}
	\caption{A range of K$_x$ computed using \emph{t} from different isopycnals (Figure~\ref{fig:tvsS}).}  \label{fig:Kxvst}
\end{figure}



\begin{figure}
\centering
\begin{subfigure}{.48\textwidth}
	\includegraphics[width=\linewidth]{chapter4/figures/P3230467_culverts.JPG}
\end{subfigure}
\begin{subfigure}{.48\textwidth}
	\includegraphics[width=\linewidth]{chapter4/figures/PB290273_culverts.JPG}
\end{subfigure}
\caption{Photographs of culverts to North Pond, left photo 23 March, 2010, right photo 29 November, 2010. There has been further deterioration of the culverts since these photographs were taken.} \label{fig:photoCulverts}
\end{figure}



\begin{figure}
\centering
	\includegraphics[width=.5\textwidth]{chapter4/figures/northpond_20080208_cropped.jpeg}
	\caption{Satellite image of the Pescadero estuary and marsh on 8 February, 2008 showing the channel from the Pescadero creek to the North Pond. Image: Google Earth} \label{fig:geNP}
\end{figure}


\begin{figure}
\centering
	\includegraphics[width=.5\textwidth]{chapter4/figures/marsh_conf_20120519_cropped.jpeg}
	\caption{Satellite image of the Pesacadero estuary and marsh showing the confluence of the Butano and Pescadero creeks, salt marsh, branching structure of the Butano creek, and shallow channels at low tide on 19 May, 2012. Image: Google Earth} \label{fig:geMarsh}
\end{figure}
