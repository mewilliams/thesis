\chapter{Salt chapter}
\label{chSalt}

\emph{to do:
\begin{enumerate}
	\item maybe move google earth images/ photos to chapter 2.
\end{enumerate}
}


\section{Introduction}
\emph{why, how to quantify mixing, what processes are involved in estuaries, and how about estuaries with peripheral stuff?}
Salt transport in small, bar-built estuaries tends to be frontal. Sharp gradients exist between freshwater and salt water, both vertically and horizontally.

In the open state of the Pescadero estuary, salt moves in and out as described in chapter 3. This general description treats salt water and freshwater as relatively immiscible, a character deemed untrue by decades of estuarine physics research and millenia of cooking pasta. Here, then, I set out to characterize salt mixing and dispersion in the Pescadero estuary. 

\emph{HOW TO QUANTIFY? Dispersion processes... turbulent mixing, shear dispersion, ... scale processes? --> why looking at K$_x$ instead of $K_z$}

\emph{IMPORTANT ESTUARINE PROCESSES, AND OTHER WORK SIMILAR. Background on what I'm trying to do.  Cite Ralston \& Stacey 2005- Longitudinal dispersion... tidal straining (Simpson et al, 1990, Burchard \& Baumert 1998?),.. longitudinal dispersion (Okubu, 1973; Lissa's thesis paper?)}, here I will cite \parencite{Ralston:2005aa} and maybe also John Simpson's paper \parencite{Simpson:1990aa}. 


\subsubsection{What papers are relevant here? Quickly describe them:}
\parencite{okubo_effect_1973}: Shoreline irregualarities trap tracers, and this acts as a mechanism for dispersion of said tracers.  In a marsh this might function as water being trapped in the vegetation or sloughs and small channels. Because water here is slower moving, tracer will disperse back out... causing \emph{dispersion}. \emph{This analysis is based on multiple tidal cycles/subtidal flow, and thus the applicabiltiy of Okobu's analysis may be limited systems where subtidal flow doesn't really mean anything \parencite{Ralston:2005aa}}. 

\parencite{Simpson:1990aa}: Tidal straining... 

\parencite{Ralston:2005aa}: \emph{Similar thing, different estuary}
Ralston and Stacey \cite{Ralston:2005aa} used a 1-d framework to quantifiy frontal mixing within the intertidal zone of mudflats in San Francisco Bay. They found \emph{K$_x$} of 10 m\textsuperscript{2} s\textsuperscript{-1} of the frontal salinity structure, and attributed this mixing to ...

\emph{Giddings et al. 2012, maybel } 


\emph{salt is conservative, implications for temperature, plankton, other stuff that advects and diffuses around...}

\emph{want to quantify mixing!}
\emph{range of K$_x$ for other estuaries? salt-wedge in particular. }

\section{Methods} \label{sec:ch4methods}

%% 1D Advection-diffusion
\subsection{1D advection-diffusion} \label{ssec:1dadvdif}

We start with the depth-averaged 1-d advection-diffusion of salinity,
\begin{eqnarray}
\frac{\partial S}{\partial t} + U\frac{\partial S}{\partial x} = K_x\frac{\partial^2S}{\partial x^2} \label{eq:1dadvdiff}
\end{eqnarray}
where \emph{S} is salinity, \emph{U} is depth-averaged velocity, and \emph{K$_x$} is the longitudinal dispersion coefficient \emph{(or mixing term..)}. In an estuary, the salt structure is advected upstream and downstream with the tides. Moving the frame of reference along with these tides, \emph{U=0} relative to the reference frame and equation~\ref{eq:1dadvdiff} becomes
\begin{eqnarray}
\frac{\partial S}{\partial t} = K_x\frac{\partial^2S}{\partial x^2} \label{eq:1ddiff}
\end{eqnarray}
Assuming salinitiy follows a step funtion at time $t=0$ and that the boundaries of the system remain at \emph{ocean salinity, $S=S_0$} and freshwater salinity$S=0$, an analytical solution to equation~\ref{eq:1ddiff} can be obtained,
\begin{eqnarray}
S(x,t) = \frac{S_0}{2}\left(1+erf\left(\frac{x-x_c}{\sqrt{4K_xt}}\right)\right) \label{eq:S}
\end{eqnarray}
where \emph{x$_c$} is the center of the salt front. Using observations of the salinity field, the value of \emph{K$_x$} can be obtained from this analytical solution.


%%% Observations --> Analytical framework
\subsection{Using observations in an analytical framework} \label{ssec:ObsInto1DAdvDiff}
To solve for \emph{K$_x$} in equation~\ref{eq:S}, the salinity structure in time and space within an estuary must be known. Salinity measurements were made in the Pescadero estuary with good temporal resolution and some longitudinal distribution during an April - June 2012 field campaign. However, longitudinal salinity gradients in the Pescadero estuary as well as in other small bar-built estuaries are very sharp, so locating instruments close enough to each other to directly measure the longitudinal gradient and getting spatial coverage to understand dyanamics of a whole system would be nearly impossible. So, we resort to recreating the salintiy structure in time using measurements from a collocated ADV and CTD. 

The sensors give a time series of salinity and velocity (0.X m above the bed at NM). Velocity is the rate of change of distance with time,

\begin{eqnarray}
U = \frac{dx}{dt} \label{eq:uEdxdt}\\
\end{eqnarray}
so integrating velocity,
\begin{eqnarray}
\int{Udt} = \int{dx} = X \label{eq:intuEx}
\end{eqnarray}
transforms the time series of point measurements of velocity and salinity (Figure~\ref{fig:UandSvsTch4}) into a spatial representation of salinity ($S(x)$, Figure~\ref{fig:SvsXall}).

%%%% Data Processing Specifics
\subsection{Data processing} \label{ssec:DataProcessing}
To make this dataset, velocity measurements were burst-averaged (\emph{8.3 min. burst?... specifics in previous chapter}) and an 800 s moving average was applied to the CTD measurements.  Averaging was intended to remove the infragravity oscillation component of velocity and salinity measurements to focus on tidal processes. 

To get \emph{K$_x$} from the recreated \emph{S(x)}, equation~\ref{eq:S} is rearranged into the linear form:
\begin{eqnarray}
erf^{-1}\left(\frac{2S}{S_0}-1\right) = \frac{x-x_c}{\sqrt{4K_xt}} \label{eq:linfitS}
\end{eqnarray}
and a linear fit ((y = ax + b).. a = $\frac{1}{\sqrt{4K_xt}}$ and b = $\frac{-x_c}{\sqrt{4K_xt}}$) is applied. Given a linear regression fit value for \emph{a}, to solve for $K_x$, a value of \emph{t} must be determined. In this case it is appropriate to use the time between when the front floods past the sensors and subsequent ebb of said front. This method assumes an infinitely sharp front at time $t_0$. Setting $t_0$ as the time at which the front floods past the sensors will overestimate \emph{K$_x$}. Thus, an estimation on the initial time from a sharp front to the flood front is made. By setting a step function at the mouth, and approximating advection velocities U = 0.1 m s$^{-1}$ and a length from the ocean to NM of 100 m, initial time, t$_0$ is 1000 s. This time is then added to the time between a selected isopycnal passing by the sensor. Specific considerations in selecting the isopycnal to follow, and influence of $t_0$ will be discussed in section~\ref{sec:Sensitivity}.



\section{Observations} \label{sec:ObsCh4}
Time series of collocated velocity and salinity measurements at NM are shown in figure~\ref{fig:UandSvsTch4}. Gaps in measured data occur when the water level is very low and sensors were out of the water. In response to coastal California's mixed semidiurnal tide, the time series shows that salt water does not remain in the system on the low high tide  as long as it does on the high high tide suggesting limited tidal excursion of the salt field. Here we focus only on the tidal cycle associated with high high water level, referred to in this test as the \textquotedblleft large tide.''

\begin{figure}[h]
	\includegraphics[width=\textwidth]{chapter4/figures/UandSvsT} 
\caption{Velocity (a) and salinity (b) records X cm above the bed at station NM. Velocity measurements have been averaged across their X s burst and rotated into the direction of principle flow. An 800 s moving average has been applied to salinity measurements. Averaging was done to remove higher frequency oscillations.} \label{fig:UandSvsTch4}
\end{figure}

The time series plots of salinity and velocity (Figure~\ref{fig:UandSvsTch4}b) show that the transition from fresh to ocean salinity water occurs rapidly on the flood and the return to fresh water on ebb is a longer duration process. Also, the ebb salinity field passes the sensors in two distinct segments. On ebb, lower salinities demonstrate a much more relaxed salinity structure than higher salinities. The flood salinities do not exhibit this feature. To investigate whether this observation is an actual transition in salinity and not just a reduction in water velocity, as well as to approach a quantification of \emph{K$_x$}, the time series is transformed into a spatial record as described in section~\ref{sec:ch4methods}. 

\begin{figure}[hp]
	\includegraphics[width=\textwidth]{chapter4/figures/SvsXfloodebb} 
\caption{S(x) for four floods - April 2012, 20cm above bed, Near Mouth (a) and four ebbs (b). In both figures, the profiles are centered (x=0) around the S = 15 PSU pycnocline.} \label{fig:SvsXall}
\end{figure}

Figure~\ref{fig:SvsXall} shows the salt front on flood (a) and on ebb (b) for the four large tides on 21 - 25 April, 2012. Spatial mapping of the salinity field shows that the estuary transforms  the salinity front from a smooth, sharp structure on flood to a two-part relaxed structure on ebb, where $S(x)$ is much more relaxed among lower salinities than among higher salinities. This mirrors the time series salinity and implies that there is a physical change in the salt field between flood and ebb. This non uniform structure requires the calculation of two different values of \emph{K$_x$} for the higher and lower salinities.

By fitting the error function curve to the two ebb salinity fields determined by the salinity of the transition point ($S_T$), values of \emph{K$_x$} are thus calculated for the higher ($K_{x,S_T\rightarrow S_0}$) and lower salinities ($K_{x,0\rightarrow S_T}$) in the four tidal cycles (Table~\ref{tab:Kxtable} and Figures~\ref{fig:Kx421}-~\ref{fig:Kx424}). Averaged across the four days, \emph{K$_x$} = 4.1 m$^2$s$^{-1}$ at higher salinity and \emph{K$_x$} = 45.4 m$^2$s$^{-1}$ at lower salinity. Values of \emph{t} for \emph{K$_x$} calculations are taken to be the time between when the 8 PSU and 20 PSU isopycnals flood and ebb past the sensor for lower and higher salinity \emph{K$_x$} values, respectively. 

Values of \emph{K$_x$} show that dispersion is much greater at lower salinities than at higher salinities in the Pescadero estuary.  Values of \emph{K$_x$} are an order of magnitude higher in lower salinities than in higher salinities. Dispersion coefficients in lower salinity water (\emph{K$_{x,S=0 \rightarrow S_T}$}), are higher in the first two days than in the last two. Dispersion coefficients in higher salinity water (\emph{K$_{x,S=S_T \rightarrow S_0}$}) are lower in the first two days than in the second two days. 

% --------------TABLES ---------------------------------------------------------------%
\begin{table}[h]
\renewcommand{\arraystretch}{1.3}

	\begin{center}
		\begin{tabular}{| l || c | c | c | c | c | c |}
		\hline
		Date & S$_{T}$ & K$_{x,S=0 \rightarrow S_T}$ &  K$_{x,S=S_T \rightarrow S_0}$ & x$_{c,fresher}$ & x$_{c,fresher}$\\
		\hline \hline
		21 Apr & 11.5 & 48.7 m$^2$ s$^{-1}$ & 2.3 m$^2$ s$^{-1}$ & -591 m & -9.4 m\\ 
		22 Apr & 10.5 & 65.1 m$^2$ s$^{-1}$ & 3.1 m$^2$ s$^{-1}$ & -713 m & -60.2 m\\ 
		23 Apr & 14.4 & 31.0 m$^2$ s$^{-1}$ & 5.2 m$^2$ s$^{-1}$ & 34.2 m & 18.95 m\\ 
		24 Apr & 14.1 & 31.6 m$^2$ s$^{-1}$ & 5.9 m$^2$ s$^{-1}$ & -137.3 m & -3.9 m\\  \hline 
	\end{tabular}
	\caption{Values of \emph{K$_x$} for the large flood and ebb on four days  in April 2012.}\label{tab:Kxtable}
\end{center}
\end{table}
% -------------------------------------------------------------------------------------%


Comparison to theory, hypothesized mechanisms of dispersion, reasons for the two-phase structure of \emph{S(x)}, and changes with physical forcing are discussed below. 
% Kx figures
\begin{figure}[hp]
\centering
	\includegraphics[width=.75\textwidth]{chapter4/figures/SvXwKx_124_165} 
\caption{Spatially reconstructed salinity field and \emph{K$_x$} values for the large tidal cycle on 21 April, 2012. } \label{fig:Kx421}
\end{figure}

\begin{figure}[hp]
\centering
	\includegraphics[width=.75\textwidth]{chapter4/figures/SvXwKx_183_263} 
\caption{Spatially reconstructed salinity field and \emph{K$_x$} values for the large tidal cycle on 22 April, 2012. } \label{fig:Kx422}
\end{figure}

\begin{figure}[hp]
\centering
	\includegraphics[width=.75\textwidth]{chapter4/figures/SvXwKx_279_357} 
\caption{Spatially reconstructed salinity field and \emph{K$_x$} values for the large tidal cycle on 23 April, 2012. } \label{fig:Kx423}
\end{figure}

\begin{figure}[hp]
\centering
	\includegraphics[width=.75\textwidth]{chapter4/figures/SvXwKx_376_454} 
\caption{Spatially reconstructed salinity field and \emph{K$_x$} values for the large tidal cycle on 24 April, 2012.} \label{fig:Kx424}
\end{figure}
% end Kx figures




\section{Comparison to theoretical ways to estimate $K_x$}

Longitudinal dispersion caused by vertical shear \emph{(in an unstratified water column?)} was derived by Elder to be
\begin{eqnarray}
K_x = 5.93hu_* \label{eq:Kshear}
\end{eqnarray}
where \emph{h} is the height of the water column and \emph{u$_*$} is the shear velocity defined by $u_*=\sqrt{\tau_0/\rho}$ where $\tau_0$ is a bed stress and $\rho$ is water density \parencite*{elder_dispersion_1959}. The friction velocity is parameterized as $u_* = c_{_D}^{1/2}U$. In the Pescadero estuary, \emph{h} ranges from 0.2 m to 2.7 m depending on location in the estuary and tidal conditions, \emph{U} is $\pm$ 0.4 m s\textsuperscript{-1} with spring tides, and surges due to infragravity motions may result in positive velocities of 0.6 m s\textsuperscript{-1}. Estimating c$_D$ as 0.01, the largest \emph{K$_x$} by this formulation would be \emph{K}$_x =$ 0.96 m$^2$ s$^{-1}$ using the maximum value of \emph{h} and the infragravity velocity of $U = 0.6$ m s$^{-1}$. More likely values of $h = 1 m$ and $U = 0.3 m s^{-1}$ result an estimated \emph{K}$_x = 0.18$ m$^2$ s$^{-1}$. In either case, this value of \emph{K$_x$} is well below observations.

\emph{longitudinal dispersion via tidal trapping...}
\begin{eqnarray}
K = \frac{K^{'}}{1+r} + \frac{ru_0^2}{2k(1+r)^2(1+r+\omega / k)} \label{eq:KxOkubo}
\end{eqnarray}

\emph{quoted from Fischer:} "where K$^{'}$ is the longitudinal diffusivity in the main channel itself. uniform velocity of flow in the main channel of velocity $u = u_0 \cos(\omega t)$ and a uniform distribution of traps along the sides having a ratio of trap volume to to channel volume of r and a characteristic exhange time between traps and main flow of k$^{-1}$, effective long. diffusivity is.." equation above. \emph{discussion here...}

\emph{If u0 is .3, r = .5, k and omega are what are in fischer... the second term is 34.5 (first is .66K)... If I figure out what that longitudinal diff term is, can get r variation with tide? Except I expect more trappign with spring tide, and my values from the 4 days suggest opposite.}

\section{Discussion} \label{sec:discCh4}

Spatial mapping of the salinity field from time series of velocity relies on the assumption of pure advection ($X = \int{U}dt$) This allow a paramerization \emph{OF WHY THIS IS AN AWESOME METHOD}. This method lacks the ability to inform where mixing is occuring, whether it is a continuous process or controlled by a bathymetric feature elsewhere in the estuary. An informed understanding of the Pescadero estuary and measurements elsewhere in the estuary help to shed light on the likely mechanisms of salinity dispersion.

\subsection{Why this two-part structure?}

Calculation of \emph{K$_x$} shows that at higher salinities ($S_T$ to $S_{ocean}$) dispersion is an order of magnitude less than dispersion in lower salinities ($S_{creek}$ to $S_T$). Having one-dimensionalized the estuary, there are two ways to think salinity dependent mixing: as a longitudinal or a vertical process. 

\begin{figure}[h]
\centering
	\includegraphics[]{chapter4/figures/WellMixedEstuary}
	\caption{.} \label{fig:WellMixedEstuary}
\end{figure}

In a well mixed estuary (Figure~\ref{fig:WellMixedEstuary}), fresh water is found on the upstream side of the estuary, salt water on the ocean end, and isopycnals (which are also isohalines in the case of salinity dependent density) are vertical. In this case, the lowest salinity water travels the farthest upstream.

\begin{figure}[h]
\centering
	\includegraphics[]{chapter4/figures/SaltWedgeEstuary}
	\caption{.} \label{fig:SaltWedgeEstuary}
\end{figure}

In a salt-wedge estuary (Figure~\ref{fig:SaltWedgeEstuary}, isopycnals are slanted. The lowest salinity water still travels the farthest upstream in an estuary. Additionally, density effects mean that in vertically stratified estuaries, even if salt water is able to intrude far up estuary, fresher water will still be on the surface. In an idealized homogenous channel estuary, whether water reaches farther upstream or has access to the surface should not matter. But, most estuaries are not spatially homogeneous. In the Pescadero estuary, the upstream ends of the estuary are in two creeks, and probably more importantly, the estuary sits amid a marsh complex - which given the salt-wedge type salinity structure would give fresher water access to shallow wetlands which the saltier water cannot access. 

\emph{There might need to be a comment here distinguishing between estuaries that operate on sub-tidal timescales and small ones that reset daily...}

Knowing that the Pescadero estuary is always vertically stratified, we begin to look at vertical differences and processes to exaplain the differences in dispersion among high and low salinities. 

\subsection{whatever i jsut wrote felt like it needed a break}

To understand the vastly different values of \emph{K$_x$} in higher and lower salities, we look to time series of salinities at different depths in the estuary. Figure~\ref{fig:SDCApr2012} shows data from three CTDs vertically distributed in the water column at the upstream DC station, approximately 200 m upstream from the NM station. The first subplot (a) shows the depth of each sensor. The bed sensor was fixed 20 cm above the bed and the surface sensor was floating on a buoy approximatly 20 cm below the surface. The middle sensor was hanging from a buoy such that at most times its line was slack, but at high enough water levels the line went taught and the sensor was pulled further underwater. The second subplot (b) gives salinity at these sensors.

Some features of vertical salinity differences are apparent in this plot. The bed sensor (blue lines on figure~\ref{fig:SDCApr2012}) shows that during this period of time salt completely leaves the estuary daily on the large ebb, but does not freshen completely on the small ebb. (Figure~\ref{fig:UandSvsTch4} confirms salt advects out, but as shown in chapter 3, salt can become trapped in the deep region where the DC mooring is located). The mid-column sensor (blue lines on figure~\ref{fig:SDCApr2012}) in (b) shows a similar structure to the CTD colocated with the ADV at NM (Figure~\ref{fig:UandSvsTch4}). The surface salinity at DC on the diurnal high high tide has two peaks (Figure ~\ref{fig:SDCApr2012}b, green line). The first peak is attributed to advection of the salt field upstream with the flood tide, and the fast drop should be the movement of saline surface waters back downstream with the coupled effect of the ebbing tide and buoyancy-driven restratification of the flow. A second peak in surface salinity occurs as the high salinity at depth begins to lower. We hypothesize that this peak represents water that had been trapped upstream now moving past the lower estuary sensors. 

\emph{this sentence says we see vertical mixing happening in the surface because of that extra peak thing... }

The structure seen in the mid-column at DC or at the NM collocated ADV and CTD is thus evidence of mixing in the upper water colum between this second pulse of salty surface waters and salinities at upper and middle depth. 

\subsection{Where does it come from?}

Several mechanisms may allow the trapping and release of water that induces a second phase of salinity mixing. Intuition suggests that surface salinity trapping probably occurs in the shallower upper reaches of the estuary. Sensors were located primarily in the lower estuary, and no surface sensor measurements were made upstream of DC, so an exact location or mechanism of trapping is not verifiable using our data. At least three features of the Pescadero estuary could contribute to the bimodal surface salinity spike seen in conjunction with the ebbing high high tide:

\subsubsection{Confluence of two creeks}
Approximately \emph{1000 m} upstream from the mouth of the Pescadero estuary, two creeks form a confluence (Figure~\ref{fig:geMarsh}). Salt water carried by the flood tide may move either up the Pescadero Creek or up the Butano Creek. (In a third process described below, it may also move through culverts into a channel to the shallow North Pond.) Water that flows up perfectly equal branching channels should ebb down these channels so that two parcels of water split by the branch arrive back together. Variation in natural channels may cause divergence of these hypothetical parcels, most noticibly via effects of momentum differences... 

 \emph{description of tidal trapping here}.. Phasing of tidal flow into these channels may cause interactions which set up a delayed pulse of salt water downstream. 

Bed sensors in each creek approximately the same distance from the confluence \emph{(check this)} (Figure~\ref{fig:PCBC421:425}) confirm that these are not completely uniform channels. Less salt water moves up Pescadero Creek than Butano Creek. This could be due to higher freshwater flow from the Pescadero Creek, which may allow for \emph{... hydraulic control at the channel that would push Pescadero water into Butano, or... (think about this some more)} 

\subsubsection{Marsh}
The Pescadero Creek flows into the lower estuary through a relatively constricted, channelized creek bed.  Meanwhile, the Butano Creek flow through an extensive salt marsh. This marsh floods significantly with a closed mouth (cf Chapter~\ref{ch2}), but as is the nature of salt marshes, some flooding occurs tidally. \emph{words on dispersion and vegetation. or maybe this also counts as tidal trapping... because of sloughs and shallow old butano creekbed}. 

Bed salinity measurements in the Pescadero and Butano Creeks suggest that within the Pescadero Creek (upstream of the confluence of both the Butano and the culverts to the North Pond) do not demonstrate the same structure seen downstream of these controls. The time series of salinity is very sharp and mostly uniform on flood and ebb (Figures~\ref{fig:PCBC421:425} and~\ref{fig:PCBC511:516}). Bed salinity measurements in the Butano Creek show a slightly more relaxed flood salinity structure, and show signs of a two-phase ebb salinity structure. The later time series (Figure~\ref{fig:PCBC511:516}) actually shows two pulses of saline water on the ebb, similar to surface measurements downstream.  The instruments were fixed closer to the bed than the colocated ADV and CTD, so these measurements may not directly translate... but do suggest something is going on in the salt marshes.  

Within the Butano Creek marshes, various sloughs exist, and the historic creek bed has undergone massive sedimentation so that the main channel is now \emph{different}. Satellite photos show this shallow bed (Figure~\ref{fig:geMarsh}), and this could certainly be a trap of slow moving water... \emph{high friction there because shallow...}. 

\subsubsection{Culverts and flow to North Pond}
The North Pond in the Pescadero marsh is connected to the Pescadero creek through a channel and dilapidated culverts (Figures~\ref{fig:geNP} and~\ref{fig:photoCulverts}). \emph{The culverts through a levee were designed to be able to regulate flow to the North Pond via gates, but the gates have rusted and culverts remain perennially open.} The highly constricted flow... 

In any of these mechanisms, lower salinity water is subject to \emph{these effects} because density effects mean higher salinity water does not access the shallow marsh or culverted environments... 

\subsection{Variation in two-phase K$_x$ structure}
Figure~\ref{fig:SsurfLT} shows surface salinity at DC. Early in the time series, the \emph{estuary is in a} spring tide, neap tides are from \emph{these dates} and then the spring tide...\emph{eh, probably should put a depth record on this one}. Lower values of \emph{S$_T$} on 21 and 22 April correspond with second pulses of salt water with lower salinity as shown by the green line in figure~\ref{fig:DC_salt_421_425}c. Higher values on 23 and 24 April correspond with pulses of salt water of higher salinity. The longer term figure shows that this second pulse of salt water follows the spring tides and shuts down during the neap tide. \emph{previous stuff makes sense} because as the water level gets higher during the spring, higher salinity water will have access to shallower \emph{regions}, trapping higher salinity water... 

Differences between the salinity magnitude of the first and second spring tides is attributed to two factors: the first is that the second spring tide had higher water levels than the first spring tide (\emph{figure from ch 3?}), and the second is that lower freshwater discharge was present during the second spring. 

\section{Assumptions \& sensitivity to these assumptions (limitations)} \label{sec:Sensitivity}
\subsection{time/salinity dependence of the method}

Figure~\ref{fig:tvsS} ~\ref{fig:Kxvst}

\emph{I used the value of t at 20 PSU for higher salinities and 10 PSU for lower salinities.  Figure ~\ref{fig:tvsS} shows that the time... isn't constant across these salinities.. will introduce error}.  Next figure ~\ref{fig:Kxvst} gives some approximate ranges of K$_x$ using different isopycnals as representative of the two salinity regimes \emph{(higher S: 18 - 25, lower S: 5 - 10)} and thus different times... 


\subsubsection{What about those high salinity waters?}
Near the mouth, the water column was seen to freshen on the semidiurnal tide (Figure~\ref{fig:UandSvsTch4}) but in the deep channel X m upstream from the mouth, freshening only occurred on the large ebb (Figure~\ref{fig:DC_salt_421_425}). Using these methods, we assumed time based on the frontal passage of the colocated ADV and CTD, a time of less than one tidal cycle. However, it is not possible to distinguish high salinity water that has been trapped at depth in the deep channel for ... \emph{wait. I think the higher value of Kx in the second two tides is because that is trapped water and those t values are off. The first tide the DC pool was freshened so we're actually seeing one tidal excursion.}

\subsection{goodness of fit?}
\emph{An R$^2$ value doesn't really mean anything because It's calculated on an inverse erf vs. another pile of variables. By inspection, the higher salinity K$_x$ on figures~\ref{fig:Kx423} and~\ref{fig:Kx424} don't seem to match up as well... probably explains the higher value of K$_x$ there more than anything previously stated. Not sure what to do here.}

\section{Conclusions of this chapter}
\emph{some conclusions here}
Have simplified complex system to look at salt dispersion... suggests two-layer vertical processes... 

Mixing and dispersion of the salt field in the Pescadero estuary has been approached via simplifying the system into the one-dimensional advection-diffusion equation for salinity, creating a spatial field with colocated velocity and salinity measurements, and fitting the dispersion coefficient, $K_x$, to the resulting curves. The resulting values show that mixing is higher at low salinities in the estuary than at high salinities. This appears to be a function of vertical mixing of salt in the upper water column on the ebb, brought on by the \emph{presence} of a second pulse of salinity in the surface, dependent on the spring-neap tidal cycle.  


% -------------- FIGURES -------------- %




\begin{figure}
	\includegraphics[width=\textwidth]{chapter4/figures/DC_salt_421_425}
	\caption{Sensor depth (a) and salinity measurements (b) at DC for the four days where $K_x$ is computed by sensors at NM. The midcolumn sensor (red) has a similar structure to that seen at the CTD collocated with the ADV, and the surface sensor (green) suggests that this structure is an effect of mixing a second pulse of saline water on ebb.} \label{fig:SDCApr2012} \label{fig:DC_salt_421_425}
\end{figure}


\begin{figure}
	\includegraphics[width=\textwidth]{chapter4/figures/DC_surfaceonly_421_513}
	\caption{A longer record of the the surface salinity sensor at DC (green line in figure~\ref{fig:SDCApr2012}. The magnitude of the second pulse of saltier water on the ebb at the surface follows the spring-neap cycle.} \label{fig:SsurfLT}
\end{figure}


\begin{figure}
	\includegraphics[width=\textwidth]{chapter4/figures/PCBC_salt_421_425}
	\caption{Bed CTD sensor measurements in the Pescadero (PC - blue line) and Butano (BC  - red line) creeks for the four dacs of collocated ADV and CTD measurements at NM. (a) gives the sensor depth and (b) gives the salinity. The salt structure appears to be more relaxed in Butano Creek than in Pescadero Creek, both on flood and ebb. The truncated depth at low tide at PC suggests salt water does not reach the Pescadero Creek sensor on the small flood tide because of bathymetric controls. Higher freshwater streamflow may also play a role in limiting salt intrusion in the Pescadero Creek.} \label{fig:PCBC421:425}
\end{figure}



\begin{figure}
	\includegraphics[width=\textwidth]{chapter4/figures/PCBC_salt_516_520}
	\caption{Bed CTD sensor measurements in the Pescadero (PC - blue line) and Butano (BC - red line) creeks 25 days after the measurements in figure~\ref{fig:PCBC421:425}. (a) gives the sensor depth and (b) gives the salinity. Here, the bed sensor in Butano creek measures two pulses of more saline water as well as an altered dS/dt similar to downstream sensors - suggesting the mechanism for bifurcated K$_x$ calculations near the mouth occurs upstream of the Butano sensor. The less truncated behavior of the PC depth record compared to figure~\ref{fig:PCBC421:425} could be from bed movement or movement of the mooring. A faulty CTD at BC only recorded for X weeks at a time, so data are missing between X and X.}  \label{fig:PCBC511:516}
\end{figure}


\begin{figure}
	\includegraphics[width=\textwidth]{chapter4/figures/timefrontpassagebysalt}
	\caption{Time \emph{t} used to compute K$_x$ for the two-phase salinity structure. Time at salinity = 20 was used for the higher salinity calculation and time at salinity = 10 was used for the lower salinity calculation.}  \label{fig:tvsS}
\end{figure}



\begin{figure}
	\includegraphics[height=\textheight]{chapter4/figures/Kx_vs_t_both}
	\caption{A range of K$_x$ computed using \emph{t} from different isopycnals (Figure~\ref{fig:tvsS}).}  \label{fig:Kxvst}
\end{figure}



\begin{figure}
\centering
	\includegraphics[width=.75\textwidth]{chapter4/figures/P3230467_culverts.JPG}

	\includegraphics[width=.75\textwidth]{chapter4/figures/PB290273_culverts.JPG}

\caption{Photographs of culverts to North Pond, upper photo 23 March, 2010, lower photo 29 November, 2010. There has been further deterioration of the culvers since these photographs were taken.} \label{fig:photoCulverts}
\end{figure}





\begin{figure}
\centering
	\includegraphics[width=.5\textwidth]{chapter4/figures/northpond_20080208_cropped.jpeg}
	\caption{Satellite image of the Pescadero estuary and marsh on 8 February, 2008 showing the channel from the Pescadero creek to the North Pond. Image: Google Earth} \label{fig:geNP}
\end{figure}


\begin{figure}
\centering
	\includegraphics[width=.5\textwidth]{chapter4/figures/marsh_conf_20120519_cropped.jpeg}
	\caption{Satellite image of the Pesacadero estuary and marsh showing the confluence of the Butano and Pescadero creeks, salt marsh, branching struture of the Butano creek, and shallow channels at low tide on 19 May, 2012. Image: Google Earth} \label{fig:geMarsh}
\end{figure}
