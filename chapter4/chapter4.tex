\chapter{Salt chapter}
\label{chSalt}


\section{Introduction}
Salt moves in and out of the Pescadero estuary as described in chapter~\ref{ch3}. The water column remains stratified, and a mixed water column only exists when high freshwater flows have forced all salt from the confines of the estuary. But, with tidal velocities and infragravity oscillations acting on the stratified salt field, there must be some mixing and it behooves us to try to quantify mixing and its effects. \emph{why?}

\section{Methods}
Focusing on velocity and salinity measurements made during the April - June 2012 field deployment (described in detail in the previous chapter).

\begin{itemize} \item{velocity and salt measurements averaged on X window} \item{then fake x made} \end{itemize}

\begin{eqnarray}
u = \frac{dx}{dt} \label{eq:uEdxdt}\\
\int{<u>dt} = \int{dx} = x \label{eq:intuEx}
\end{eqnarray}
So, by taking the integral of the ADV velocity, equation~\ref{eq:intuEx} gives a way to transform a point velocity measurement into an approximation of \emph{X or Length or...}. 


\section{Here i'm going to write the advection-diffusion equations}
\begin{eqnarray}
\frac{dS}{dt} + u\frac{dS}{dx} = K_x\frac{d^2S}{dx^2} \label{eq:1dadvdiff}
\end{eqnarray}


\section{Observations}
\emph{Insert figure: U and S from colocated ADV, CTD vs. time}



\section{Quantifying $K_x$}
