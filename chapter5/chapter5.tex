\chapter{Bar-built estuarine response to tsunami forcing}
\label{chTsunami}




\section{Introduction}

The 9.0 Mw Tohoku earthquake hit Japan at 05:46 GMT on March 11, 2011,
triggering a devastating tsunami that would propogate across the Pacific
Ocean, reaching as far as Chile \citep{mori_ea,lagos_haro_agu}. On
the California coast, the first waves arrived at low tide, but despite
the timing and small amplitude of the tsunami in the Eastern Pacific,
extnsive damage occurred to harbors in Santa Cruz and Crescent City
both with the initial arrival of waves and later when the high high
tide and tsunami combined to create strong currents \citep{Wilsonetal2012}.

Coastal California is dotted with bar-built estuaries where rivers
and streams arrive to the ocean through small estuaries set behind
sandy beaches (Figure \ref{fig:coastline_map_with_inlets}a). Combined
nearshore sediment transport and low seasonal streamflow choke the
mouths of many of these estuaries closed with sand, limiting or completely
cutting off the tidal exchange. Even in the open states of these estuaries,
the mouth is shallow and narrow, attenuating tidal amplitudes and
currents, and maybe cutting off ocean connection during the low low
tide. \citep{williams_stacey_inprep}. This limited forcing allows
these shallow estuaries to remain strongly salt-stratified. Tidally,
salt water advects upstream as a salt-wedge on the flood and may become
trapped in deeper pools of the estuary on the slow ebb {[}e.g. \citealp{Largier_taljaard1991,ranasinghe_pattiaratchi99}{]},
transitioning these small estuaries into vertically salt-stratified
lakes. Except for when low low tide in the ocean lowers the surf and
swash zones below the level of the perched mouth, infragravity motions
in the nearshore induce fast water velocities in these estuaries\citep{williams_stacey_inprep}. 


%--------------- figure 1 - maps--------------------------------

\begin{figure}
\includegraphics{chapter5/figures/fig1_v2.pdf}\protect\caption{Northern California (a) and Pescadero (b) coastlines. The California
coastline map (a) shows locations of other inlets and bar-built estuaries
plotted as black diamonds, the Pescadero estuary plotted within the
red circle, the Monterey and San Francisco NOAA tide gauges as blue
stars and the NDBC Buoy 46012 as a blue square. Bathymetry on the
Pescadero map (b) is given in meters. Bathymetry data were collected
during an extended mouth closure, so the highly variable depth at
the sandy mouth was not recorded. The downstream (DS) mooring with
an ADCP and CTDs is at the circle and the upstream (US) CTD mooring
is at the square. \label{fig:coastline_map_with_inlets}}
\end{figure}
%--------------- end figure 1 - maps ----------------------------

Our analysis of the Pescadero estuary in Northern California (Figure
\ref{fig:coastline_map_with_inlets}b) has established that nearshore
processes with timescales of 30 seconds to three minutes modulated
by the semidiurnal tide dominate the dynamics of these systems \citep{williams_stacey_inprep}.
Here, we examine a perturbation of the system by the arrival of the
Tohoku tsunami, including the impacts on estuarine salinity and mixing. 


\section{Field Setup and Conditions}

Two moorings were deployed in the Pescadero estuary during the March
2011 tsunami (Figure \ref{fig:coastline_map_with_inlets}b). Installed
on March 10, 2011, downstream station DS consisted of one upward-looking
ADCP (RDI 1200kHz Workhorse Monitor) burst sampling at 2 Hz for ten
minutes of every fifteen. Co-located with the ADCP were four vertically
distributed CTDs (RBR XR-420) sampling once per minute. A surface
CTD at this location was not replaced until March 15, 2011 so no surface
salinity record exists before that date. Upstream station US was in
place from January 27, 2011 and contained one bed and one mid-column
subsurface CTD sampling at the same rate as other CTDs. The ADCP deployed
was always submerged, but did not successfully compute water velocities
when the estuary depth was very shallow or when velocities were too
high\emph{ }or too low. 

The Northern California coast experiences mixed semi-diurnal tides.
During the days immediately prior to and during the tsunami, tides
in the nearshore had a neap-tide range of just under\emph{ }1.5 m
(Figure \ref{fig:Pescadero-stacked-variance_presreving_spectra}b).
Large waves picked up on March 11, peaking at 4 m significant wave
height offshore (NOAA NDBC 46012), with significant wave heights between
2 and 5 m for the subsequent days\emph{. }Freshwater streamflow into
the estuary was estimated to be between 1.3 and 2.1 m\textsuperscript{3}
s\textsuperscript{-1}, based on measurements by the USGS Pescadero
gauge (11162500) located five miles upstream of the estuary.\emph{
}The creek discharge was sufficient to freshen the estuary tidally,
due to its small size. 

Nearshore tsunami conditions are inferred from NOAA tsunami-capable
tide stations in Monterey (9413450) and San Francisco (9414290) (Figure
\ref{fig:coastline_map_with_inlets}a). Records indicate that the
arriving tsunami propogated generally North to South but waves reached
the more southern Monterey before San Francisco due to the effect
of the deep Monterey Canyon on tsunami celerity, consistent with previous
tsunami observations \citep{Gonzalez_ea_1995}. 


\section{Pre-tsunami Conditions}

Observations made immediately prior to the tsunami offer a framework
for understanding non-tsunami conditions in bar-built estuaries (Figure
\ref{fig:Station-DS-CTD_stack}a, c, e). The estuarine sea level is
modulated by ocean tides, but the constricted, shallow mouth limits
tidal amplitudes within the estuary. Tidal range the day prior to
the tsunami was approximately 75 cm within the Pescadero estuary compared
to twice that in the nearshore (Figure \ref{fig:Pescadero-stacked-variance_presreving_spectra}b,d).
Limited tidal elevation change is coupled with limited tidal velocities
in the estuary. Depth-averaged tidal velocities 400 m upstream from
the mouth were measured as limited to \textpm 20 cm s\textsuperscript{-1}.
On tidal timescales, the system behaves as a salt-wedge estuary, where
salt water is slowly advected upstream by flooding water. Despite
the shallow nature of this estuary, in the presence of limited velocities,
ocean water and freshwater in the system retain their salt character
due to limited mixing. Between 14:00 and 16:00 GMT on March 11 the
range of ebbing salinities at the DS sensor was 3 to 29 PSU between
50 cm and 200 cm depth (Figure \ref{fig:Station-DS-CTD_stack}c).
Completely fresh water was likely present at the surface above the
recording sensors. 


%--------------- figure 3 - u,s,d pretsunami and tsunam at DS -------------

\begin{figure}
\includegraphics{chapter5/figures/fig3.pdf}

\protect\caption{Depth-average velocity (a, b), CTD salinity (c, d) and CTD sensor
depth (e, f) at the DS mooring for the tidal cycles corresponding
mostly to pre-tsunami (a, c, e) and tsunami (b, d, f) conditions in
the Pescadero estuary from March 10 - 12, 2011. The first waves arrive
at 16:05 on March 11 (a, c, e). The deepest CTD sensor was mounted
20 cm above the bed. \label{fig:Station-DS-CTD_stack}}
\end{figure}

%------------ end figure 3 - u,s,d pretsunami and tsunam at DS -------------

%--------------- figure 2 - spectra of NOAA and PDO depths -------------

\begin{figure}
\textsf{\includegraphics{chapter5/figures/fig2.pdf}}

\protect\caption{Variance preserving spectra calculated from the NOAA Monterey (9413450)
tide gauge (a), (b) and from the US station in the Pescadero estuary
(c), (d).  The logrithmic color scale is one order of magnitude larger
on the NOAA data than on the Pescadero data.\label{fig:Pescadero-stacked-variance_presreving_spectra}}
\end{figure}
%--------------- end figure 2 - spectra of NOAA and PDO depths -----------

Unique to bar-built estuaries on wave-dominated coasts, ocean waves
significantly affect estuarine hydrodynamics. The tidal ranges in
choked lagoons are modulated by the magnitude of ocean waves {[}e.g.
\citealp{malhada_ea_wavesetup}{]}. We see evidence of an elevated
lagoon level due to large waves (Figure \ref{fig:Pescadero-stacked-variance_presreving_spectra}d)
in the pre-tsunami state of the Pescadero estuary. Most relevant to
the dynamics of these systems, oscillations corresponding to surf
beat frequencies are visible in pressure and velocity measurements
in this system. When the estuary and nearshore are sufficiently connected,
depending on tidal stage, mouth elevation, and ebb velocities, 30
s to 3 min oscillations are observed, notably changing the estuarine
sea level (Figure \ref{fig:Station-DS-CTD_stack}e) and inducing velocity
oscillations around the tidal velocities (Figure \ref{fig:Station-DS-CTD_stack}a).
Velocity oscillations between -10 cm s\textsuperscript{-1} and 50
cm s\textsuperscript{-1} were observed during the pre-tsunami high
high tide. 

Although the estuary remains salt stratified even in the presence
of high velocities induced by infragravity motions, wave forcing likely
affects the salinity structure in several ways: Oscillations in the
mid-column CTD salinity data result from the oscillatory flow advecting
the salt field upstream and downstream, and shearing and straining
of this salt field must induce some mixing. When wave forcing cuts
off on the ebb tide, the estuary appears to relax from a longitudinally
stratified system to a largely vertically stratified system. On the
ebb following the pre-tsunami high high tide, salt water remains trapped
in the deep pool due to low water velocities and bathymetric constraints
(after 12:00, Figure \ref{fig:Station-DS-CTD_stack}c). Typically,
this water is flushed from the salt trap low frequency pulses of water
begin to enter the estuary on the next flood tide, pushing downstream-sourced
fresh water into the salt pocket \citep{williams_stacey_inprep}. 


\section{Tsunami Forcing on Estuary}

The first waves from the Tohoku tsunami arrived to the pressure sensors
in the Pescadero estuary at 16:05 GMT on March 11, preceding arrival
of waves to San Francisco. Two twenty centimeter bore-shaped waves
are visible at 16:05 and 16:45 in the pressure record (Figure \ref{fig:Station-DS-CTD_stack}e),
but only slightly increase the depth of the shallow ebbing estuary.
The salinity record indicates that these two waves freshened the depths
of the estuary (Figure \ref{fig:Station-DS-CTD_stack}c), probably
as fresher downstream water is forced into the deep salt traps. Overall,
the low tide stage of the estuary and nearshore ocean restricted the
estuarine response to the early tsunami. 

As the tide rose following the onset of the tsunami, connection between
the estuary and the ocean was reestablished, and tsunami-frequency
motions begin to appear in the depth record (Figure \ref{fig:Pescadero-stacked-variance_presreving_spectra}c,d).
The low high tide on March 11-12 was characterized by larger depth
oscillations than common at this stage of the tide (Figure \ref{fig:Station-DS-CTD_stack}f).
Velocity measurements, though sparse, indicate a similar trend (Figure
\ref{fig:Station-DS-CTD_stack}b). Given freshwater inflow and tidal
stage, salt water was not present at the moorings during this period. 

Starting at 07:00 GMT on March 12, fast velocities are observed as
the flood tide couples with tsunami forcing. Maximum observed velocities
of 60 cm s\textsuperscript{-1} are comparable to the previous day's
maximum velocities, but occur much more frequently. The oscillation
between positive and negative velocities are also much stronger, and
a tidal velocity is not obvious, but present. The salt-wedge is advected
upstream, past the DS sensors, but also undergoes periods of intense
vertical mixing, eventually resulting in a reduced bed salinity of
24, compared to 29 on the previous day (Figure \ref{fig:Station-DS-CTD_stack}d). 



%------------------------ figure 4 - salinity, depth at DS -----------------
\begin{figure}
\includegraphics{chapter5/figures/fig4.pdf}

\protect\caption{CTD salinity (a) and CTD sensor depth (b) at the DS mooring. \label{fig:ctd_longer_record}}
\end{figure}
%------------------------ end figure 4 - salinity, depth at DS -----------------

As the tide recedes, the ocean and estuary become disconnected and
tsunami and surf beat forcing are immediately cut off around 13:15
GMT. At this point, the salt field begins to relax and be advected
downstream. Tsunami frequencies in the estuary are observed on the
following high tide, ultimately persisting for about two days (Figure
\ref{fig:Pescadero-stacked-variance_presreving_spectra}c), much less
time than the duration of tsunami frequencies at the Monterey tide
gauge of over five days (Figure \ref{fig:Pescadero-stacked-variance_presreving_spectra}a).
Some of this limit on tsunami forcing within the estuary is likely
due to geomorphic changes at the mouth. Depth data show clearly that
the mouth becomes nearly choked closed with sand on March 15 as the
lagoon depth remains high (Figure \ref{fig:ctd_longer_record}a).
This leads us to expect some geomorphic changes also occurred on days
prior to the near-closure. The unusually timed mouth closure allowed
the estuary to be converted into a salt-stratified lake with highly
limited tidal forcing (Figure \ref{fig:ctd_longer_record}). Rainfall
would force open the mouth on March 19, flushing all salt from the
system and ending the temporary alteration of the natural estuarine
state by the Tohoku tsunami. 


\section{Altered Estuarine Dynamics}

During the tsunami, kinetic energy was raised in the estuary as flood
and ebb conditions occurred with higher than normal frequency, and
associated velocities were elevated (Figure \ref{fig:kinetic_energy}).
We estimate kinetic energy associated with the large flood was raised
between 20\% and 40\% as compared with the previous tide, which already
existed in an elevated energy state due to large ocean waves. Vertical
mixing was shown to increase due to turbulence associated with these
elevated velocities (Figure \ref{fig:Station-DS-CTD_stack}b,d). Over
the tidal timescale, we expect that this increase in mixing resulted
in an overall decrease of salt intrusion into the estuary. Salt intrusion
occurs by a combination of advection and shear dispersion, but in
the presence of vertical mixing shear dispersion is reduced. Expecting
little difference in advection when pre-tsunami and tsunami high water
levels are similar, we attribute the expected overall reduction in
salt intrusion to reduced longitudinal dispersion caused by reduced
shear dispersion. 

%------------------------  figure 5 - kinetic energy  -----------------
\begin{figure}
\includegraphics{chapter5/figures/fig5.pdf}

\protect\caption{A comparison of kinetic energy basd on depth-averaged velocity for
a normal tidal cycle (a, c) and for tsunami conditions (b, d). \label{fig:kinetic_energy}}
\end{figure}

%------------------------ end figure 5 - kinetic energy  -----------------

The high velocities attributed to tsunami forcing were not much greater
than the fastest velocities commonly observed in the Pescadero estuary,
but tsunami velocities are persistant while during a normal tide these
high velocities are seen as relatively uncommon pulses. The constrained,
frictional mouth of California bar-built estuaries which normally
limits tidal velocities but allows infragravity wave-induced pulses
allows the energetic tsunami to increase the kinetic energy of the
high tide (Figure \ref{fig:kinetic_energy}c,d), while completely
cutting of tsunami forcing on the ebb. 

Observations in the Pescadero estuary during the March 2011 Tohoku
tsunami elucidate some effects of a small amplitude tsunami on a highly
stratified bar-built estuary. Similar forcing likely occurred in the
more than 200 small California estuaries as well as in similar
systems all along the Eastern Pacific coast of the Americas. 





