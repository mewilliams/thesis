\newcommand{\undefinedpagestyle}{} % added by Audric Collignon 06/08/2012 to get rid of an error message when compiling

\begin{abstract}

Shallow, bar-built estuaries on wave-dominated coasts in Mediterranean climates experience an intermittent connection to the ocean. Their inlets may completely close as a result of nearshore sand transport, but even the open condition these inlets remain constricted. Extensive field measurements in the highly salt-stratified Pescadero estuary in Northern California show that the shallow mouth causes these estuaries to experience discontinuous tidal forcing. While the ocean and estuary are fully connected with near equal water levels, tidal velocities are slow but infragravity motions in the nearshore cause large velocity oscillations. As the ocean tide falls, infragravity forcing is cut off because the estuarine mouth is perched above the low tide ocean water level, and ebbing velocities are set by bed friction. Observations reveal this oscillation between ocean-forced conditions and frictionally-controlled conditions which characterizes and sets the hydrodynamics and salt dynamics in these estuaries. Additional wave setup of the lagoon emphasizes the dependence of these estuaries on nearshore ocean conditions, but the diurnal or semidiurnal retreat of the ocean below the mouth cuts off this nearshore influence. The salt field responds to this discontinuous forcing, being transported upstream on the flood and becoming trapped in deep pools of the estuary on ebb. Here we present unique detailed observations and a framework for understanding dynamics in small, shallow bar-built estuaries. 


The March 2011 Tohoku earthquake offshore of Japan generated a tsunami that was observed across the entire Pacific Ocean. Hydrodynamic observations were being made in the Pescadero estuary of Northern California at the time of the tsunami. Observations show that the presence of a tsunami-period signature in this shallow, bar-built estuary was modulated by tidal stage and that the tsunami-induced rise and fall of the ocean water level modulated the velocities seen within the estuary. The initial arrival of waves due to the tsunami caused some flushing of the estuary. Later, high velocities induced more mixing than typically seen in these estuaries and after several days sand transport nearly choked closed the mouth of the estuary. These observations point to mechanisms which were likely important during the 2011 Tohoku tsunami
in many Eastern Pacific estuaries. 

\end{abstract}